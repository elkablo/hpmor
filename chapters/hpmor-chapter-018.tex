\chapter{Hackordnung}

\emph{„Das hört sich wie etwas an, was ich tun würde, nicht wahr?“} 

\later 

\lettrine{E}{s} war Frühstückszeit am Freitag Morgen. Harry biss noch ein großes Stück von seinem Toast ab und versuchte dann, sein Gehirn daran zu erinnern, dass er nicht schneller in die Kerker kommen würde, wenn er sein Frühstück runterschlang. Zwischen dem Frühstück und dem Beginn des Zaubertränke-Unterrichts hatten sie noch eine Freistunde. 

Aber Kerker! In Hogwarts! Harrys Vorstellungskraft malte sich schon die Abgründe aus; die engen Brücken, Fackeln an den Wänden und Flecken von leuchtendem Moos. Ob es dort Ratten gab? Ob es dort \emph{Drachen} gab? 

„Harry Potter“, sagte eine leise Stimme hinter ihm. 

Harry sah über die Schulter und erblickte Ernie Macmillan, der einen eleganten, gelb gesäumten Umhang trug und etwas besorgt aussah. 

„Neville meinte, ich solle dich warnen“, sagte Ernie leise. „Ich glaube, er hat Recht. Sei vorsichtig, wenn du dem Zaubertränkemeister heute im Unterricht begegnest. Die älteren Hufflepuffs haben uns erzählt, dass Professor Snape sich gegenüber Leuten, die er nicht mag, wirklich unangenehm verhält; und er mag kaum jemanden, der kein Slytherin ist. Wenn du dich neunmalklug verhältst … es könnte wirklich schlimm für dich enden, soweit ich gehört habe. Versuche einfach nicht aufzufallen und gib ihm keinen Grund, es auf dich abzusehen.“ 

Es war einen Moment lang still, während Harry das verarbeitete, dann hob er die Augenbrauen. (Harry hätte gerne nur eine Augenbraue gehoben, wie Spock, doch er konnte das nicht.) „Danke“, sagte Harry. „Du hast mir vielleicht gerade eine Menge Ärger erspart.“

Ernie nickte und drehte sich weg, um zurück zum Hufflepuff-Tisch zu gehen. 

Harry aß seinen Toast weiter. 

Ungefähr vier Bissen später sagte jemand „Entschuldige bitte“, Harry drehte sich um und sah einen älteren Ravenclaw, der etwas besorgt aussah. 

Etwas später aß Harry sein drittes Schinkenbrot auf. (Er hatte gelernt, zum Frühstück viel zu essen. Er konnte beim Mittagessen immer noch etwas weniger essen, wenn er den Zeitumkehrer bis dahin nicht benutzt hatte.) Dann erklang erneut eine Stimme hinter ihm: „Harry?“ 

„Ja“, sagte Harry müde, „ich werde versuchen, Professor Snapes Aufmerksamkeit nicht auf mich zu ziehen –“ 

„Oh, das ist aussichtslos“, sagte Fred. 

„Völlig aussichtslos“, sagte George. 

„Darum haben wir die Hauselfen gebeten, einen Kuchen für dich zu backen“, sagte Fred. 

„Für jeden Punkt, der Ravenclaw deinetwegen abgezogen wird, werden wir eine Kerze daraufstellen“, sagte George. 

„Und beim Mittagessen werden wir am Gryffindor-Tisch eine Party für dich schmeißen“, sagte Fred. 

„Wir hoffen, dass dich das wieder aufmuntert“, schloss George. 

Harry schluckte den letzten Bissen vom Schinkenbrot runter und drehte sich um. „Schon gut“, sagte Harry. „Nachdem ich Professor Binns erlebt hatte, wollte ich die Frage gar nicht stellen, wirklich nicht, aber … wenn Professor Snape wirklich \emph{so} schrecklich ist, warum wurde er dann noch nicht gefeuert?“ 

„Gefeuert?“, sagte Fred. 

„Du meinst, entlassen?“, sagte George. 

„Ja“, sagte Harry. „Das macht man mit schlechten Lehrern. Man feuert sie. Dann stellt man stattdessen einen besseren Lehrer ein. Hier gibt es doch keine Gewerkschaften oder Beamten, nicht wahr?“ 

Fred und George starrten ihn etwa so an, wie ein Stamm von Jägern und Sammlern schauen würde, wenn man ihnen von Integralrechnung erzählte. 

„Ich weiß es nicht“, sagte Fred nach einer Weile. „Ich habe nie darüber nachgedacht.“ 

„Ich auch nicht“, sagte George. 

„Ja“, sagte Harry, „das höre ich öfter. Wir sehen uns beim Mittagessen, Jungs, und wenn auf dem Kuchen keine Kerzen stehen, dann sagt nicht, ich hätte euch nicht gewarnt.“ 

Fred und George lachten, als ob Harry etwas Lustiges gesagt hätte, nickten ihm dann zu und gingen zurück zu den Gryffindors. 

Harry wandte sich wieder dem Frühstückstisch zu und griff nach einem Cupcake. Sein Magen fühlte sich schon voll an, doch er hatte das Gefühl, dass er an diesem Vormittag viele Kalorien gebrauchen könnte. 

Während er seinen Cupcake aß, dachte Harry an den schlechtesten Lehrer, den er bisher getroffen hatte: Professor Binns, der Geschichte lehrte. Professor Binns war ein Geist. Von dem, was Hermine über Geister erzählt hatte, erschien es unwahrscheinlich, dass sie bei vollem Bewusstsein waren. Geister hatten keine neuen Entdeckungen gemacht oder überhaupt etwas Neues geschaffen, egal was sie zu Lebzeiten getan hatten. Geistern fiel es oft schwer, sich daran zu erinnern, in welchem Jahrhundert sie sich gerade befanden. Hermine hatte gesagt, dass sie wie versehentliche Porträts waren, die beim plötzlichen Tod eines Zauberers von einem Ausbruch von psychischer Energie in die Umgebung projiziert wurden. 

Harry waren während seiner abgebrochenen Schullaufbahn ein paar dumme Lehrer begegnet – als sein Vater dann Studenten als Tutoren angestellt hatte, hatte er sie natürlich sehr viel sorgfältiger ausgesucht –, aber im Geschichtsunterricht war ihm zum ersten Mal ein Lehrer begegnet, der wortwörtlich nicht bei Bewusstsein war. 

Und man merkte es. Harry hatte es nach fünf Minuten aufgegeben und begonnen, ein Lehrbuch zu lesen. Als klar war, dass „Professor Binns“ sich daran nicht störte, hatte Harry sogar in seinen Beutel gegriffen und Ohrenstöpsel herausgeholt. 

Bekamen Geister kein Gehalt? Lag es daran? Oder war es tatsächlich unmöglich, einen Hogwarts-Lehrer zu feuern, \emph{selbst wenn er starb?} 

Es schien, dass Professor Snape sich gegenüber jedem, der kein Slytherin war, vollkommen abscheulich verhielt, doch trotzdem war niemand auch nur \emph{auf die Idee gekommen}, seinen Vertrag zu kündigen. 

Und der Schulleiter hatte ein Hühnchen in Brand gesteckt. 

„Entschuldige bitte“, sagte eine besorgte Stimme hinter ihm. 

„Also wirklich“, sagte Harry, ohne sich umzudrehen, „diese Einrichtung ist fast achteinhalb Prozent so schlimm wie das, was Dad über Oxford erzählt.“ 

\later 

Harry stampfte durch den steinernen Flur und sah zugleich beleidigt, genervt und verärgert aus. 

„Kerker!“, zischte Harry. „\emph{Kerker!} Das ist kein Kerker! Das ist ein Keller! Ein \emph{Keller}!“ 

Einige Ravenclaw-Mädchen warfen ihm komische Blicke zu. Die Jungs hatten sich inzwischen an ihn gewöhnt. 

Das Geschoss, in dem das Zaubertränke-Klassenzimmer sich befand, wurde offenbar deswegen „Kerker“ genannt, weil es unterirdisch lag und etwas kühler als der Rest des Schlosses war. 

Auf \emph{Hogwarts}! Auf \emph{Hogwarts}! Harry hatte sein ganzes Leben lang gewartet und wartete nun \emph{immer noch}; wenn es \emph{irgendwo auf der Erde} vernünftige Kerker gäbe, dann doch wohl auf Hogwarts! Musste Harry sein eigenes Schloss bauen, bloß weil er einen klitzekleinen bodenlosen Abgrund sehen wollte? 

Kurz darauf kamen sie im Zaubertränke-Klassenzimmer an und Harrys Laune besserte sich deutlich. 

Im Klassenzimmer schwebten seltsame Kreaturen in großen Einmachgläsern, die auf Regalen aufgereiht jeden Zentimeter Wand zwischen den großen Schränken verdeckten. Harry hatte bereits weit genug gelesen, dass er einige der Wesen erkannte, etwa die zabriskische Fontema. Wenngleich die fünfzig Zentimeter große Spinne wie eine Acromantula \emph{aussah}, war sie doch zu klein, um eine zu \emph{sein}. Er wollte Hermine fragen, doch sie schien nicht daran interessiert, in die Richtung zu schauen, in die er zeigte. 

Harry betrachtete gerade eine große Staubkugel mit Augen und Füßen, als der Assassin in den Raum eilte. 

Dies war der erste Gedanke, der Harry in den Kopf kam, als er Professor Severus Snape sah. Es hatte etwas Ruhiges, Tödliches an sich, wie der Mann durch die Reihen der Tische schritt. Sein Umhang war unsauber, sein Haar befleckt und fettig. Etwas an ihm erinnerte an Lucius, obwohl die beiden sich kein bisschen ähnlich sahen, und man bekam den Eindruck, dass Lucius mit makelloser Eleganz töten würde, während dieser Mann schlicht töten würde. 

„Setzt euch“, sagte Professor Severus Snape. „Jetzt.“ 

Harry und einige andere Kinder, die dagestanden und miteinander geredet hatten, eilten zu den Tischen. Harry hatte eigentlich vorgehabt, neben Hermine zu sitzen, doch irgendwie fand er sich am nächsten freien Tisch neben Justin Finch-Fletchley wieder (Ravenclaw und Hufflepuff hatten den Unterricht gemeinsam), zwei Tische links von Hermine. 

Severus setzte sich an den Lehrertisch und sagte, ohne Unterbrechung oder Vorstellung, „Hannah Abbott.“ 

„Hier“, sagte Hannah in einer leicht zitternden Stimme. 

„Susan Bones.“ 

„Anwesend.“ 

Und so ging es weiter, niemand wagte es, mehr zu sagen, bis: 

„Ah, ja. Harry Potter. Unsere neue … \emph{Berühmtheit}.“ 

„Die Berühmtheit ist anwesend, \emph{Sir}.“ 

Die halbe Klasse zuckte zusammen und einige der klügeren Schüler sahen plötzlich so aus, als ob sie rausrennen wollten, solange das Klassenzimmer noch da war. 

Severus lächelte erwartungsfroh und rief den nächsten Namen auf der Liste auf. 

Harry seufzte innerlich. Das war viel zu schnell geschehen, als dass er etwas hätte tun können. Ach ja. Offenbar mochte dieser Mann ihn aus irgendeinem Grund nicht. Und, wo Harry darüber nachdachte, es war sehr viel besser, wenn dieser Zaubertränke-Lehrer es auf \emph{ihn} abgesehen hätte, statt etwa auf Neville oder Hermine. Harry war sehr viel besser dazu in der Lage, sich selbst zu verteidigen. Ja, so war es wohl am besten. 

Als er die Anwesenheit geprüft hatte, ließ Severus seinen Blick über die Schüler schweifen. Seine Augen waren so leer wie ein sternenloser Nachthimmel. 

„Ihr seid hier“, sagte Severus in einer leisen Stimme, so dass die hinten sitzenden Schüler Mühe hatten, ihn zu verstehen, „um die subtile Wissenschaft und exakte Kunst der Zaubertrankbrauerei zu lernen. Da es hier wenig albernes Zauberstabgefuchtel gibt, werden viele von euch kaum glauben, dass es sich um Zauberei handelt. Ich erwarte nicht, dass ihr tatsächlich versteht, welche Schönheit dem sanft brodelnden Kessel und den schimmernden Dämpfen innewohnt; welch feine Macht in den Flüssigkeiten steckt, die durch menschliche Adern kriechen“, all das in einem fast liebevollen, eifrigen Ton, „die den Verstand bezaubern, die Sinne verführen“, es wurde immer unheimlicher. „Ich kann euch lehren, wie man Ruhm in Flaschen abfüllt, Ansehen zusammenbraut, sogar den Tod verkorkt – sofern ihr kein so großer Haufen Dummköpfe seid, wie ich sie sonst immer unterrichten muss.“ 

Severus schien den skeptischen Blick von Harry irgendwie zu spüren; zumindest richtete er seine Augen plötzlich auf Harrys Platz. 

„Potter!“, sagte der Zaubertränke-Lehrer plötzlich. „Was würde ich erhalten, wenn ich einem Wermutaufguss geriebene Affodillwurzel hinzufüge?“ 

Harry blinzelte. „Stand das in \emph{Zaubertränke und Zauberbräue}?“, sagte er. „Ich habe es gerade fertig gelesen und ich kann mich an kein Rezept erinnern, wo Wermut–“ 

Hermines Hand schoss hoch und Harry warf ihr einen düsteren Blick zu, worauf sie ihre Hand noch höher hob. 

„Tja, tja“, sagte Severus seidig. „Ruhm ist nunmal nicht alles.“ 

„Tatsächlich?“, sagte Harry. „Aber Sie haben uns gerade erzählt, dass Sie uns beibringen werden, wie man Ruhm in Flaschen abfüllt. Sagen Sie, \emph{wie} genau funktioniert das? Man trinkt ihn und wird zu einer Berühmtheit?“ 

Drei Viertel der Klasse zuckten zusammen. 

Hermines Hand sank langsam. Nun, das war nicht überraschend. Sie mochte seine Rivalin sein, doch sie war niemand, der mitspielen würde, wenn klar war, dass der Lehrer bewusst versuchte, ihn bloßzustellen. 

Harry gab sich Mühe, sein Temperament zu zügeln. Die erste Antwort, die ihm in den Sinn gekommen war, war „Abrakadabra“ gewesen. 

„Versuchen wir’s nochmal“, sagte Severus. „Potter, wo würdest du nachschauen, wenn ich dir sagte, du solltest einen Bezoar finden?“ 

„Das steht auch nicht im Lehrbuch“, sagte Harry, „aber in einem Muggelbuch habe ich gelesen, dass ein Trichobezoar eine Knäuel aus verfestigtem Haar ist, das in einem menschlichen Magen gefunden wird, und dass die Muggel glaubten, es könne jede Vergiftung heilen –“ 

„Falsch“, sagte Severus. „Einen Bezoar findet man im Magen einer Ziege, er besteht nicht aus Haar und er kann viele, aber nicht alle Vergiftungen heilen.“ 

„Ich \emph{sagte} nicht, dass er das kann, ich sagte, dass ich das in einem Muggelbuch gelesen hatte –“ 

„Niemand hier interessiert sich für deine \emph{jämmerlichen} Muggelbücher. Letzter Versuch. Was, Potter, ist der Unterschied zwischen Eisenhut und Wolfswurz?“ 

Das war’s. 

„Wissen Sie“, sagte Harry eiskalt, „in einem meiner äußerst \emph{faszinierenden} Muggelbücher wird eine Studie beschrieben, in der Leute es schafften, sehr klug auszusehen, indem sie Fragen über irgendwelche Dinge stellten, die nur sie wussten. Offenbar bemerkten die Zuschauer nur, dass die Fragesteller es wussten und die Antwortenden nicht, aber sie konnten die Unfairness der zugrundeliegenden Situation nicht kompensieren. Also, Professor, können Sie mir sagen, wie viele Elektronen sich im äußersten Orbital eines Kohlenstoffatoms befindet?“ 

Severus’ Lächeln wurde breiter. „Vier“, sagte er. „Das ist jedoch ein nutzloser Fakt, den niemand aufschreiben sollte. Zu deiner Information, Potter, Affodill und Wermut ergeben einen Schlaftrank, der so mächtig ist, dass er als Trank der Lebenden Toten bekannt ist. Was Eisenhut und Wolfswurz angeht, so sind sie die selbe Pflanze, die auch unter dem Namen Aconitum bekannt ist, wie du gewusst hättest, wenn du \emph{Tausend Zauberkräuter und -pilze} gelesen hättest. Dachtest wohl, dass du kein Buch anrühren müsstest, bis du hier bist, was, Potter? Ihr Anderen solltet das aufschreiben, damit ihr nicht so unwissend bleibt wie er.“ Severus wartete und sah sehr zufrieden mit sich selbst aus. „Das macht dann … fünf Punkte? Nein, sagen wir zehn Punkte Abzug von Ravenclaw wegen Widerworten.“ 

Hermine keuchte, ebenso wie einige Andere. 

„Professor Severus Snape“, presste Harry hervor. „Ich weiß nicht, womit ich Ihre Feindseligkeit verdient habe. Wenn Sie irgendein Problem mit mir haben, von dem ich nichts weiß, dann schlage ich vor –“ 

„Sei leise, Potter. Nochmal zehn Punkte Abzug von Ravenclaw. Ihr Anderen, öffnet das Buch auf Seite 3.“ 

Harry spürte nur ein leichtes, sehr schwaches Brennen in der Kehle und hatte keine Tränen in den Augen. Wenn weinen keine wirksame Strategie war, diesen Zaubertränke-Lehrer zu vernichten, dann war weinen nutzlos. 

Langsam setzte Harry sich sehr aufrecht hin. Sein ganzes Blut schien verschwunden und durch flüssigen Stickstoff ersetzt zu sein. Er wusste, dass er sein Temperament zügeln wollte, doch es fiel ihm nicht mehr ein, weshalb. 

„Harry“, flüsterte Hermine zwei Tische neben ihm eifrig, „hör auf, bitte, es ist okay, das zählt nicht –“ 

„Im Unterricht reden, Granger? Drei –“ 

„Also“, sagte eine Stimme, die kälter als null Kelvin war, „wie reicht man eine offizielle Beschwerde gegen Machtmissbrauch eines Lehrers ein? Spricht man mit der Stellvertretenden Schulleiterin, schreibt man einen Brief an die Schulräte … würden Sie mir erklären, wie das geht?“ 

Die Schüler waren vollkommen erstarrt. 

„Einen Monat Strafarbeiten, Potter“, sagte Severus und lächelte noch breiter. 

„Ich weigere mich, Ihre Autorität als Lehrer anzuerkennen und werde keine Strafarbeit ableisten, die Sie mir geben.“ 

Die Anderen hielten den Atem an. 

Severus’ Lächeln verschwand. „Dann wirst du –“, seine Stimme brach ab. 

„Rausgeschmissen, wollten Sie sagen?“ Harry hingegen lächelte nun schmal. „Aber dann schienen Sie an Ihrer Fähigkeit zu zweifeln, die Drohung auszuführen; oder Sie fürchten die Konsequenzen, wenn es gelänge. Ich hingegen habe weder Zweifel noch Angst davor, dass ich eine Schule mit Lehrern finde, die ihre Position nicht missbrauchen. Oder vielleicht sollte ich eigene Tutoren anstellen, wie ich es gewohnt bin, um in meiner vollen Geschwindigkeit zu lernen. Ich habe genug Geld in meinem Verlies. Irgendwas mit Kopfgeld, das auf einen Dunklen Lord ausgesetzt war, den ich besiegt habe. Doch es \emph{gibt} Lehrer auf Hogwarts, die ich recht gern habe, also wäre es, denke ich, einfacher, einen Weg zu finden, wie ich Sie stattdessen loswerde.“ 

„Mich loswerden?“, sagte Severus, der nun auch ein schmales Lächeln trug. „Welch amüsante Selbstüberschätzung. Wie glaubst du, wirst du das anstellen, Potter?“ 

„Soweit ich weiß, gab es reichlich Beschwerden von Schülern und Eltern über Ihr Verhalten“, das war geraten, aber sehr wahrscheinlich, „also bleibt nur die Frage, warum Sie nicht längst verschwunden sind. Ist die finanzielle Lage von Hogwarts so schlecht, dass die Schule sich keinen richtigen Zaubertränke-Lehrer leisten kann? Falls es das ist, könnte ich etwas dazugeben. Ich bin mir sicher, dass die Schule einen besseren Lehrer finden könnte, wenn sie das Doppelte Ihres derzeitigen Gehalts anbietet.“ 

Zwei eisige Punkte schickten bittere Kälte durchs Klassenzimmer. 

„Du wirst feststellen“, sagte Severus sanft, „dass die Schulräte keinerlei Interesse an deinem Angebot haben.“ 

„Lucius …“, sagte Harry. „\emph{Deswegen} sind Sie noch hier. Vielleicht sollte ich mit Lucius darüber reden. Ich glaube, er möchte sich mit mir treffen. Ich frage mich, ob ich irgendetwas besitze, das er will?“ 

Hermine schüttelte wild den Kopf. Harry bemerkte es aus dem Augenwinkel, doch seine Aufmerksamkeit lag alleine auf Severus. 

„Du bist ein sehr dummer Junge“, sagte Severus. Er lächelte nun gar nicht mehr. „Du hast nichts, das Lucius höher schätzt als meine Freundschaft. Und wenn du so etwas hättest, habe ich andere Verbündete.“ Seine Stimme wurde hart. „Und ich finde es immer unglaubwürdiger, dass du nicht nach Slytherin gekommen bist. Wie hast du es geschafft, nicht in mein Haus zu kommen? Ach, genau, der Sprechende Hut behauptete, dass es ein \emph{Scherz} war. Zum ersten Mal in der Geschichte der Schule. Worüber hast du wirklich mit dem Sprechenden Hut \emph{geplaudert}, Potter? Hattest du etwas, das er wollte?“ 

Harry erwiderte Severus’ kalten Blick und ihm fiel ein, dass der Sprechende Hut ihn gewarnt hatte, niemandem in die Augen zu sehen, während er an – Harry senkte seinen Blick auf Severus’ Tisch. 

„Sie wollen mir nicht in die Augen sehen, Potter!“ 

Plötzlich verstand er – „Also hat der Sprechende Hut mich vor \emph{Ihnen} gewarnt!“ 

„Was?“, sagte Severus, der aufrichtig überrascht wirkte, obwohl Harry ihm natürlich nicht ins Gesicht sah. 

Harry stand von seinem Platz auf. 

„Setz’ dich, Potter“, sagte eine wütende Stimme aus irgendeiner Richtung, in die er nicht blickte. 

Harry ignorierte es und sah sich im Klassenzimmer um. „Ich habe nicht vor, mir von einem unprofessionellen Lehrer meine Zeit in Hogwarts verderben zu lassen“, sagte Harry mit tödlicher Ruhe. „Ich glaube, ich werde diesen Unterricht verlassen und entweder einen Tutor beschäftigen, der mir Zaubertränke beibringt, während ich hier bin, oder – falls die Schulräte tatsächlich so uneinsichtig sind – es über den Sommer lernen. Wenn jemand von euch beschließt, dass er nicht von diesem Mann schikaniert werden will, dann steht mein Unterricht euch offen.“ 

\emph{„Setz’ dich, Potter!“} 

Harry ging durch den Raum und griff nach dem Türknauf. 

Er ließ sich nicht drehen. 

Harry drehte sich langsam um und erblickte Severus’ hässliches Lächeln, bevor er daran dachte wegzuschauen. 

„Öffnen Sie diese Tür.“ 

„Nein“, sagte Severus. 

„Dann fühle ich mich bedroht“, sagte eine Stimme, die so eisig war, dass sie gar nicht nach Harry klang, „und das ist ein Fehler.“ 

Severus’ Stimme lachte. „Was willst du dagegen tun, kleiner Junge?“ 

Harry ging sechs große Schritte von der Tür weg, bis er neben der hintersten Tischreihe stand. 

Dann richtete Harry sich auf und hob mit einer einzigen schrecklichen Bewegung die rechte Hand, die Finger bereit zum Schnippen. 

Neville schrie auf und tauchte unter seinem Tisch ab. Andere Kinder zuckten weg oder rissen instinktiv die Arme hoch, um sich zu schützen. 

\emph{„Harry, nicht!“}, schrie Hermine. „Was auch immer du ihm antun wolltest, tu es nicht!“ 

„Seid ihr alle \emph{verrückt} geworden?“, blaffte Severus’ Stimme. 

Langsam senkte Harry seine Hand. „Ich wollte ihn nicht verletzen, Hermine“, sagte Harry, mit etwas ruhigerer Stimme. „Ich wollte nur die Tür aufsprengen.“ 

Aber jetzt, wo Harry darüber nachdachte – man durfte nichts in Dinge verwandeln, die dann verbrannt werden sollten, also wäre es wohl keine gute Idee, später durch die Zeit zurück zu reisen und Fred und George zu überreden, irgendwas in eine sorgfältig abgemessene Menge Sprengstoff zu verwandeln … 

\emph{„Silencio“}, sagte Severus’ Stimme. 

Harry versuchte, „Was?“ zu sagen, und stellte fest, dass kein Ton herauskam. 

„Das ist lächerlich geworden. Ich glaube, du hast dir für heute genug Ärger eingebrockt, Potter. Du bist der störendste und aufsässigste Schüler, den ich jemals gesehen habe, und ich weiß nicht, wie viele Punkte Ravenclaw derzeit hat, aber ich bin mir sicher, dass es mir gelingen wird, sie alle auszulöschen. Zehn Punkte Abzug von Ravenclaw. Zehn Punkte Abzug von Ravenclaw. Zehn Punkte Abzug von Ravenclaw! Fünfzig Punkte Abzug von Ravenclaw! Jetzt setz’ dich hin und schau der Klasse beim Unterricht zu!“ 

Harry griff mit der Hand in seinen Beutel und versuchte, „Stift“ zu sagen, doch er brachte natürlich keinen Ton hervor. Einen kurzen Augenblick lang hielt ihn das auf, dann kam Harry auf die Idee, mit dem Finger S-T-I-F-T zu malen und es gelang. B-L-O-C-K und er hatte einen Block Papier. Harry ging zu einem unbesetzten Tisch rüber, nicht dem, an dem er ursprünglich gesessen hatte, und kritzelte eine kurze Nachricht. Er riss das Stück Papier ab, steckte Block und Stift in eine Tasche seines Umhangs, um schneller wieder darauf zugreifen zu können, und hielt seine Botschaft hoch, die sich nicht an Snape, sondern an die anderen Schüler richtete. 

\begin{writtenNote}
ICH GEHE.\\
WILL SONST JEMAND\\
MIT RAUS KOMMEN?
\end{writtenNote}

„Du bist verrückt, Potter“, sagte Severus mit kalter Verachtung. 

Außer ihm sprach niemand. 

Harry verbeugte sich spöttisch in Richtung des Lehrers, ging zur Wand und öffnete mit einer fließenden Bewegung eine Schranktür, trat hinein und warf die Tür hinter sich zu. 

Das gedämpfte Geräusch schnippender Finger erklang, dann war es still. 

Im Klassenzimmer sahen die Schüler einander verwirrt und verängstigt an. 

Das Gesicht des Zaubertränke-Lehrers war nun wutverzerrt. Er durchkreuzte den Raum mit drohenden Schritten und zerrte die Schranktür auf. 

Der Schrank war leer. 

\later 

Eine Stunde früher befand Harry sich im geschlossenen Schrank und lauschte. Er hörte keine Geräusche von draußen, wollte aber kein Risiko eingehen. 

U-M-H-A-N-G buchstabierte sein Finger. 

Sobald er unsichtbar war, öffnete er die Schranktür sehr langsam und vorsichtig einen Spalt breit und guckte heraus. Niemand schien im Klassenzimmer zu sein. 

Die Tür war nicht verschlossen. 

Als Harry diesen gefährlichen Ort hinter sich gelassen hatte und in den Gängen unsichtbar in Sicherheit war, verrauchte ein Teil des Ärgers und ihm wurde klar, was er gerade getan hatte. 

Was er gerade getan hatte. 

Harrys unsichtbares Gesicht war schreckensverzerrt. 

Er hatte sich einen Lehrer zum Feind gemacht – um Größenordnungen schlimmer als jemals zuvor. Er hatte damit gedroht, Hogwarts zu verlassen, und würde diese Drohung vielleicht wahrmachen müssen. Er hatte alle Hauspunkte verloren, die Ravenclaw gehabt hatte und er hatte den Zeitumkehrer benutzt … 

Vor seinem inneren Auge sah er, wie seine Eltern ihn anschrien, nachdem er von der Schule geschmissen wurde; er sah Professor McGonagall, die von ihm enttäuscht war. Es war einfach zu schmerzvoll und er konnte es nicht ertragen und er \emph{wusste nicht, wie er das verhindern konnte} – 

Dann ließ Harry einen Gedanken zu: Wenn seine Wut ihm all diesen Ärger eingebrockt hatte, dann konnte die Wut ihm vielleicht helfen, einen Ausweg zu finden; die Dinge erschienen ihm klarer, wenn er wütend war. 

Und Harry verbot sich den Gedanken, dass er der Zukunft einfach nicht entgegen treten konnte, wenn er nicht wütend war. 

Also dachte er zurück, erinnerte sich an die schreckliche Demütigung – 

\emph{Tja, tja. Ruhm ist nunmal nicht alles.}

\emph{Zehn Punkte Abzug von Ravenclaw wegen Widerworten.}

Die beruhigende Kälte floß wieder durch seine Adern, wie eine Welle, die von einer Mauer zurückgeworfen wurde, und Harry atmete tief aus. 

Okay. Jetzt konnte er wieder klar denken. 

Er war tatsächlich etwas enttäuscht von seinem nicht-wütenden Selbst, weil es einfach so zusammengebrochen war und nur dem Ärger entgehen wollte. Professor Severus Snape war für \emph{alle} ein Problem. Normal-Harry hatte das vergessen und nach einer Möglichkeit gesucht, \emph{sich selbst} zu schützen. Und alle anderen Opfer zu verraten? Die Frage lautete nicht, wie er sich selbst schützen konnte; die Frage lautete, wie er diesen Zaubertränke-Lehrer vernichten konnte. 

\emph{Das ist also meine dunkle Seite, ja? Ein ganz schön belasteter Begriff; meine helle Seite scheint viel eigennütziger und ängstlicher zu sein, und verwirrter und panischer noch dazu.} 

Und jetzt, wo er klar denken konnte, war ihm ebenso klar, was als Nächstes zu tun war. Er hatte sich schon eine Stunde zur Vorbereitung verschafft und konnte fünf weitere Stunden bekommen, falls nötig … 

\later 

Minerva McGonagall wartete im Büro des Schulleiters. 

Dumbledore saß in seinem gepolsterten Thron hinter seinem Schreibtisch, gekleidet in vier übereinanderliegende, förmliche, lavenderfarbene Umhänge. Minerva saß schräg vor ihm auf einem Stuhl, Severus auf einem weiteren Stuhl auf der anderen Seite. Ihnen gegenüber stand ein leerer hölzerner Stuhl. 

Sie warteten auf Harry Potter. 

\emph{Harry,} dachte Minerva verzweifelt, \emph{du hast versprochen, dass du keine Lehrer beißen würdest!} 

Und im Kopf sah sie die Reaktion klar vor sich; Harrys wutverzerrtes Gesicht und seine empörte Antwort: \emph{Ich sagte, dass ich niemanden beiße, der mich nicht zuerst beißt!} 

Jemand klopfte an die Tür. 

„Herein!“, rief Dumbledore. 

Die Tür schwang auf und Harry Potter trat ein. Minerva hätte fast laut aufgekeucht. Der Junge sah kühl und kontrolliert aus, er hatte sich vollkommen unter Kontrolle. 

„Guten Mor–“ Harrys Stimme brach plötzlich ab. Sein Mund stand offen. 

Minerva verfolgte Harrys Blick und sah, dass Harry den Phönix Fawkes auf seiner goldenen Sitzstange anstarrte. Fawkes flatterte mit seinen rot-goldenen Flügeln wie eine aufflackernde Flamme, senkte seinen Kopf und nickte dem Jungen gemessen zu. 

Harry wandte sich zu Dumbledore. 

Dumbledore zwinkerte ihm zu. 

Minerva hatte das Gefühl, dass ihr etwas entging. 

Plötzlich huschte ein unsicherer Ausdruck über Harrys Gesicht. Seine Kontrolle schwand. Angst stand in seinen Augen, dann Wut, und dann war der Junge wieder ruhig. 

Minerva lief ein kalter Schauer den Rücken hinunter. Irgendwas ging hier nicht mit rechten Dingen zu. 

„Bitte, setz dich“, sagte Dumbledore. Sein Gesichtsausdruck war nun wieder ernst. 

Harry setzte sich. 

„Also, Harry“, sagte Dumbledore. „Ich habe einen Bericht von Professor Snape gehört. Möchtest du mir in deinen eigenen Worten sagen, was passiert ist?“ 

Harrys Blick blinkte abschätzig zu Severus. „Es ist nicht kompliziert“, sagte der Junge mit einem schmalen Lächeln. „Er hat versucht, mich so zu terrorisieren, wie er es mit jedem nicht-Slytherin an dieser Schule getan hat, seit Lucius ihn der Schule aufgezwungen hat. Was die weiteren Details angeht, verlange ich ein Vieraugengespräch mit Ihnen. Man kann von einem Schüler, der Misshandlungen eines Lehrers meldet, schließlich kaum erwarten, dass er vor ebenjenem Lehrer offen spricht.“ 

Diesmal konnte Minerva nicht an sich halten und keuchte auf. 

Severus lachte einfach. 

Und der Gesichtsausdruck des Schulleiters wurde ernster. „Mr~Potter“, sagte der Schulleiter, „so spricht man nicht über einen Hogwarts-Lehrer. Ich fürchte, dass Sie einem schweren Missverständnis unterliegen. Professor Severus Snape hat mein volles Vertrauen und arbeitet auf meinen Wunsch hin auf Hogwarts, nicht auf Wunsch von Lucius Malfoy. 

Einen Moment lang war es still. 

Als der Junge wieder sprach, war seine Stimme eiskalt. „Übersehe ich hier etwas?“ 

„Einige Dinge, Mr~Potter“, sagte der Schulleiter. „Als erstes sollten Sie sich bewusst machen, dass wir hier besprechen wollen, welche Disziplinarmaßnahmen wir wegen Ihres Verhaltens heute Morgen ergreifen werden.“ 

„Dieser Mann hat Ihre Schule seit Jahren in Angst und Schrecken versetzt. Ich habe mit Schülern gesprochen und genug Geschichten gesammelt, damit eine Zeitungskampagne die Eltern gegen ihn aufbringen kann. Einige der jüngeren Schüler haben geweint, während sie mir davon erzählten. Ich habe fast geweint, als ich das hörte! \emph{Sie haben zugelassen, dass} diese Person \emph{hier frei herumläuft? Sie haben Ihren Schülern das angetan? Warum?}“ 

Minerva hatte einen Kloß in der Kehle. Sie hatte – das auch gedacht, manchmal, aber irgendwie war sie nie – 

„Mr~Potter“, sagte der Schulleiter, nun mit strenger Stimme, „es geht hier nicht um Professor Snape. Es geht um Sie und Ihre Missachtung der Schulregeln. Professor Snape hat vorgeschlagen, und ich stimme dem zu, dass eine drei Monate dauernde Strafarbeit angemessen –“ 

„Abgelehnt“, sagte Harry eiskalt. 

Minerva war sprachlos. 

„Das ist keine Bitte, Mr~Potter“, sagte der Schulleiter. Die gesamte Macht seines Blickes richtete sich auf den Jungen. „Das ist ihre Stra–“ 

„Sie werden mir erklären, warum Sie zugelassen haben, dass dieser Mann die Ihnen anvertrauten Kinder verletzt und wenn die Erklärung nicht gut genug ist, dann werde ich meine Zeitungskampagne starten und gegen \emph{Sie} richten.“ 

Minervas Oberkörper taumelte zurück angesichts dieser kraftvollen Erwiderung, dieser rohen \emph{Majestätsbeleidigung}. 

Selbst Severus wirkte schockiert. 

„Das, Harry, wäre äußerst unklug“, sagte Dumbledore langsam. „Ich bin die führende Figur, die Lucius auf dem Spielfeld gegenübersteht. Wenn du so etwas tätest, würde ihn das sehr stärken, und ich glaube nicht, dass du dich für jene Seite entschieden hast.“ 

Der Junge war lange still. 

„Diese Unterhaltung wird privat“, sagte Harry. Seine Hand zuckte in Severus’ Richtung. „Schicken Sie ihn weg.“ 

Dumbledore schüttelte den Kopf. „Harry, habe ich dir nicht gesagt, dass Severus Snape mein volles Vertrauen genießt?“ 

Der Schock war auf dem Gesicht des Jungen klar zu erkennen. „Das Mobbing dieses Mannes macht Sie angreifbar! Ich bin nicht der Einzige, der eine Zeitungskampagne gegen Sie anfangen könnte! Das ist wahnsinnig! Warum tun Sie das?“ 

Dumbledore seufzte. „Es tut mir Leid, Harry. Es hat mit Dingen zu tun, die zu hören du im Moment noch nicht bereit bist.“ 

Der Junge starrte Dumbledore an. Dann wandte er sich Severus zu. Dann wieder zurück zu Dumbledore. 

„Es \emph{ist} Wahnsinn“, sagte der Junge langsam. „Sie weisen ihn nicht in die Schranken, weil Sie denken, dass das alles \emph{dazugehört}; dass es \emph{Teil eines Musters} ist. Dass Hogwarts einen bösen Zaubertränkelehrer braucht, um eine richtige Zaubererschule zu sein, ebenso wie es einen Geist braucht, der Geschichte unterrichtet.“ 

„Das hört sich wie etwas an, was ich tun würde, nicht war?“, sagte Dumbledore lächelnd. 

„Inakzeptabel“, sagte Harry bestimmt. Sein Blick war nun kalt und düster. „Ich werde dieses Mobbing, diesen Machtmissbrauch nicht dulden. Ich hatte verschiedene Vorgehensweise erwogen, aber ich werde es einfach machen. Entweder dieser Mann verschwindet oder ich verschwinde.“ 

Minerva keuchte wieder auf. Severus’ Augen flackerten seltsam auf. 

Nun wurde auch Dumbledores Blick kalt. „Ein Schulverweis, Mr~Potter, ist die schwerstmögliche Strafe, die einem Schüler angedroht werden kann. Er dient üblicherweise nicht als Drohung eines Schülers gegenüber dem Schulleiter. Dies ist die beste Zaubererschule auf der gesamten Welt und die Möglichkeit, hier ausgebildet zu werden, erhält nicht jeder. Stehen Sie unter dem Eindruck, dass Hogwarts ohne Sie nicht weiter existieren könnte?“ 

Und Harry saß sanft lächelnd da. 

Plötzlich graute es Minerva. Sicher würde Harry nicht – 

„Sie vergessen“, sagte Harry, „dass Sie nicht der einzige sind, der Muster erkennen kann. \emph{Dies wird privat. Jetzt schicken Sie ihn} –“ Harrys Hand zuckte wieder in Severus’ Richtung, dann unterbrach er sich mitten im Satz, mitten in der Geste. 

Minerva konnte auf Harrys Gesicht sehen, wann er sich erinnerte. 

Sie hatte es ihm schließlich gesagt. 

„Mr~Potter“, sagte der Schulleiter, „erneut: Severus Snape genießt mein volles Vertrauen.“ 

„Sie haben es ihm erzählt“, flüsterte der Junge. „Sie schrecklicher Narr.“ 

Dumbledore reagierte auf die Beleidigung gar nicht. „Was habe ich ihm erzählt?“ 

„Dass der Dunkle Lord am Leben ist.“ 

\emph{„Wovon, in Merlins Namen, sprechen Sie da, Potter?“}, rief Severus vollkommen verwundert und wütend. 

Harry blickte kurz zu ihm und lächelte grimmig. „Oh, also sind wir \emph{doch} ein Slytherin“, sagte Harry. „Ich hatte mich schon gewundert.“ 

Dann war es still. 

Schließlich sprach Dumbledore. Seine Stimme klang milde. „Harry, \emph{wovon} sprichst du?“ 

„Es tut mir Leid, Albus“, flüsterte Minerva. 

Severus und Dumbledore drehten sich zu ihr. 

„Professor McGonagall hat es mir nicht gesagt“, sagte Harrys Stimme eiliger und weniger ruhig als zuvor. „Ich habe es erraten. Wie gesagt, auch ich kann die Muster erkennen. Ich habe es vermutet und sie hatte ihre Reaktion ebenso unter Kontrolle wie Severus. Um ein Haar wäre es perfekt gewesen. Doch so konnte ich erkennen, dass sie es vortäuschte, dass die Reaktion nicht echt war.“ 

„Und ich habe ihm erzählt“, sagte Minerva mit leicht zitternder Stimme, „dass du und ich und Severus die Einzigen sind, die davon wissen.“ 

„Diese Information hat sie mir zugestanden, um zu verhindern, dass ich umherlaufe und irgendwelchen Leuten Fragen stelle, wie ich es angedroht hatte, wenn sie mir nichts erzählte“, sagte Harry. Der Junge gluckste kurz. „Ich hätte wirklich versuchen sollen, einen von Ihnen alleine abzupassen und zu sagen, dass sie mir alles erzählt hätte; nur um zu schauen, ob Ihnen etwas entgleitet. Hätte vermutlich nicht funktioniert, aber einen Versuch wäre es wert gewesen.“ Der Junge lächelte wieder. „Die Drohung liegt aber noch auf dem Tisch und ich erwarte, dass ich irgendwann \emph{umfassend} informiert werde.“ 

Severus warf ihr einen verachtungsvollen Blick zu. Minerva reckte das Kinn hoch und ertrug es. Sie wusste, dass sie es verdient hatte. 

Dumbledore lehnte sich im gepolsterten Thron zurück. Das letzte Mal, dass Minerva seine Augen so kalt gesehen hatte, war an dem Tag gewesen, als sein Bruder gestorben war. „Und Sie drohen nun, uns mit Voldemort alleine zu lassen, wenn wir Ihre Wünsche nicht erfüllen.“ 

Harrys Stimme war rasiermesserscharf. „Es tut mir Leid, dass ich Ihnen mitteilen muss, dass Sie nicht der Mittelpunkt des Universums sind. Ich drohe nicht damit, die britische Zaubererwelt im Stich zu lassen. Ich drohe damit, \emph{Sie} im Stich zu lassen. Ich bin kein sanftmütiger kleiner Frodo. Dies ist \emph{meine} Mission und wenn Sie teilnehmen wollen, werden Sie nach \emph{meinen} Regeln spielen.“ 

Dumbledores Gesichtsausdruck war weiterhin kalt. „Ich beginne, an Ihrer Eignung zum Helden zu zweifeln, Mr~Potter.“ 

Harrys Blick war ebenso eisig. „Ich beginne, an Ihrer Eignung zum Gandalf zu zweifeln, \emph{Mr Dumbledore}. Boromir war zumindest ein entschuldbarer Fehler. Was macht dieser \emph{Nazgûl} unter meinen Gefährten?“ 

Minerva war völlig verloren. Sie blickte zu Severus, um zu sehen, ob er dem folgen konnte, und sie sah, dass Severus sein Gesicht von Harry abgewandt hatte und lächelte. 

„Ich denke“, sagte Dumbledore langsam, „dass dies aus Ihrer Sicht eine verständliche Frage ist. Also, Mr~Potter, wenn Professor Snape Sie fortan in Ruhe lässt, wird dies dann das letzte Mal sein, dass diese Angelegenheit aufkommt, oder werden Sie jede Woche mit einer neuen Forderung hier erscheinen?“ 

„\emph{Mich} in Ruhe lässt?“ Harrys Stimme war empört. „Ich bin nicht sein einziges Opfer und beileibe nicht das verletzlichste. \emph{Haben Sie vergessen, wie wehrlos Kinder sind? Wie sehr sie leiden?} Fortan wird Severus \emph{jeden} Schüler auf Hogwarts angemessen und rücksichtsvoll behandeln, oder Sie werden sich einen anderen Zaubertränke-Lehrer suchen oder Sie werden sich einen anderen Helden suchen!“ 

Dumbledore begann zu lachen – ein herzliches, warmes, humorvolles Gelächter, als ob Harry gerade einen urkomischen Tanz vorgeführt hätte. 

Minerva wagte es nicht, sich zu bewegen. Ihre Augen blitzten hin und her und sie sah, dass Severus ebenso bewegungslos war. 

Harrys Miene wurde noch kälter. „Sie irren sich, Schulleiter, wenn Sie denken, dass ich scherze. Das ist keine Bitte. Das ist Ihre Strafe.“ 

„Mr~Potter –“, sagte Minerva. Sie wusste nicht einmal, was sie sagen würde. Sie konnte das einfach nicht so stehen lassen. 

Harry machte eine abwinkende Geste und sprach weiter mit Dumbledore. „Und wenn Ihnen das unhöflich erschien“, sagte Harry, mit nun etwas sanfterer Stimme, „so erschien es kein Stück weniger unhöflich, als Sie es zu mir sagten. Sie würden so etwas zu niemandem sagen, den Sie als richtigen Menschen wahrnehmen statt als unbedeutendes Kind, und ich werde Sie ebenso höflich behandeln, wie sie mich behandeln –“ 

„Oh, wahrlich, wahrlich, dies ist meine Strafe – und was für eine! \emph{Natürlich} erpresst du mich hier, um deine Mitschüler zu retten, nicht dich selbst! Ich kann mir nicht vorstellen, warum ich jemals etwas anderes vermutet hatte!“ Dumbledore lachte jetzt noch lauter. Er schlug drei Mal mit der Faust auf den Tisch. 

Harrys Blick wurde unsicher. Er wandte sich ihr zu, sprach sie zum ersten Mal an. „Entschuldigen Sie“, sagte Harry. Seine Stimme schien zu erzittern. „Sollte er jetzt seine Medikamente nehmen, oder so?“ 

„Äh …“ Minerva wusste absolut nicht, was sie antworten konnte. 

„Nun“, sagte Dumbledore. Er wischte die Tränen beiseite, die ihm in den Augen standen. „Ich bitte um Verzeihung. Entschuldigung, dass ich Sie unterbrochen habe. Bitte, fahren Sie mit der Erpressung fort.“ 

Harry öffnete seinen Mund, schloss ihn dann wieder. Er wirkte nun etwas unsicher. „Ähm … er wird auch aufhören, die Gedanken der Schüler zu lesen.“ 

„Minerva“, sagte Severus mit mörderischer Stimme, „du –“ 

„Der Sprechende Hut hat mich gewarnt“, sagte Harry. 

\emph{„Was?“} 

„Mehr kann ich nicht sagen. Wie dem auch sei, ich glaube, das war’s. Ich bin fertig.“ 

Stille. 

„Was nun?“, sagte Minerva, als offensichtlich war, dass niemand sonst etwas sagen würde. 

„Was nun?“, wiederholte Dumbledore. „Naja, nun gewinnt der Held natürlich.“ 

\emph{„Was?“}, sagten Severus, Minerva und Harry. 

„Nun, er hat uns offenbar in eine Ecke gedrängt“, sagte Dumbledore glücklich lächelnd. „Doch Hogwarts \emph{braucht} einen bösen Zaubertränke-Lehrer, sonst wäre es einfach keine richtige Zaubererschule, nicht wahr? Wie wäre es also, wenn Professor Snape sich nur gegenüber Schülern ab dem fünften Schuljahr fürchterlich verhält?“ 

\emph{„Was?“}, sagten alle drei erneut. 

„Wenn es dir um die verletzlichsten Opfer geht. Vielleicht hast Du Recht, Harry. Vielleicht \emph{habe} ich über die Jahrzehnte vergessen, wie es ist, ein Kind zu sein. Schließen wir also einen Kompromiss. Severus wird weiterhin unfair Punkte an Slytherin verteilen und Fehlverhalten in seinem Haus durchgehen lassen und er wird sich gegenüber Schülern anderer Häuser ab dem fünften Schuljahr fürchterlich verhalten. Gegenüber Jüngeren wird er sich furchteinflößend verhalten, aber seine Position nicht missbrauchen. Hogwarts hat seinen bösen Zaubertränke-Lehrer und die verletzlichsten Opfer, wie du sagtest, sind in Sicherheit.“ 

Minerva McGonagall war so schockiert wie nie zuvor in ihrem Leben. Sie blickte unsicher zu Severus rüber, dessen Gesicht vollkommen neutral war, als wüsste er selbst nicht, welchen Ausdruck er aufsetzen sollte. 

„Ich denke, das ist akzeptabel“, sagte Harry. Seine Stimme klang etwas seltsam. 

„Das ist nicht Ihr Ernst“, sagte Severus, seine Stimme ebenso ausdruckslos wie sein Gesichtsausdruck. 

„Ich bin sehr dafür“, sagte Minerva langsam. Sie war so sehr dafür, dass ihr Herz unter dem Umhang heftig pochte. „Aber was sollen wir den Schülern bloß erzählen? Sie haben es wohl nicht hinterfragt, solange Severus sich … allen gegenüber fürchterlich verhielt, aber –“ 

„Harry kann den anderen Schülern erzählen, dass er ein schreckliches Geheimnis von Severus entdeckt hat und uns ein wenig erpresst hat“, sagte Dumbledore. „Es ist schließlich war; er hat entdeckt, dass Severus Gedanken lesen kann, und er hat uns wahrlich erpresst.“ 

„Das ist Wahnsinn!“, entfuhr es Severus. 

„Mua har har!“, sagte Dumbledore. 

„Äh …“, sagte Harry unsicher. „Und wenn mich jemand fragt, warum Fünftklässler und Ältere Pech haben? Ich könnte es ihnen nicht übel nehmen, wenn sie deswegen verärgert sind; und das war nun nicht gerade meine Idee –“ 

„Erzähle ihnen“, sagte Dumbledore, „dass nicht Du den Kompromiss vorgeschlagen hast, dass es alles war, was Du heraushandeln konntest. Und weigere Dich dann, ihnen mehr zu erzählen. Auch das ist wahr. Es ist eine Kunst; mit etwas Übung wirst Du sie erlernen.“ 

Harry nickte langsam. „Und die Punkte, die er Ravenclaw abgezogen hat?“ 

„Sie werden die Punkte nicht zurückbekommen.“ 

Minerva hatte das gesagt. 

Harry sah sie an. 

„Es tut mir Leid, Mr~Potter“, sagte sie. Es \emph{tat} ihr Leid, doch es war notwendig. „Ihr Fehlverhalten \emph{muss} irgendwelche Folgen haben, sonst versinkt die Schule im Chaos.“ 

Harry zuckte mit den Schultern. „Akzeptabel“, sagte er tonlos. „Aber in Zukunft wird Severus weder meinen Hausgenossen schaden indem er mir Punkte abzieht, noch wird er meine wertvolle Zeit mit Strafarbeiten vergeuden. Sollte er der Meinung sein, dass mein Verhalten der Korrektur bedarf, so wird er dieses Anliegen Professor McGonagall mitteilen.“ 

„Harry“, sagte Minerva, „wirst Du Dich der Schulordnung unterwerfen, oder wirst Du in Zukunft über dem Gesetz stehen, so wie Severus bisher?“ 

Harry sah sie an. Er verspürte etwas Warmes in ihrem Blick; nur kurz, bevor es unterdrückt wurde. „Ich werde ein normaler Schüler sein, für alle Lehrer, die weder verrückt noch böse sind, vorausgesetzt, dass sie nicht von anderen unter Druck gesetzt werden, die verrückt oder böse sind.“ Harry blickte kurz zu Severus, wandte sich dann wieder Dumbledore zu. „Lassen Sie Minerva in Ruhe, dann werde ich in ihrer Gegenwart ein gewöhnlicher Hogwarts-Schüler sein. Ohne Sonderrechte oder Immunitäten.“ 

„Wundervoll“, sagte Dumbledore ernst. „Gesprochen wie ein wahrer Held.“ 

„Und“, sagte sie, „Mr~Potter muss sich öffentlich für sein heutiges Handeln entschuldigen.“ 

Harry warf ihr noch einen Blick zu. Diesmal war der Blick skeptischer. 

„Die Schuldisziplin hat unter Ihren Handlungen schwer gelitten, Mr~Potter“, sagte Minerva. „Sie muss wiederhergestellt werden.“ 

„Ich denke, Professor McGonagall, dass Sie den Wert der sogenannten Schuldisziplin viel zu hoch einschätzen, verglichen mit Dingen wie einem lebendigen Geschichtslehrer oder dem zivilisierten Umgang mit Schülern. Die gegenwärtige Hierarchie aufrechtzuerhalten und die bestehenden Regeln durchzusetzen erscheint so viel weiser und moralischer und wichtiger, wenn man selbst an der Spitze steht und die Regeln durchsetzt, als wenn man darunter leidet, und ich kann Ihnen falls nötig Studien zitieren, die das belegen. Ich könnte diesen Punkt noch mehrere Stunden lang erörtern, aber ich will es hierbei belassen.“ 

Minerva schüttelte mit dem Kopf. „Mr~Potter, Sie unterschätzen die Bedeutung von Disziplin, da Sie selbst keine benötigen –“ Sie unterbrach sich. Das hatte sich nicht so angehört wie sie beabsichtigte und Severus, Dumbledore und sogar Harry sahen sie komisch an. „Um zu lernen, meine ich. Nicht jedes Kind kann in Abwesenheit von Autoritäten lernen. Und es wird den anderen Kindern schaden, Mr~Potter, wenn sie glauben, Ihrem Beispiel folgen zu können.“ 

Harrys Lippen verzogen sich zu einem schiefen Lächeln. „Die erste und letzte Zuflucht ist die Wahrheit. Die Wahrheit ist, ich hätte nicht wütend werden sollen, ich hätte den Unterricht nicht stören sollen, ich hätte nicht tun sollen, was ich getan habe und ich habe für alle Anderen ein schlechtes Beispiel abgegeben. Die Wahrheit ist auch, dass Severus sich auf eine Weise verhalten hat, die sich für einen Hogwarts-Lehrer nicht gehört und dass er von nun an mehr Rücksicht auf die verletzten Gefühle der Schüler bis zum vierten Schuljahr nehmen wird. Wir beide könnten aufstehen und die Wahrheit sprechen. Damit könnte ich leben.“ 

„In Ihren Träumen, Potter!“, spie Severus. 

„Denn“, sagte Harry mit grimmigem Lächeln, „wenn die Schüler sehen, dass die Regeln für \emph{alle} gelten … auch für Lehrer, nicht nur für arme, hilflose Schüler, die unter diesem System nur leiden … nun, die positiven Auswirkungen auf die Schuldisziplin dürften \emph{enorm} sein.“ 

Einen Moment lang war es still, dann gluckste Dumbledore. „Minerva denkt, dass Sie sehr viel mehr Recht haben, als Sie dürften.“ 

Harrys Blick zuckte weg von Dumbledore, zu Boden. „\emph{Sie} lesen \emph{ihre} Gedanken?“ 

„Gesunder Menschenverstand wird oft mit Legilimentik verwechselt“, sagte Dumbledore. „Ich werde die Angelegenheit mit Severus besprechen und von Ihnen keine Entschuldigung fordern, solange er sich nicht ebenfalls entschuldigt. Und nun erkläre ich diese Angelegenheit für beendet, zumindest bis zum Mittagessen.“ Er zögerte. „Allerdings, Harry, fürchte ich, dass Minerva mit Dir über eine weitere Angelegenheit sprechen wollte. Und das geschieht nicht auf Druck meinerseits. Minerva, wenn ich bitten darf?“ 

Minerva stand aus dem Stuhl auf und wäre fast zusammengebrochen. Zu viel Adrenalin war in ihrem Blut, ihr Herz schlug zu schnell. 

„Fawkes“, sagte Dumbledore, „begleite sie, bitte.“ 

„Das ist nicht –“, begann sie. 

Dumbledore warf ihr einen Blick zu und sie verstummte. 

Der Phönix schwebte durch den Raum wie eine sanfte, züngelnde Flamme und landete auf ihrer Schulter. Sie spürte die Wärme im ganzen Körper, auch durch den Umhang. 

„Bitte folgen Sie mir, Mr~Potter“, sagte sie nun bestimmt und sie verließen das Büro. 

\later 

Sie standen auf der sich drehenden Treppe und bewegten sich still abwärts. 

Minerva wusste nicht, was sie sagen sollte. Sie kannte die Person nicht, die neben ihr stand. 

Und Fawkes begann zu summen. 

Es war zärtlich und weich, wie ein Kaminfeuer klingen würde, wenn es eine Melodie hätte, und es durchdrang Minervas Gedanken, linderte und besänftigte alles, was es berührte … 

„\emph{Was} ist \emph{das}?“, flüsterte Harry neben ihr. Seine Stimme war unsicher, schwankte, sprang zwischen den Tonlagen umher. 

„Der Gesang des Phönix“, sagte Minerva, die sich ihrer Worte nicht völlig bewusst war; ihre Konzentration galt nur der seltsamen, ruhigen Musik. „Auch er heilt.“ 

Harry wandte sich von ihr ab, doch sie hatte einen kurzen Schmerz erblickt. 

Der Weg hinab schien sehr lange zu dauern oder vielleicht schien auch nur die Musik sehr lange zu dauern, und als sie durch die Öffnung schritten, die sonst vom Wasserspeier versperrt war, hielt sie Harrys Hand fest in ihrer. 

Als der Wasserspeier an seinen Platz zurückkehrte, verließ Fawkes ihre Schulter, stieß hernieder und blieb vor Harry in der Luft stehen. 

Harry starrte Fawkes an wie jemand, der vom ewig wandelbaren Licht eines Feuers hypnotisiert wurde. 

„Was soll ich tun, Fawkes?“, flüsterte Harry. „Ich hätte sie nicht beschützen können, wenn ich nicht wütend gewesen wäre.“ 

Die Flügel des Phönix flatterten weiter, er schwebte weiter an der Stelle. Außer den Flügelschlägen ertönten keine Geräusche. Plötzlich blitzte es auf, wie eine Flamme, die aufloderte und dann erlosch, und Fawkes war verschwunden. 

Beide blinzelten, als wären sie aus einem Traum erwacht, oder vielleicht als wären sie wieder eingeschlafen. 

Minerva sah nach unten. 

Harry Potters leuchtendes junges Gesicht sah zu ihr auf. 

„Sind Phönixe Menschen?“, sagte Harry. „Ich meine, sind sie intelligent genug, um als Menschen zu zählen? Könnte ich mit Fawkes reden, wenn ich wüsste, wie?“ 

Minerva blinzelte angestrengt. Dann blinzelte sie erneut. „Nein“, sagte Minerva mit zitternder Stimme. „Phönixe sind Geschöpfe aus machtvoller Magie. Diese Magie gibt ihrer Existenz eine Bedeutung, die kein einfaches Tier besitzen könnte. Sie sind Feuer, Licht, Heilung, Wiedergeburt. Aber letzlich, nein.“ 

„Wo kann ich einen bekommen?“ 

Minerva beugte sich nieder und umarmte ihn. Sie hatte es nicht beabsichtigt, aber sie schien keine andere Wahl zu haben. 

Als sie wieder aufstand, fiel es ihr schwer zu sprechen. Doch sie musste fragen. „Was ist heute geschehen, Harry?“ 

„Die Antworten auf die wichtigen Fragen kenne ich auch nicht. Abgesehen davon würde ich erstmal wirklich gerne nicht mehr darüber nachdenken.“ 

Minerva nahm seine Hand wieder in ihre und den Rest des Weges blieben sie still. 

Es war nur ein kurzer Weg, da das Büro der Stellvertretenden Schulleiterin natürlich in der Nähe des Schulleiterbüros war. 

Minerva setzte sich hinter ihren Schreibtisch. 

Harry setzte sich vor ihren Schreibtisch. 

„Also“, flüsterte Minerva. Sie hätte fast alles gegeben, um dies nicht zu tun, oder um nicht diejenige zu sein, die es tun musste, oder um es zu irgendeinem anderen Zeitpunkt zu tun als jetzt. „Es geht um eine Frage der Schuldisziplin. Von der Sie nicht ausgenommen sind.“ 

„Nämlich?“, sagte Harry. 

Er wusste es nicht. Er hatte noch nicht daran gedacht. Sie spürte wie ihre Kehle sich zusammenschnürte. Doch es war ihre Aufgabe und sie würde sich nicht davor drücken. 

„Mr~Potter“, sagte Professor McGonagall, „ich muss bitte Ihren Zeitumkehrer sehen.“ 

Sofort verschwand all der Friede des Phönix’ von seinem Gesicht und Minerva fühlte sich, als hätte sie mit einem Dolch zugestochen. 

\emph{„Nein!“}, sagte Harry. Seine Stimme klang panisch. „Ich brauche ihn, ich kann sonst nicht zur Schule gehen, ich werde nicht schlafen können!“ 

„Sie werden schlafen können“, sagte sie. „Das Ministerium hat die Schutzhülle für Ihren Zeitumkehrer zugeschickt. Ich werde sie verzaubern, so dass sie sich nur zwischen neun Uhr abends und Mitternacht öffnen lässt.“ 

Harrys Gesicht verzog sich. „Aber – aber ich –“ 

„Mr~Potter, wie oft haben Sie den Zeitumkehrer seit Montag benutzt? Wie viele Stunden?“ 

„Ich …“, sagte Harry. „Moment, lassen Sie mich nachrechnen …“ Er blickte auf seine Uhr. 

Minerva wurde traurig. Sie hatte es sich gedacht. „Es waren also nicht nur zwei Stunden pro Tag. Ich vermute, wenn ich mich bei Ihren Mitschülern erkundigen würde, würde ich hören, dass Sie Mühe hatten, lang genug wach zu bleiben, um zu einer vernünftigen Zeit ins Bett zu gehen, und dass Sie morgens immer früher und früher aufgestanden sind. Habe ich Recht?“ 

Harrys Gesichtsausdruck sagte ihr alles, was sie wissen musste. 

„Mr~Potter“, sagte sie sanft, „einigen Schülern kann man keine Zeitumkehrer anvertrauen, weil sie süchtig danach werden. Wir geben ihnen Zaubertränke, um ihren Schlafzyklus um die nötige Zeit zu verlängern, aber sie benutzen den Zeitumkehrer für andere Dinge, nicht nur um ihren Unterricht zu besuchen. Und dann müssen wir sie zurücknehmen. Mr~Potter, Sie haben angefangen, den Zeitumkehrer als Lösung für alle Probleme zu sehen, oft in völlig unnötigen Situationen. Sie haben ihn benutzt um ein Erinnermich zurück zu bekommen. Sie haben zugelassen, dass andere Schüler merken, wie sie aus einem Wandschrank verschwinden, statt sich wieder darin zu verstecken, sobald sie draußen waren und mir oder jemand anderem Bescheid gesagt hatten, damit ich die Tür öffnen könnte. 

Harrys Gesichtsausdruck sagte klar und deutlich, dass er daran einfach nicht gedacht hatte. 

„Und was viel wichtiger ist“, sagte sie, „Sie hätten einfach im Unterricht von Professor Snape sitzen sollen. Und zusehen. Und am Ende des Unterrichts gehen. Genau wie Sie es getan hätten, wenn Sie keinen Zeitumkehrer gehabt hätten. Es gibt Schüler, denen man keinen Zeitumkehrer anvertrauen kann, Mr~Potter. Sie gehören dazu. Es tut mir Leid.“ 

„Aber ich \emph{brauche} ihn“, platzte es aus Harry hervor. „Was wenn Slytherins mich bedrohen und ich entkommen muss? Er \emph{schützt} mich –“ 

„Jeder andere Schüler in diesem Schloss ist den gleichen Gefahren ausgesetzt und ich versichere Ihnen, dass die alle überleben. In diesem Schloss ist seit fünfzig Jahren kein Schüler gestorben. Mr~Potter, Sie werden mir den Zeitumkehrer aushändigen, und zwar jetzt.“ 

Harrys Gesicht verzog sich gepeinigt, doch er holte den Zeitumkehrer unter seinem Umhang hervor und gab ihn ihr. 

Aus ihrem Schreibtisch zog Minerva eine der Schutzhüllen, die Hogwarts bekommen hatte. Sie legte die Hülle um das Stundenglas, ließ sie einschnappen und verschloss sie mit ihrem Zauberstab. Dann strich sie mit dem Zauberstab über die Hülle und sprach eine einfache aber dauerhafte Verzauberung. 

\emph{„Das ist nicht fair!“}, kreischte Harry. „Ich habe heute halb Hogwarts vor Professor Snape gerettet. Ist es richtig, dass ich dafür bestraft werde? Ich habe Ihren Gesichtsausdruck gesehen, Sie \emph{hassen} sein Verhalten!“ 

Minerva sprach einen Moment lang nicht. Sie zauberte. 

Als sie fertig war und aufsah, wusste sie, dass ihr Gesichtsausdruck streng war. Vielleicht war es falsch, sich so zu verhalten. Aber vielleicht war es auch richtig. Ihr gegenüber saß ein starrköpfiges Kind und dass hieß \emph{noch lange nicht}, dass irgendwas mit dem Universum nicht stimmte. 

„\emph{Fair}, Mr~Potter?“, erwiderte sie. „Ich musste innerhalb von \emph{zwei} Tagen \emph{zwei} Berichte wegen öffentlicher Verwendung eines Zeitumkehrers beim Ministerium einreichen! Sie sollten \emph{außerordentlich} dankbar sein, dass Sie den Zeitumkehrer überhaupt behalten dürfen! Der Schulleiter ist persönlich zum Ministerium gefloht um dort um eine Ausnahme zu bitten und wenn Sie nicht der Junge, der lebt wären, hätte selbst das nicht ausgereicht!“ 

Harry starrte sie mit offenem Mund an. 

Sie wusste, dass er die wütende Seite von Professor McGonagall sah. 

Harrys Augen füllten sich mit Tränen. 

„Es … tut mir Leid“, flüsterte er mit gebrochener und erstickter Stimme. „Es tut mir Leid … dass ich … Sie enttäuscht habe …“ 

„Es tut mir auch Leid, Mr~Potter“, sagte sie ernst und reichte ihm den nun eingeschränkten Zeitumkehrer. „Sie dürfen gehen.“ 

Harry wandte sich ab und eilte schluchzend aus ihrem Büro. Sie hörte seine Schritte den Gang entlang eilen, bis die Tür sich schloss und das Geräusch verstummte. 

„Es tut mir auch Leid, Harry“, flüsterte sie im stillen Zimmer. „Es tut mir auch Leid.“ 

\later 

Fünfzehn Minuten nach Beginn des Mittagessens. 

Niemand sprach mit Harry. Einige Ravenclaws warfen ihm wütende Blicke zu, andere mitleidige, einige der jüngsten Schüler sogar bewundernde, doch niemand sprach mit ihm. Selbst Hermine hatte nicht versucht, rüber zu kommen. 

Fred und George waren vorsichtig näher gekommen. Sie hatten nichts gesagt. Das Angebot war klar; ebenso, dass es freiwillig war. Harry hatte ihnen gesagt, dass er erst dann rüber kommen würde, wenn der Nachtisch erschien. Sie hatten genickt und waren schnell weg gegangen. 

Es lag vermutlich an Harrys vollkommen ausdruckslosen Gesichtsausdruck. 

Die Anderen dachten vermutlich, dass er Ärger unterdrückte, oder Verbitterung. Sie wussten, da Flitwick ihn abgeholt hatte, dass er ins Büro des Schulleiters gebeten wurde. 

Harry versuchte nicht zu lächeln, denn wenn er lächelte, würde er anfangen zu lachen, und wenn er anfing zu lachen, dann würde er nicht mehr damit aufhören, bis die netten Leute in weißen Uniformen kämen um ihn mitzunehmen. 

Es war zu viel. Es war einfach alles zu viel. Harry hatte sich fast auf die Dunkle Seite geschlagen. Seine dunkle Seite hatte Dinge getan, die im Nachhinein wahnsinnig erschienen; seine dunkle Seite hatte einen unmöglichen Sieg errungen, der entweder wohlverdient war oder einer plötzlichen Laune eines verrückten Schulleiters entsprang; seine dunkle Seite hatte seine Freunde beschützt. Er hielt es alles nicht mehr aus. Fawkes müsste ihm nochmal etwas vorsingen. Er müsste den Zeitumkehrer benutzen und eine ruhige Stunde verbringen, um sich zu erholen, doch das war nicht mehr möglich und diesen Verlust spürte er wie ein Loch in seiner Existenz, doch er konnte nicht darüber nachdenken, sonst würde er wohl anfangen zu lachen. 

Zwanzig Minuten. Alle Schüler, die zum Mittagessen erscheinen würden, waren nun anwesend, noch fast niemand war gegangen. 

Ein Löffel, der gegen ein Glas getippt wurde, erklang in der Großen Halle. 

„Wenn ich um Eure Aufmerksamkeit bitten darf“, sagte Dumbledore. „Harry Potter hat etwas, das er uns mitteilen möchte.“ 

Harry atmete tief durch und stand auf. Er ging vor zum Lehrertisch, alle Augen waren auf ihn gerichtet. 

Harry drehte sich um und blickte auf die vier Tische. 

Es wurde immer schwerer, nicht zu lächeln, doch Harrys Gesicht blieb ausdruckslos, während er die kurze Rede hielt, die er auswendig gelernt hatte. 

„Die Wahrheit ist heilig“, sagte Harry tonlos. „Eines meiner am höchsten geschätzten Besitztümer ist ein Anstecker, auf dem steht ‚Sprich die Wahrheit, selbst wenn deine Stimme zittert.‘ Dies nun ist die Wahrheit. Denkt daran. Ich sage es nicht, weil ich dazu gezwungen werde. Ich sage es, weil es wahr ist. Was ich in Professor Snapes Unterricht getan habe, war närrisch, dumm, kindisch und ein unentschuldbarer Verstoß gegen die Regeln von Hogwarts. Ich habe den Unterricht gestört und meinen Mitschülern die unersetzliche Unterrichtszeit genommen. All das, weil ich meine Laune nicht unter Kontrolle hatte. Ich hoffe, dass kein einziger von euch jemals meinem Beispiel folgt. Ich selbst habe zweifelsohne vor, es nie wieder geschehen zu lassen.“ 

Viele der Schüler, die Harry ansahen, trugen nun einen ernsten, unglücklichen Gesichtsausdruck, wie auf der Trauerfeier für einen gefallenen Helden. Unter den jüngeren Schülern am Gryffindor-Tisch war dieser Ausdruck fast allgegenwärtig. 

Bis Harry seine Hand hob. 

Er hob sie nicht hoch. Das hätte vermessen wirken können. Er erhob sie ganz bestimmt nicht gegen Severus. Harry hob seine Hand einfach auf Brusthöhe und schnippte sanft mit den Fingern – eine sichtbare, aber kaum hörbare Geste. Womöglich hatten viele am Lehrertisch sie nicht einmal gesehen. 

Diese scheinbare Geste des Widerstands erntete ein plötzliches Lächeln von jüngeren Schülern und Gryffindors, kühles, abschätziges Schnauben von Slytherins und Stirnrunzeln und besorgte Blicke von allen anderen. 

Harrys Gesicht blieb ausdruckslos. „Danke“, sagte er. „Das war alles.“ 

„Danke, Mr~Potter“, sagte der Schulleiter. „Und nun möchte Professor Snape uns ebenfalls etwas mitteilen.“ 

Severus stand mit einer fließenden Bewegung an seinem Platz am Lehrertisch auf. „Ich wurde darauf aufmerksam gemacht“, sagte er, „dass mein eigenes Handeln zu dem anerkanntermaßen unentschuldbaren Ärger von Mr~Potter beigetragen hat und in der folgenden Diskussion wurde mir bewusst, dass ich vergessen hatte, wie leicht die Gefühle junger und unreifer Menschen verletzt werden können. Das Zaubertränke-Klassenzimmer ist ein gefährlicher Ort und ich bin weiterhin der Ansicht, dass strikte Disziplin notwendig ist –“ 

In diesem Moment gaben viele Schüler ein ersticktes Husten von sich. 

Severus fuhr fort, als hätte er nichts gehört. „– aber zukünftig werde ich die … emotionale Fragilität … von Schülern bis zum vierten Schuljahr stärker beachten. Bei den Punktabzügen für Ravenclaw bleibt es, aber die Strafarbeit für Mr~Potter werde ich zurückziehen. Danke.“ 

Ein einziges Händepaar klatschte am Gryffindor-Tisch und blitzschnell hielt Severus seinen Zauberstab in der Hand und stellte den Täter mit einem \emph{Quietus!} ruhig. 

„Ich verlange weiterhin Disziplin und Respekt von \emph{allen} Schülern“, sagte Severus kühl, „und jeder, der sich mit mir anlegt, wird es bereuen.“ 

Er setzte sich. 

„Vielen Dank!“, sagte Schulleiter Dumbledore fröhlich. „Esst weiter!“ 

Und Harry, das Gesicht immer noch ausdruckslos, begann wieder zu seinem Platz zwischen den Ravenclaws zu gehen. 

Plötzlich begann ein Stimmengewirr. Zwei Worte waren anfangs klar vernehmbar. Das erste war ein „Was –“, das Sätze wie „Was war das –“ oder „Was zum Teufel –“ einleitete. Das zweite war \emph{„Scourgify!“}, als Schüler sich selbst, die Tischdecke und einander vom Essen, das ihnen aus der Hand gefallen war, und von den vor Überraschung ausgespuckten Getränken säuberten. 

Einige Schüler weinten offen. Professor Sprout ebenso. 

Am Gryffindor-Tisch, wo ein Kuchen mit einundfünfzig nicht angezündeten Kerzen stand, flüsterte Fred: „Ich glaube, wir spielen nicht in seiner Liga, George.“ 

Und egal was Hermine erzählte, von diesem Tag an war es auf Hogwarts allgemein anerkannt, dass Harry Potter einfach alles geschehen lassen konnte, indem er mit den Fingern schnippte.
