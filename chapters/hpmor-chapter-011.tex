\chapter{Omake I, II und III\protect\footnotemark}
\authorsnotetext{„Omake“ ist Bonusmaterial, das nicht Teil der eigentlichen Handlung ist. In Kapitel 12 geht die eigentliche Geschichte weiter, dann experimentiert Harry mit Seltsaft und wir lernen Professor Quirrell kennen. Seid gespannt!}
\authorsnotetext{Die kurzen Texte in Omake III zu übersetzen war teils gar nicht so einfach – mein Dank gilt an dieser Stelle Annie für zwei exzellente Vorschläge.}
\authorsnotetext{Die Bemerkungen am Anfang von Omake II und III stammen vom Originalautor, Eliezer Yudkowsky.}

\section{Omake I: 72 Stunden bis zum Sieg}

(Auch bekannt als „Was Passiert, Wenn Du Harry Veränderst, Aber Alle Anderen Charaktere Gleich Lässt“) 

Dumbledore schaute über seinen Schreibtisch hinweg zu Harry und blinzelte freundlich. Der Junge war mit einem furchtbar ernsten Ausdruck auf seinem kindlichen Gesicht bei ihm erschienen – Dumbledore hoffte, dass, was auch immer der Grund dafür war, er nicht \emph{zu} schwerwiegend sein würde. Harry war viel zu jung, um sich den Prüfungen seines Lebens schon jetzt stellen zu müssen. „Worüber wolltest du mit mir sprechen, Harry?“ 

Harry James Potter-Evans-Verres lehnte sich in seinem Stuhl vor und lächelte grimmig. „Schulleiter, ich habe einen stechenden Schmerz in meiner Narbe verspürt, als der Sprechende Hut die Häuser verkündete. In Anbetracht der Tatsache, wie und wo ich diese Narbe erhalten habe, schien es mir keine gute Idee, dies einfach zu ignorieren. Zuerst dachte ich, es sei wegen Professor Snape, aber dann bin ich der Baconschen Methode gefolgt, die Bedingungen für die Anwesenheit und Abwesenheit des Phänomens herauszufinden, und ich habe festgestellt, dass meine Narbe dann, und genau dann schmerzt, wenn die Rückseite von Professor Quirrells Kopf – also, was auch immer sich unter seinem Turban befindet – mir zugewandt ist. Obwohl es sich um etwas Harmloseres handeln \emph{könnte,} denke ich, dass wir das Schlimmste annehmen sollten, dass es Du-Weißt-Schon-Wer – warten Sie, schauen Sie nicht so entsetzt, das ist eine ausgezeichnete Gelegenheit …“

\later 

\section{Omake II: I ain't afraid of Dark Lords}

Dies war die ursprüngliche Version von Kapitel 9. Sie wurde ersetzt, nachdem viele Leser sie zwar mochten, viele andere Leser aber äußerst allergisch auf Songs in Fanfics reagierten (aus Gründen, die wohl nicht viel Erklärung brauchen). Ich wollte keine Leser verjagen bevor sie Kapitel 10 erreichten. 
Lee Jordan ist im Canon der Mit-Scherzbold von Fred und George. Der Name „Lee Jordan“ klang für mich nach einem Muggelgeborenen, so dass er Fred und George ein Muggellied beibringen könnte, das Harry auch kennen würde. Für einige Leser war das nicht so selbstverständlich wie für euren Autor. 

\later

Draco kam nach Slytherin und Harry atmete erleichtert durch. Es \emph{schien} klar wie Kloßbrühe, aber man wusste nie, welches kleine Ereignis den Ablauf des eigenen Masterplans durcheinanderbringen könnte. 

Nun waren bald die P’s an der Reihe … 

Am Gryffindortisch wurde Geflüster laut. 

\emph{„Was, wenn es ihm nicht gefällt?“}

\emph{„Er hat keine andere Wahl –“}

\emph{„– nicht nach dem Streich, den er diesem Jungen –“}

\emph{„– Neville Longbottom hieß er –“}

\emph{„– er ist das ideale Angriffsziel.“}

\emph{„In Ordnung. Pass nur auf, dass du deinen Teil nicht vergisst.“}

\emph{„Wir haben es oft genug geprobt –“}

\emph{„– in den letzten drei Stunden.“}

Und Minerva McGonagall, die am Lehrertisch auf einem Podium stand, sah auf den nächsten Namen auf ihrer Liste. \emph{Bitte schicke ihn nicht nach Gryffindor, bitte schicke ihn nicht nach Gryffindor, OH BITTE schicke ihn nicht nach Gryffindor …} Sie holte tief Luft und rief: 

„Potter, Harry!“ 

Es wurde plötzlich still im Saal. Alle geflüsterten Gespräche stoppten. 

Die Stille wurde von einem schrecklichen summenden Geräusch unterbrochen, das sich in einer scheußlichen Persiflage auf melodische Musik veränderte. 

Minervas Kopf schoss schockiert herum und bemerkte, dass das summende Geräusch aus der Richtung des Gryffindortisches stammte, wo \emph{sie} auf dem Tisch \emph{standen}, eine Art Gerät an \emph{ihre} Lippen hielten und hineinbliesen. Minervas Hand griff zum Zauberstab, um \emph{Silencio} auf die Bande zu sprechen, aber ein anderes Geräusch hielt sie davon ab. 

Dumbledore kicherte. 

Minervas Augen richteten sich auf Harry Potter, der gerade erst losgegangen war, bevor er stockte und anhielt. 

Dann begann der Junge wieder zu laufen und dabei seine Beine auf eine seltsame schwungvolle Art zu bewegen, seine Arme vor und zurück baumeln zu lassen und synchron zu \emph{ihrer} Musik mit den Fingern zu schnipsen. 

\begin{center}
\emph{Zur Melodie von „Ghostbusters“}
\emph{(Gespielt auf dem Kazoo von Fred and George Weasley,
und gesungen von Lee Jordan.)}

.

\emph{There's a Dark Lord near?\\
Got no need to fear\\
Who you gonna call?}
\end{center}

„\shout{Harry Potter}!“, schrie Lee Jordan, und die Weasley-Zwillinge stimmten jubilierend mit ein. 

\begin{center}
\emph{With a Killing Curse?\\
Well it could be worse.\\
Who you gonna call?}
\end{center}

„\shout{Harry Potter}!“ Dieses Mal stimmten mehr Schreie mit ein. 

Die Weasley-Schrecken gingen nun in ein erweitertes Heulen über, begleitet von einigen älteren Muggelstämmigen, die ihre eigenen kleinen Geräte hervorgeholt hatten, zweifelsohne aus dem Silberbesteck der Schule transfiguriert. Als die Musik ihren Tiefpunkt erreichte, schrie Harry Potter: 

\begin{center}
\emph{I ain’t afraid of Dark Lords!}
\end{center}

Es gab einen Applaus, insbesondere am Gryffindor-Tisch, und noch mehr Studenten holten ihre eigenen Musikinstrumentimitate hervor. Das schlimme Summen verdoppelte seine Lautstärke und baute sich zu einem weiteren, nicht minder schrecklichen Crescendo auf: 

\begin{center}
\emph{I ain’t afraid of Dark Lords!}
\end{center}

Mit einer vagen Befürchtung dessen, was sie dort sehen würde, schielte Minerva zu beiden Seiten des Lehrertisches. 

Trelawney fächelte sich wie verrückt Luft zu, Flitwick schaute neugierig drein, Hagrid klatschte mit der Musik mit, Sprout wirkte ernst und Quirrell betrachtete den Jungen mit hämischer Freude. Direkt zu ihrer Linken summte Dumbledore mit, und direkt zu ihrer Rechten hatte Snape seinen Weinkelch so fest gekrallt, dass seine Knöchel weiß wurden und sich das Silber des Kelches langsam verformte. 

\begin{center}
\emph{Dark robes and a mask?\\
Impossible task?\\
Who you gonna call?\\
\shout{Harry Potter}!}

\emph{Giant Fire-Ape?\\
Old bat in a cape?\\
Who you gonna call?\\
\shout{Harry Potter}!}
\end{center}

Minervas Lippen verzogen sich zu einer weißen Linie. Sie würde mit \emph{ihnen} noch über die letzte Zeile reden, falls \emph{sie} dachten, dass sie keine Macht hätte, da es der erste Schultag war und Gryffindor keine Punkte hatte, die man wegnehmen könnte. Wenn \emph{sie} sich keine Sorgen um Nachsitzen machten, dann würde sie etwas anderes finden. 

Dann, mit einem plötzlichen Schrecken, schaute sie in Snapes Richtung – \emph{sicherlich} erkannte er, dass der Potter-Junge keine Ahnung hatte, von wem da gerade die Rede war – 

– und Snapes Gesicht hatte sich jenseits aller Rage zu einer Art angenehmer Indifferenz gewandelt. Ein kaum wahrnehmbares Lächeln spielte über seine Lippen. Er sah in die Richtung von Harry Potter, nicht zum Gryffindor-Tisch, und seine Hand komprimierte die Überreste dessen, was einmal ein Weinkelch gewesen war … 

Harry schritt voran, bewegte seine Arme und Beine im Ghostbusters-Tanz und behielt ein Lächeln auf seinem Gesicht. Es war ein toller Streich und hatte ihn komplett überrascht. Das Mindeste, was er machen konnte, war mitzuspielen und es nicht zu ruinieren. 

Alle jubelten ihm zu. Es gab ihm ein warmes Gefühl, war aber gleichzeitig auch unangenehm. 

Sie jubelten ihm wegen etwas zu, was er getan hatte, als er ein Jahr alt war. Etwas, was er nicht einmal richtig abgeschlossen hatte. Irgendwo, irgendwie war der Dunkle Lord noch am Leben. Würden sie wohl auch so laut jubeln, wenn sie das wüssten? 

Aber die Macht des Dunklen Lords \emph{war} einmal gebrochen worden. 

Und Harry würde sie wieder beschützen. Wenn es tatsächlich eine Prophezeiung gab, und die das vorhersagte. Nun, eigentlich unabhängig davon, was irgendeine verdammte Prophezeiung besagte. 

All diese Leute glaubten an ihn und jubelten ihm zu – Harry könnte es nicht ertragen, sie zu enttäuschen. Zu leuchten und zu verblassen, wie so viele andere Wunderkinder. Eine Enttäuschung zu sein. Seiner Reputation als Symbol des Lichtes nicht gerecht zu werden, unabhängig davon, \emph{wie} er sie bekommen hatte. Er würde absolut, ganz sicher, und koste es, was es wolle – solange es auch dauerte und was auch immer es ihm abverlangte, auch wenn es ihn umbrachte – ihre Erwartungen erfüllen. Und diese dann noch \emph{übertreffen}, sodass sich die Leute im Nachhinein fragen würden, warum sie nur so wenig von ihm verlangt hatten. 

Und er schrie die eine Lüge, die er sich ausgedacht hatte, weil sie so gut klang und der Song danach verlangte, laut in die Welt hinaus: 

\begin{center}
\emph{I ain’t afraid of Dark Lords!\\
I ain’t afraid of Dark Lords!}
\end{center}

Harry machte seine letzten Schritte zum Sprechenden Hut, als die Musik endete. Er verbeugte sich vor dem Orden des Chaos am Gryffindortisch, drehte wandte sich dann der anderen Seite der Halle zu, verbeugte sich wieder und wartete, bis der Applaus und das Kichern langsam abebbten. 

\later 

\section{Omake III: Alternative Enden von „Selbst-Bewusstsein“}

Das Angebot, jedem, der vor der Veröffentlichung von Kapitel 10 errät, was „noch nie passiert“ ist, hat zu einer ganzen Menge interessanter Versuche geführt. Der erste Text unten stammt direkt aus meiner persönlichen Lieblingsantwort von Meteoricshipyards. Der zweite basiert auf Kazumas Vorschlag, der dritte auf einer Mischung aus yoyoente und dougal74, der vierte auf dem Review von wolf550 zu Kapitel 10. Der Text, der mit „K“ beginnt und der direkt darüber stammen von DarkHeart81. Die Anderen sind von mir. Jeder, der eine meiner eigenen Ideen aufgreifen und weiterschreiben möchte (insbesondere die letzte) ist herzlich dazu eingeladen. Und bevor ich 100 empörte Beschwerden bekomme, ja, es ist mir durchaus bewusst, dass die Legislative im UK das House of Commons ist. 

\later

In der hintersten Ecke seines Kopfes fragte er sich, ob der Sprechende Hut tatsächlich bewusst handelte, in dem Sinne, dass er sich seines eigenen Bewusstseins bewusst war – und falls ja, ob es ihn wirklich zufriedenstellte, ein einziges Mal im Jahr mit Elfjährigen zu sprechen. Sein Lied hatte sich danach angehört: \emph{„Oh, ich bin der Hut und mir geht‘s gut / Der einmal im Jahr seine Arbeit tut …“} 

Als es wieder still wurde, setzte Harry sich auf den Stuhl und platzierte das 800 Jahre alte telepathische Artefakt vergessener Magie \emph{vorsichtig} auf seinem Kopf. 

Dabei dachte er, so sehr er nur konnte: \emph{Steck mich noch nicht in ein Haus! Ich habe Fragen an Dich! Wurde jemals ein Vergessenszauber auf mich gesprochen? Hast du den Dunklen Lord in ein Haus gesteckt, und kannst du mir von seinen Schwächen erzählen? Kannst du mir sagen, warum mein Zauberstab mit dem des Dunklen Lords verwandt ist? Ist der Geist des Dunklen Lords mit meiner Narbe verbunden, und ist das der Grund dafür, dass ich manchmal so zornig werde? Das sind die wichtigsten Fragen, aber falls du noch etwas Zeit hast, kannst du mir etwas darüber erzählen, wie ich die verlorene Magie wiederentdecke, die dich einst geschaffen hat?} 

Und der Sprechende Hut antwortete: „Nein. Ja. Nein. Nein. Ja und nein, stelle beim nächsten Mal keine kombinierten Fragen. Nein.“ Und laut: „RAVENCLAW!“ 

\later

\emph{„Meine Güte. Das ist ja noch nie passiert …“}

\emph{Was?}

\emph{„Ich reagiere auf dein Haarshampoo allergisch –“}

Und dann nieste der Sprechende Hut und ein mächtiges „HA-TSCHI!“ hallte von den Wänden wider. 

„Na dann!“, rief Dumbledore heiter. „Wie es scheint, wurde Harry Potter in das neue Haus Hatschi gesteckt! Minerva, du wirst die neue Hauslehrerin vom Haus Hatschi. Du solltest dich mit dem Stundenplan beeilen, morgen ist der erste Schultag!“ 

„Aber, aber, aber“, stotterte McGonagall, in deren Kopf völliges Durcheinander herrschte, „wer soll dann Hauslehrer von Gryffindor sein?“ Es war alles, was ihr einfiel, sie \emph{musste} das irgendwie stoppen … 

Dumbledore legte einen Finger an die Wange und sah nachdenklich aus. „Snape.“ 

Snapes Protestschrei übertönte fast McGonagalls Frage, „Aber wer soll dann Hauslehrer von \emph{Slytherin} sein?“ 

„Hagrid.“ 

\later

\emph{Steck mich noch nicht in ein Haus! Ich habe Fragen an Dich! Wurde jemals ein Vergessenszauber auf mich gesprochen? Hast du den Dunklen Lord in ein Haus gesteckt, und kannst du mir von seinen Schwächen erzählen? Kannst du mir sagen, warum mein Zauberstab mit dem des Dunklen Lords verwandt ist? Ist der Geist des Dunklen Lords mit meiner Narbe verbunden, und ist das der Grund dafür, dass ich manchmal so zornig werde? Das sind die wichtigsten Fragen, aber falls du noch etwas Zeit hast, kannst du mir etwas darüber erzählen, wie ich die verlorene Magie wiederentdecke, die dich einst geschaffen hat?} 

Einen Moment lang war es still. 

\emph{Hallo? Soll ich die Fragen wiederholen?} 

Der Sprechende Hut schrie; ein schrecklich schrilles Geräusch, das von den Wänden widerhallte und die meisten Schüler dazu brachte, sich die Ohren zuzuhalten. Mit einem verzweifelten Jaulen sprang er von Harry Potters Kopf, hüpfte über den Boden, wobei er sich mit seiner Krempe voranschob, und legte den halben Weg zum Lehrertisch hinter sich, bevor er explodierte. 

\later

„SLYTHERIN!“ 

Als er Harrys vor Schrecken bleiches Gesicht sah, dachte Fred Weasley schneller, als er je in seinem Leben gedacht hatte. In einer einzigen Bewegung zog er seinen Zauberstab, flüsterte \emph{„Silencio!“}, dann \emph{„Änderemeinestimmio!“} und schließlich \emph{„Ventriliquo!“} 

„Nur ein Scherz!“, sagte Fred. „GRYFFINDOR!“ 

\later

\emph{„Meine Güte. Das ist ja noch nie passiert …“}

\emph{Was?}

\emph{„Normalerweise würde ich dich mit der Frage an den Schulleiter verweisen, der wiederum mich fragen könnte, wenn er wollte. Aber einige der Informationen, nach denen du fragst, sind nicht nur mit deinen Benutzerrechten nicht verfügbar, sondern nicht mal mit denen des Schulleiters.“}

\emph{Wie kann ich meine Benutzerrechte erweitern?}

\emph{„Ich fürchte, diese Frage kann ich angesichts deiner derzeitigen Benutzerrechte nicht beantworten.“}

\emph{Welche Möglichkeiten} habe \emph{ich mit meinen derzeitigen Benutzerrechten?} 

Danach dauerte es nicht mehr lange … 

„ROOT!“ 

\later

\emph{„Meine Güte. Das ist ja noch nie passiert …“}

\emph{Was?}

\emph{„Ich musste schon Schülerinnen mitteilen, dass sie Mütter sind – es würde dir das Herz brechen, wenn du wüsstest, was ich in ihren Gedanken sah – aber das ist das erste Mal, dass ich jemandem sagen muss, dass er ein Vater ist.“}

\emph{\shout{Was?}}

\emph{„Draco Malfoy ist von dir schwanger.“}

\emph{\shout{Waaaaas?}}

\emph{„Ich wiederhole: Draco Malfoy ist von dir schwanger.“}

\emph{Aber wir sind erst elf –}

\emph{„Draco ist in Wirklichkeit dreizehn Jahre alt.“}

\emph{A-a-aber Jungen können nicht schwanger werden –}

\emph{„Und er ist ein Mädchen.“}

\emph{\shout{Aber wir hatten nie Sex, du Idiot!}}

\emph{\shout{„Sie hat nach der Vergewaltigung deine Erinnerungen gelöscht, Dummkopf!“}}

Harry Potter verlor das Bewusstsein. Sein lebloser Körper fiel mit einem dumpfen Plumps vom Hocker. 

„RAVENCLAW!“, rief der Hut, der auf Harrys Kopf am Boden lag. Das war noch lustiger gewesen als sein erster Einfall. 

\later

„ELF!“ 

Hä? Harry erinnerte sich daran, dass Draco irgendwas mit „Haus…elf“ gesagt hatte, aber was genau sollte das sein? 

Nach den entsetzten Gesichtern zu urteilen, die ihn ansahen, war es nichts Gutes … 

\later

„LEBKUCHEN!“ 

\later

„REPRÄSENTANTEN!“ 

\later

\emph{„Meine Güte. Das ist ja noch nie passiert …“}

\emph{Was?}

\emph{„Ich habe noch niemanden in ein Haus gesteckt, der eine Wiedergeburt von Godric Gryffindor UND Salazar Slytherin UND Naruto war.“}

\later

„ATREIDES!“ 

\later

„Wieder reingelegt! HUFFLEPUFF! SLYTHERIN! HUFFLEPUFF!“ 

\later

„REINGELEGTE SCHMERDBEEREN!“ 

\later

„KHAAANNNN!“ 

\later

Am Lehrertisch lächelte Dumbledore weiterhin gütig; gelegentliche leise, metallische Geräusche kamen aus Snapes Richtung, während er die verbogenen Überreste eines schweren silbernen Weinkelches zusammenpresste; Minerva McGonagall umklammerte derweil mit weißen Knöcheln die Tischplatte. Sie wusste, dass Harry Potters ansteckendes Chaos irgendwie den Sprechenden Hut selbst angesteckt hatte. 

Minervas Kopf spielte ein Szenario nach dem anderen durch; eines schlimmer als das andere. Der Hut würde sagen, dass Harry zu ausgeglichen sei, als dass er in ein Haus gesteckt werden könne, und entscheiden, dass er in alle gehöre. Der Hut würde ausrufen, dass Harrys Gehirn zu seltsam sei, um ihn in ein Haus einzuteilen. Der Hut würde fordern, dass Harry aus Hogwarts rausgeworfen werde. Der Hut ist in ein Koma gefallen. Der Hut würde darauf bestehen, dass nur für Harry Potter ein vollkommen neues Haus der Verdammnis geschaffen wird – und \emph{Dumbledore würde sie dazu bringen …} 

Minerva erinnerte sich, was Harry ihr während dem schrecklichen Ausflug in die Winkelgasse erzählt hatte, vom … Planungstrugschluss hieß es, dachte sie … und dass Menschen normalerweise viel zu optimistisch wären, selbst wenn sie dachten, sie wären pessimistisch. Es war eine der Informationen, die im Gehirn blieben, sich dort festsetzten und Albträume produzierten … 

Aber was war das \emph{Schlimmste,} was passieren könnte? 

Nun … im \emph{allerschlimmsten Fall} würde der Hut Harry in ein vollkommen neues Haus stecken. Dumbledore würde darauf bestehen, dass sie es tut – nur für ihn ein vollkommen neues Haus erschafft – und sie müsste gleich am ersten Tag alle Stundenpläne abändern. Und Dumbledore würde sie als Hauslehrerin von Gryffindor absetzen und ihr geliebtes Haus … Professor Binns übergeben, dem Geschichts-Geist; sie würde Hauslehrerin von Harrys Haus der Verdammnis werden und völlig umsonst versuchen, das Kind zu ermahnen, würde Punkte über Punkte abziehen, ohne damit irgendwas zu erreichen, während man sie für ein Desaster nach dem anderen verantwortlich machen würde. 

War das der allerschlimmste Fall? 

Minerva konnte sich wirklich nicht vorstellen, wie es noch schlimmer kommen könnte. 

Und selbst im allerschlimmsten Fall – egal \emph{was} mit Harry auch geschah – würde es nach sieben Jahren vorbei sein. 

Minerva hielt sich immer noch am Podium fest, merkte aber, wie sich langsam ihr verkrampfter Griff lockerte. Harry hatte Recht gehabt, es war in gewisser Weise beruhigend, direkt in die tiefste Dunkelheit zu starren; zu wissen, dass man sich seinen schlimmsten Ängsten gestellt hatte und nun darauf vorbereitet war. 

Die angstvolle Stille wurde von einem einzelnen Wort durchbrochen. 

„Schulleiter!“, rief der Sprechende Hut. 

Am Lehrertisch erhob Dumbledore sich mit verwirrtem Gesichtsausdruck. „Ja?“, sprach er den Hut an. „Was gibt es?“ 

„Ich habe nicht mit Ihnen gesprochen“, sagte der Hut. „Ich habe Harry Potter dorthin gesteckt, wo er am besten hinpasst, nämlich ins Büro des Schulleiters.“
