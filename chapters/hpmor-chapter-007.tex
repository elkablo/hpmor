\chapter{Erwiderung}

\emph{„Dein Vater ist fast so toll wie mein Vater.“} 

\later 

\lettrine{P}{etunia} Evans-Verres' Lippen bebten und ihre Augen füllten sich mit Tränen, als Harry ihr auf Gleis Neun des Bahnhofs King's Cross um den Hals fiel. „Bist du sicher, dass du nicht willst, dass ich mit dir komme, Harry?“ 

Harry schaute zu ihr auf. Sein Blick streifte seinen Vater Michael Verres-Evans, der stereotyp ernst-aber-stolz dreinschaute, und dann wieder seine Mutter, die ziemlich … aufgewühlt wirkte. „Mama, ich weiß, dass du die Zaubererwelt nicht sehr magst. Du musst nicht mitkommen. Wirklich nicht.“ 

Petunia zuckte zusammen. „Harry, um mich solltest du dir keine Sorgen machen, ich bin deine Mutter und wenn du jemanden bei dir brauchst -“ 

„Mama, ich werde in Hogwarts \emph{monatelang} alleine sein. Wenn ich schon mit dem Gleis nicht alleine fertig werde, sollten wir das lieber gleich herausfinden, so dass wir das Ganze abbrechen können.“ Er senkte seine Stimme zu einem Flüstern. „Außerdem, Mama, lieben mich dort alle. Wenn ich irgendwelche Probleme habe, muss ich nur mein Stirnband abnehmen“, Harry deutete auf das Schweißband, das seine Narbe verdeckte, „und ich werde \emph{sehr viel mehr} Hilfe haben, als mir lieb ist.“ 

„Oh, Harry“, flüsterte Petunia. Sie kniete nieder und umarmte ihn fest, Gesicht an Gesicht, ihre Wangen aneinander ruhend. Harry konnte ihr stoßweises Atmen fühlen, und dann hörte er, wie ein Schluchzer ihren Lippen entwich; erstickt und dumpf zwar, aber er war da. „Oh, Harry, ich liebe dich, vergiss das nie.“ 

\emph{Es ist, als hätte sie Angst, mich nie wieder zu sehen} – der Gedanke tauchte plötzlich in Harrys Kopf auf. Er wusste, dass der Gedanke richtig war, aber er wusste nicht, warum seine Mutter solche Angst hatte. 

Also versuchte er zu raten. „Mama, du weißt, dass ich mich nicht in deine Schwester verwandeln werde, bloß, weil ich zaubern lerne, ja? Ich werde jeden Zauber ausführen, den du haben willst – wenn ich kann, meine ich – oder wenn du nicht willst, dass ich zu Hause zaubere, werde ich das auch nicht machen. Ich verspreche dir, ich werde die Zauberei niemals zwischen uns –“ 

Eine enge Umarmung unterbrach seine Worte. „Du hast ein gutes Herz“, flüsterte seine Mutter ihm ins Ohr, „ein sehr gutes Herz, mein Sohn.“ 

Harry spürte einen Kloß im Hals. 

Seine Mutter ließ ihn los und stand auf. Sie nahm ein Taschentuch aus ihrer Jackentasche und tupfte sich mit zitternder Hand über die Augen und die verlaufene Schminke. 

Er brauchte gar nicht fragen, ob sein Vater ihn auf die magische Seite des Bahnhofs King's Cross begleiten wollte. Sein Vater hatte schon Schwierigkeiten, Harrys Koffer direkt anzuschauen. Zauberei lag in der Familie, aber Michael Verres-Evans hatte keinen Sinn dafür. 

Stattdessen räusperte sich sein Vater bloß. „Viel Glück in der Schule, Harry“, sagte er. „Glaubst du, ich habe dir genug Bücher gekauft?“ 

Harry hatte seinem Vater erklärt, dass dies womöglich seine große Chance wäre, etwas wirklich Revolutionäres und Bedeutendes zu erreichen – woraufhin Professor Verres-Evans genickt und zwei Tage in seinem Terminkalender freigeschaufelt hatte, an denen sie im Rahmen der \emph{Größten Antiquariatstour Aller Zeiten} vier Städte durchkämmt und insgesamt \emph{dreißig} Kisten wissenschaftlicher Bücher gekauft hatten, die nun in den Tiefen von Harrys Koffer verstaut waren. Die meisten davon hatten nur ein oder zwei Pfund gekostet, doch einige andere waren definitiv \emph{nicht} so billig gewesen – etwa die aktuellste Ausgabe des \emph{Handbook of Chemistry and Physics} oder die vollständige 1972er-Ausgabe der \emph{Encyclopaedia Britannica}. Sein Vater hatte zwar den Kassenzettel schnell eingesteckt, doch Harry hatte überschlagen, dass all die Bücher \emph{mindestens} tausend Pfund gekostet hatten. Harry hatte seinem Vater gesagt, dass er alles zurückzahlen würde, sobald er wüsste, wie man Zauberergold in Muggelgeld umtauschte, woraufhin sein Vater erwidert hatte, das möge er sich ganz schnell aus dem Kopf schlagen. 

Und jetzt fragte er also: \emph{Glaubst du, ich habe dir genug Bücher gekauft?} Harry wusste, welche Antwort sein Vater hören wollte. 

Aus irgendeinem Grund klang Harry etwas heiser. „Man kann nie genug Bücher haben“, zitierte er das Familienmotto der Verres, und sein Vater beugte sich zu ihm runter und umarmte ihn kurz und fest. „Aber du hast es \emph{tatsächlich} versucht“, sagte Harry und spürte wieder den Kloß im Hals. „Und es war ein wirklich, wirklich, \emph{wirklich} guter Versuch.“ 

Sein Vater stand auf. „So …“, sagte er. „Siehst \emph{du} hier irgendwo ein Gleis neundreiviertel?“ 

King's Cross war riesig und wirkte hektisch, Wände und Boden mit ganz normalen, dreckigen Fliesen bedeckt, voller ganz normaler Menschen, die ihrem ganz normalen Alltag nachgingen und ganz normale Gespräche führten, die ganz normalen Lärm verursachten. King's Cross hatte ein Gleis neun (an dem sie standen) und ein Gleis zehn (direkt gegenüber), aber dazwischen war absolut nichts, außer einigen nichtssagenden Mauerpfeilern. Große Fenster im Dach ließen genug Licht herein, um die vollkommene Nichtexistenz eines Gleises neundreiviertel zweifellos ersichtlich zu machen. 

Harry starrte suchend umher bis seine Augen feucht wurden und dachte dabei \emph{komm schon, magischer Blick, komm schon!}, doch er sah nichts Neues. Er überlegte, ob er seinen Zauberstab rausholen und umherwedeln sollte, doch McGonagall hatte ihn gewarnt, das nicht zu tun. Falls wieder bunte Funken aus dem Zauberstab rauskämen, liefe er zudem Gefahr, verhaftet zu werden, weil er mitten im Bahnhof einen Feuerwerkskörper gezündet habe. Ganz zu schweigen davon, dass sein Zauberstab auch ganz andere Dinge tun könnte, beispielsweise den gesamten Bahnhof in die Luft jagen. Harry hatte seine Schulbücher bisher nur kurz überflogen (obwohl selbst das schon bizarr genug gewesen war) um herauszufinden, welche Art von Büchern er in den nächsten 48 Stunden heraussuchen musste. 

Nun, ihm blieb noch – Harry sah auf die Uhr – eine ganze Stunde um das Rätsel zu lösen, da er um elf Uhr im Zug sitzen sollte. Vielleicht war das eine Art IQ-Test und die dummen Kinder durften keine Zauberer werden. (Und die Zeit, die man zum Schluss noch übrig hatte, bestimmte, wie gewissenhaft man an das Problem herangegangen war – den zweitwichtigsten Faktor für wissenschaftlichen Erfolg.) 

„Ich werde es rausfinden“, sagte Harry seinen wartenden Eltern. „Es ist vermutlich so eine Art Test.“ 

Sein Vater runzelte die Stirn. „Hm … vielleicht musst du nach auffälligen Fußspuren auf dem Boden suchen, die an irgendeinen völlig unerwarteten Ort führen.“ 

„\emph{Papa!}“, rief Harry. „Hör auf! Ich habe nicht mal \emph{angefangen}, selbst nachzudenken!“ Außerdem war es ein wirklich guter Vorschlag, was das ganze noch schlimmer machte. 

„Entschuldige“, beschwichtigte sein Vater ihn. 

„Aber …“, sagte seine Mutter. „Ich glaube nicht, dass sie das einem Schüler antun würden, oder? Bist du dir sicher, dass Professor McGonagall nichts gesagt hat?“ 

„Vielleicht war sie abgelenkt“, sagte Harry, ohne drüber nachzudenken. 

„\emph{Harry!}“, zischten sein Vater und seine Mutter unisono. „\emph{Was hast du angestellt?}“ 

„Ich, ähm …“ Harry schluckte. „Schaut mal, wir haben jetzt keine Zeit für –“ 

„\emph{Harry!}“ 

„Ich meine es ernst! Wir haben jetzt keine Zeit dafür! Es ist eine lange Geschichte und ich muss erstmal rausfinden, wie ich zur Schule komme!“ 

Seine Mutter hatte die Hände vors Gesicht geschlagen. „Wie schlimm war es?“ 

„Naja …“ \emph{Ich kann aus Gründen nationaler Sicherheit nicht drüber sprechen.} „ungefähr halb so schlimm wie die Angelegenheit mit dem Jugend Forscht-Projekt?“ 

„\emph{Harry!}“ 

„Ich, ähm, oh, schaut mal, da sind Leute mit einer Eule! Ich werde sie mal fragen, wie ich zum Gleis komme!“ Und Harry rannte, gefolgt von seinem Koffer, von seinen Eltern weg, hin zu der Familie aus Rotschöpfen. 

Die mollige Frau wandte sich zu ihm, als er näher kam. „Hallo mein Lieber! Erstes Mal nach Hogwarts? Ron ist auch neu –“ und dann erstarrte sie. Sie musterte ihn genau. „\emph{Harry Potter?}“ 

Vier Jungs, ein rothaariges Mädchen und eine Eule drehten sich auf dem Fleck zu ihm um und erstarrten dann ebenso. 

„Ach, kommt schon!“, protestierte Harry. Er hatte eigentlich geplant, den Namen Mr Verres zu tragen bis sie in Hogwarts ankamen. „Ich habe mir extra ein Stirnband gekauft! Woher wisst ihr, dass ich es bin?“ 

„Eine sehr gute Frage“, sagte Harrys Vater, der mit langen Schritten herantrat, „woher wissen Sie, wer er ist?“ Seine Stimme hatte einen sorgenvollen Unterton. 

„Dein Bild war in der Zeitung“, sagte einer der identisch aussehenden Zwillinge. 

„\emph{HARRY!}“ 

„\emph{Papa!} Es ist nicht so, wie du denkst! Das liegt daran, dass ich den Dunklen Lord Du-weißt-schon-wer besiegt habe als ich ein Jahr alt war.“ 

„\emph{WAS?}“ 

„Mama kann dir alles erklären.“ 

„\emph{WAS?}“ 

„Ähm … Michael, Schatz … es gibt da ein paar Sachen, mit denen ich dich bisher lieber nicht belasten wollte …“ 

„Entschuldigen Sie“, sagte Harry zu der rothaarigen Familie, die ihn allesamt anstarrten, „aber es wäre außerordentlich hilfreich, wenn Sie mir sagen könnten, wie ich \emph{schnellstmöglich} auf Gleis neundreiviertel komme.“ 

„Ähm …“, sagte die Frau. Sie zeigte mit der Hand auf einen die Mauer zwischen den Bahnsteigen. „Lauf einfach schnurstracks auf die Absperrung zwischen Gleis neun und Gleis zehn zu. Halt nicht an und hab keine Angst, du könntest dagegenknallen, das ist sehr wichtig! Renn lieber ein bisschen, wenn du nervös bist.“ 

„Und egal was du tust, denke auf keinen Fall an einen Elefanten!“ 

„\emph{George!} Ignoriere ihn, Harry, mein Lieber. Es gibt absolut keinen Grund, nicht an einen Elefanten zu denken.“ 

„Ich bin Fred, Mutter, nicht George.“ 

„Danke“, sagte Harry und rannte auf die Absperrung zu – 

Moment mal, es würde nicht funktionieren, \emph{außer wenn er daran glaubte?} 
In Momenten wie diesen hasste Harry seinen Verstand dafür, dass er schnell genug arbeitete um zu bemerken, dass hier ein selbsterfüllender Zweifel vorlag; das heißt, wenn er geglaubt hätte, dass er durch die Absperrung kommen würde, wäre alles glatt gegangen – aber jetzt, wo er sich Sorgen machte, ob er ausreichend daran \emph{glaubte}, durch die Absperrung zu kommen, wo er sich also Sorgen machte, ob er dagegenknallen würde – 

„\emph{Harry! Komm sofort zurück, du hast einiges zu erklären!}“ Das war sein Vater. 

Harry schloss die Augen und ignorierte alles, was er über gerechtfertigte Annahmen wusste und versuchte einfach, richtig fest daran zu glaube, dass er durch die Absperrung kommen würde und – 

– und die Geräusche um ihn herum änderten sich. 

Harry öffnete die Augen und blieb stehen. Er fühlte sich ein bisschen schäbig, weil er tatsächlich versucht hatte, etwas zu glauben. 

Er stand auf einem hellen Außenbahnsteig neben einem langen Zug aus vierzehn Wagen und einer großen, scharlachroten Dampflok, deren graue Rauchsäule wohl die ganze Fahrt über geschlossene Fenster verhieß. Der Bahnsteig war bereits einigermaßen gefüllt (obwohl Harry eine ganze Stunde zu früh erschienen war) und dutzende Kinder mitsamt ihrer Eltern waren an Bänken, Tischen und einigen Verkaufsständen verteilt. 

Es verstand sich von selbst, dass ein solcher Ort im Bahnhof King's Cross nicht existierte und dass auch gar kein Platz da war, um ihn zu verstecken. 

\emph{Okay, also entweder (a) wurde ich gerade an einen völlig anderen Ort teleportiert, oder (b) die können Raumdimensionen zusammenfalten, als ob es nichts wäre, oder (c) die ignorieren einfach sämtliche Regeln.} 

Harry hörte ein schlitterndes Geräusch hinter sich und drehte sich um, so dass er sah, dass sein Koffer ihm tatsächlich auf den kleinen, klauenbesetzten Tentakeln hinterhergekrochen war. Anscheinend hatte sein Gepäck es aus magischen Gründen ebenfalls geschafft, ausreichend stark zu glauben, um die Absperrung durchqueren zu können. Wenn Harry genauer darüber nachdachte, war das ziemlich verstörend. 

Einen Moment später kam der jüngste rothaarige Junge durch das schmiedeeiserne Tor (ein schmiedeeisernes Tor?) gerannt und schaffte es gerade so, vor Harry abzubremsen. Harry kam sich etwas dumm vor, weil er nicht daran gedacht hatte, aus dem Weg zu gehen. Er eilte ein paar Schritte beiseite und der rothaarige Junge folgte ihm und zerrte seinen Koffer hinter sich her. Einen Moment später flog eine weiße Eule durch das Tor und setze sich auf der Schulter des Jungen nieder. 

„Wow“, sagte der Junge. „Bist du wirklich Harry Potter?“ 

\emph{Nicht das schon wieder.} „Ich habe keine Möglichkeit, das zu wissen. Meine Eltern haben mich in dem Glauben aufgezogen, dass ich Harry Potter bin und viele Leute hier haben mir mitgeteilt, dass ich genau wie meine Eltern – ich meine, wie meine anderen Eltern – aussehe, aber“, Harry brach ab und dachte kurz nach, „nach allem was \emph{ich} weiß, könnte es Zaubersprüche geben, die das Aussehen eines kleinen Kindes nach Wunsche verwandeln und –“ 

„Ähm, wie bitte?“ 

\emph{Okay, schonmal kein Ravenclaw, so viel ist sicher.} „Ja, ich bin Harry Potter.“ 

„Ich bin Ron Weasley“, sagte das große, dünne, sommersprossige Kind und streckte eine Hand aus, die Harry im Gehen höflich schüttelte. Die Eule begrüßte Harry mit einem bemerkenswert gewichtigem und höflichem Zwitschern (genauer gesagt war es eine Art \emph{iiieehhhk}-Geräusch, was Harry überraschte). 

In dem Moment bemerkte Harry die Möglichkeit einer unmittelbar bevorstehenden Katastrophe und fand einen Weg, sie abzuwenden. „Warte kurz“, sagte er zu Ron, öffnete die Schublade seines Koffers, die – wenn er sich richtig erinnerte – Wintersachen enthielt – das tat sie – und holte seinen dünnsten Schal unter dem Wintermantel hervor. Harry nahm sein Stirnband ab, entfaltete blitzschnell den Schal und wickelte ihn um sein Gesicht. Es war etwas heiß, gerade jetzt im Sommer, doch damit konnte Harry leben. 

Dann schloss er die Schublade (die jetzt das nutzlose Stirnband enthielt, obwohl es dort gar nicht hingehörte) und holte aus einer anderen seinen schwarzen Zaubererumhang hervor, den er sich jetzt, da sie außer Sichtweite von Muggeln waren, überwarf. 

„So“, sagte Harry zufrieden. Die Töne kamen etwas dumpf unter dem Schal hervor. Er drehte sich zu Ron. „Wie sehe ich aus? Dämlich, ich weiß, aber bin ich noch als Harry Potter erkenntlich?“ 

„Ähm“, sagte Ron. Er schloss seinen Mund, der bisher offen gestanden hatte. „Nicht wirklich, Harry.“ 

„Sehr gut“, sagte Harry. „Außerdem wirst du mich, um die ganze Mühe nicht zunichte zu machen, von nun an mit“ – Verres könnte jetzt nicht mehr ausreichen – „Mr Spoo anreden.“ 

„Okay, Harry“, sagte Ron zögerlich. 

\emph{Die Macht ist definitiv nicht mit ihm.} „Nenn … mich … Mr … Spoo.“ 

„Okay, Mr Spoo“ – Ron zögerte. „Ich kann das nicht tun, ich komm mir dabei dumm vor.“ 

\emph{Du kommst dir nicht nur so vor.} „Okay. Dann denk \emph{du} dir einen Namen aus.“ 

„Mr Cannon“, sagte Ron sofort. „Wie die Chudley Cannons.“ 

„Ähm …“ Harry hatte eine düstere Vorahnung, dass er diese Frage schrecklich bereuen würde. „Wer oder was sind die Chudley Cannons?“ 

„\emph{Wer sind die Chudley Cannons?} Die sind das genialste Team in der gesamten Geschichte des Quidditch. Klar, sie waren letzte Saison ganz unten in der Liga, aber –“ 

„Was ist Quidditch?“ 

Auch diese Frage war ein Fehler. 

„Verstehe ich das richtig“, sagte Harry, als Rons Erklärung (samt ausschweifenden Gestikulierens) sich dem Ende zuneigte. „Den Schnatz zu fangen ist \emph{einhundertfünfzig} Punkte wert?“ 

„Ja …“ 

„Wenn man den Schnatz mal beiseite lässt; wie viele Zehn-Punkte-Tore schafft eine Mannschaft üblicherweise pro Spiel?“ 

„Ähm, vielleicht fünfzehn oder zwanzig, wenn Profimannschaften spielen.“ 

„Das ist einfach \emph{falsch}. Das verletzt sämtliche Regeln der Spielgestaltung. Ich meine, der Rest des Spiels klingt ja noch halbwegs sinnvoll – für eine Sportart zumindest –, aber den Schnatz zu fangen wirft dann fast jede realistische Punkteverteilung um. Die zwei Sucher fliegen umher, suchen nach dem Schnatz und interagieren üblicherweise nicht miteinander und wer den Schnatz zuerst sieht, ist meist eine Frage des Glücks –“ 

„Es ist kein Glück“, protestierte Ron. „Man muss die Augen richtig hin und her bewegen und –“ 

„Das ist aber nicht interaktiv, da gibt es keine Berührungspunkte mit dem anderen Sucher und überhaupt: Wie spannend ist es denn, jemandem zuzuschauen, der unglaublich gut darin ist, seine Augen zu bewegen? Und der Sucher, der mehr Glück hat, findet den Schnatz zuerst, schnappt ihn sich und macht dann die Arbeit aller anderen hinfällig. Das ist so, als ob jemand ein richtiges Spiel genommen hat und diese zusätzliche Position hinzugefügt hat, nur damit irgendjemand \emph{Der Wichtigste Spieler} sein konnte, ohne wirklich mitzuspielen oder die Regeln zu kennen. Wer war denn der erste Sucher – der dümmliche Königssohn, der mitspielen wollte, aber die Regeln nicht verstand?“ 
Nun, wo Harry genauer drüber nachdachte, schien das eine erstaunlich gute Hypothese zu sein. \emph{Setze ihn einfach auf einen Besen und sag ihm, er soll das glitzernde Ding fangen.} 

Ron schaute finster drein. „Bloß weil du Quidditch nicht magst, brauchst du dich nicht darüber lustig machen!“ 

„Wenn man nichts kritisiert, kann man nichts optimieren. Ich mache Vorschläge, wie man das Spiel \emph{verbessern} könnte. Und es ist ganz einfach: Schafft den Schnatz ab.“ 

„Sie werden das Spiel nicht ändern, bloß weil \emph{du} es so willst!“

„Weißt du, ich bin der Junge-der-lebt. Die Leute hören mir zu. Und falls ich sie überzeugen kann, das Spiel auf Hogwarts zu ändern, wird sich die Innovation vielleicht ausbreiten.“ 

Große Furcht zeichnete sich auf Rons Gesicht ab. „Aber, aber, aber … wenn du den Schnatz abschaffst, wie soll man dann wissen, wann das Spiel zu Ende ist?“ 

„\emph{Kauft … eine … Uhr.} Das wäre viel fairer als wenn das Spiel mal nach zehn Minuten endet und mal nach vielen Stunden immer noch läuft; und für die Zuschauer wäre das auch viel einfacher.“ Harry seufzte. „Ach, lass doch diesen entsetzten Blick, ich werde vermutlich ohnehin nicht die Zeit haben, diesen lächerlichen Abklatsch einer Nationalsportart zu zerstören und nach meinem Entwurf neu zu erschaffen. Ich habe sehr, \emph{sehr} viel wichtigere Sachen zu tun.“ Harry sah gedankenverloren drein. „Andererseits würde es vermutlich nicht allzu viel Zeit in Anspruch nehmen, die 95 Thesen der Schnatzlosen Reformation niederzuschreiben und an eine Kirchentür zu nageln …“ 

„Potter“, schnarrte eine Jungenstimme hinter ihm. „\emph{Was} hast du da um dein Gesicht gewickelt und \emph{was} steht da neben dir?“ 

Rons Furcht wurde von abgrundtiefem Hass verdrängt. „\emph{Du!}“ 

Harry drehte seinen Kopf; es war tatsächlich Draco Malfoy, der zwar die normale Schuluniform tragen musste, dies allerdings mit einem Koffer wettmachte, der mindestens ebenso magisch und weitaus eleganter aussah als Harrys eigener, dekoriert mit Silber und Smaragden und versehen mit einer wunderschönen, langzahnigen Schlange, die sich um zwei gekreuzte Zauberstäbe wand – das Familienwappen der Malfoys, vermutete Harry. 

„Draco“, sagte Harry. „Ähm, oder Malfoy, falls dir das lieber ist, obwohl ich dabei eher an Lucius denken muss. Ich bin froh, dass es dir nach unserem letzten Treffen so gut geht. Das ist Ron Weasley. Und ich versuche nicht erkannt zu werden, also nenne mich bitte“, Harry sah nachdenklich auf seinen Umhang, „Mr Black.“ 

„\emph{Harry}“, zischte Ron. „Du kannst diesen Namen nicht benutzen!“ 

Harry blinzelte. „Wieso nicht?“ Es klang schön düster, wie ein international berühmter Mann der Mysterien. 

„Ich halte es für einen angemessenen Namen“, sagte Draco, „aber das Noble und Uralte Haus der Blacks wäre wohl nicht einverstanden. Wie wäre es mit Mr Silber?“ 

„Halt \emph{du} dich ja fern von … von Mr Gold“, sagte Ron hitzig und ging einen Schritt vor. „Er hat es nicht nötig, mit jemandem wie dir zu reden!“ 

Harry hob beschwichtigend die Hände. „Ich werde mich wohl Mr Bronze nennen, vielen Dank für das Namensschema. Und, Ron, ähm …“ Harry überlegte angestrengt, wie er das sagen sollte. „Ich bin froh, dass du mich so … enthusiastisch verteidigst, aber es stört mich nicht, mit Draco zu reden.“ 

Das brachte bei Ron offenbar das Fass zum Überlaufen, denn jetzt drehte er sich mit vor Empörung flammenden Augen zu Harry. „\emph{Was? Hast du überhaupt eine Ahnung, wer das ist?}“ 

„Ja, Ron“, sagte Harry. „Du erinnerst dich vielleicht, dass ich ihn mit 'Draco' ansprach, ohne dass er sich vorgestellt hat.“ 

Draco kicherte. Dann strahlten seine Augen, als er die weiße Eule auf Rons Schulter bemerkte. „Oh, was ist \emph{das} denn?“, schnarrte Draco mit einem boshaften Unterton. „Wo ist die berühmte Ratte der Weasleys?“ 

„Im Garten begraben“, sagte Ron kühl. 

„Oh, wie traurig. Pot – ähm, Mr Bronze, ich sollte erwähnen, dass die Familie Weasley allseits für die \emph{beste Haustier-Geschichte aller Zeiten} bekannt ist. Magst du sie erzählen, Weasley?“ 

Rons Gesicht verzog sich. „Du würdest es nicht so lustig finden, wenn es \emph{deiner} Familie passiert wäre!“ 

„Ach tatsächlich“, schnurrte Draco. „Aber es \emph{würde} den Malfoys nie passieren.“ 

Rons Hände ballten sich zu Fäusten. 

„Das reicht“, sagte Harry und versuchte dabei so viel ruhige Autorität in seine Stimme einfließen zu lassen, wie er nur konnte. Wovon auch immer hier die Rede war, für den Rothaarigen war es offenbar eine schmerzhafte Erinnerung. „Wenn Ron nicht darüber sprechen will, dann braucht er nicht darüber sprechen – und dann bitte ich dich, ebenso wenig darüber zu sprechen.“ 

Draco sah Harry überrascht an, während Ron nickte. „Richtig so, Harry! Ich meine, Mr Bronze. Siehst du, was für ein Mensch der ist? Jetzt sag ihm, er soll abhauen!“ 

Harry zählte innerlich bis zehn – bei ihm war es ein sehr schnelles \emph{12345678910}, eine seltsame Angewohnheit aus dem Alter von fünf Jahren: Seine Mutter hatte ihn angewiesen, bis zehn zu zählen und Harry hatte argumentiert, dass seine Variante schneller ging und ebenso effektiv sein sollte. „Ron“, sagte Harry ruhig, „ich werde ihn nicht weg schicken. Er kann gerne mit mir sprechen, wenn er mag.“ 

„Na gut, aber ich gebe mich mit niemandem ab, der sich mit Draco Malfoy abgibt“, verkündete Ron kühl. 

Harry zuckte mit den Schultern. „Das steht dir frei. \emph{Ich} hingegen habe nicht vor, mir von irgendwem diktieren zu lassen, mit wem ich reden darf und mit wem nicht.“ Innerlich murmelte er \emph{geh einfach weg, geh einfach weg} … 

Rons Gesichtszüge entglitten ihm für einen Moment, als ob er tatsächlich erwartet hatte, dass dieser Satz bei Harry wirkte. Dann drehte Ron sich um, zerrte an seinem Koffer und stürmte ans andere Ende des Bahnsteigs. 

„Wenn du ihn nicht leiden kannst“, fragte Draco neugierig, „warum bist du nicht einfach weggegangen?“ 

„Naja … seine Mutter hat mir verraten, wie ich vom Bahnhof King's Cross zu diesem Gleis komme, also konnte ich ihn nicht einfach so verscheuchen. Und es ist ja nicht so, dass ich diesen Ron \emph{hasse}“, sagte Harry, „es ist nur, dass ich … dass …“ Harry suchte nach passenden Worten. 

„Dass du keinen Grund siehst, warum er existieren sollte?“, bot Draco an. 

„So ungefähr.“ 

„Auf jeden Fall, Potter … wenn du tatsächlich bei Muggeln aufgewachsen bist …“ Draco unterbrach sich einen Moment, als ob er einen Widerspruch erwartete, doch Harry sagte nichts. „… dann mag dir nicht ganz klar sein, was es bedeutet, berühmt zu sein. Die Leute wollen all deine Zeit in Anspruch nehmen. Du \emph{musst} lernen, nein zu sagen.“ 

Harry nickte langsam und setzte einen nachdenklichen Gesichtsausdruck auf. „Das klingt nach einem sehr guten Ratschlag.“ 

„Wenn du versuchst, nett zu ihnen zu sein, dann bedeutet das bloß, dass du die meiste Zeit mit den aufdringlichsten Leuten verbringst. Entscheide dich, mit wem du deine Zeit verbringen \emph{willst} und sage den anderen, sie sollen abhauen. Die Leute werden dich danach beurteilen, mit wem du dich umgibst und du willst nicht mit jemandem wie Ron Weasley gesehen werden.“ 

Harry nickte wieder. „Falls ich fragen darf – woran hast du mich erkannt?“ 

„\emph{Mister Bronze}“, schnarrte Draco, „ich habe dich schon einmal getroffen, denk dran. Ich habe dich \emph{recht gut kennengelernt}. Und dann sah ich jemanden, der mit einem Schal um den Kopf rumlief und absolut lächerlich aussah, also habe ich eine \emph{wilde Vermutung} geäußert.“ 

Harry neigte seinen Kopf um für das Kompliment zu danken. „Es tut mir \emph{schrecklich} Leid, was da passiert ist“, sagte er. „Bei unserem ersten Treffen, meine ich. Ich wollte dich nicht vor Lucius beschämen.“ 

Draco winkte ab und warf Harry einen seltsamen Blick zu. „Ich wünschte bloß, dass Vater reingekommen wäre, als \emph{du} um \emph{mich} herumscharwenzelt bist.“ Draco lachte. „Aber ich danke dir für das, was du zu Vater gesagt hast. Ohne das wäre es mir deutlich schwerer gefallen, es ihm zu erklären.“ 

Harry neigte den Kopf etwas tiefer. „Und ich danke \emph{dir}, dass du es bei Professor McGonagall erwidert hast.“ 

„Nichts zu danken. Allerdings muss eine der Gehilfinnen ihre engste Freundin zu absolutem Stillschweigen verpflichtet haben, denn Vater sagt, dass einige \emph{wilde Gerüchte} die Runde machen, laut derer wir zwei in einen Kampf oder so etwas verwickelt waren.“ 

„Autsch …“, sagte Harry zusammenzuckend. „Das tut mir wirklich Leid –“ 

„Nicht nötig, wir sind es gewohnt. Bei Merlin, es gibt schon so viele Gerüchte über die Familie Malfoy …“ 

Harry nickte. „Ich bin froh zu hören, dass du nicht in Schwierigkeiten geraten bist.“ 

Draco lächelte. „Vater hat, sagen wir, einen sehr \emph{differenzierten} Sinn für Humor, aber er versteht etwas vom Freundschaften schließen. Er versteht \emph{sehr viel} davon. Ich musste das sogar den letzten Monat über jede Nacht vor dem Einschlafen aufsagen: 'Ich werde auf Hogwarts Freundschaften schließen.' Als ich ihm dann alles erklärt habe und er verstand, dass es mir darum ging, hat er sich nicht nur entschuldigt, sondern mir sogar ein Eis spendiert.“ 

Harrys Unterkiefer klappte auf. „\emph{Du hast es geschafft,} das \emph{in ein Eis zu verwandeln?}“ 

Draco nickte und sah genau so zufrieden aus, wie diese Meisterleistung es erwarten ließ. „Naja, Vater \emph{erkannte} natürlich, dass ich das vorhatte, aber er war immerhin derjenige, der es mir beibrachte und wenn ich ihm richtig zuzwinkere, während ich es mache, dann ist das so eine Vater-Sohn-Angelegenheit und dann \emph{muss} er mir einfach ein Eis kaufen, sonst werfe ich ihm diesen traurigen Blick zu, als ob ich glaube, dass ich ihn enttäuscht habe.“ 

Harry musterte Draco prüfend und spürte die Anwesenheit eines anderen Meisters. „Er hat dir \emph{beigebracht}, wie man Leute manipuliert?“ 

„Schon solange ich mich erinnern kann“, sagte Draco stolz. „Vater hat mir Tutoren bezahlt.“ 

„Wow“, sagte Harry. Robert Cialdinis \emph{Einfluss: Theorie und Praxis} kam dagegen wohl nicht an (obwohl es ein wahnsinnig gutes Buch war). „Dein Vater ist fast so toll wie mein Vater.“ 

Dracos Augenbrauen schossen in die Höhe. „Tatsächlich? Und was macht \emph{dein} Vater?“ 

„Er kauft mir Bücher.“ 

Draco dachte einen Moment darüber nach. „Das hört sich nicht sehr beeindruckend an.“ 

„Du hättest es sehen müssen. Auf jeden Fall bin ich froh, das zu hören. So wie Lucius dich angeschaut hat, bekam ich Angst, dass er dich k–kreuzigen würde.“ 

„Mein Vater liebt mich sehr“, sagte Draco bestimmt. „Er würde sowas niemals tun.“ 

„Ähm“, sagte Harry und erinnerte sich an das weißhaarige Musterbeispiel für Perfektion, das im schwarzen Mantel und mit elegantem, todbringendem, silbernem Gehstock bei Madam Malkin hereingetreten war. Es war so schwer, sich diesen perfekten Mörder als liebevollen Vater vorzustellen. „Versteh das bitte nicht falsch – aber woher \emph{weißt} du das?“ 

„Hä?“ Es war klar, dass Draco sich diese Frage normalerweise nicht stellte. 

„Ich stelle dir die fundamentale Frage der Rationalität: Warum glaubst du das, was du glaubst? Was glaubst du zu wissen und woher glaubst du es zu wissen? Was hast du \emph{beobachtet}, was dich zu dem Glauben gebracht hat, dass Lucius dich nicht ebenso opfern würde wie eine beliebige andere Spielfigur?“ 

Draco warf Harry wieder einen seltsamen Blick zu. „Was genau weißt \emph{du} über Vater?“ 

„Ähm … Sitz im Zaubererrat, Sitz im Beirat der Schule, unglaublich reich, hat Gehör und Vertrauen von Minister Fudge, vermutlich auch einige äußerst peinliche Fotos von ihm, bekanntester Reinblut-Fanatiker seitdem der Dunkle Lord gefallen ist, besitzt das dunkle Mal und gehörte einst zum innersten Kreis der Todesser, ist jedoch davon gekommen, weil er behauptete, unter dem Imperius-Fluch gewesen zu sein, was jederman für geradezu lächerlich unwahrscheinlich hält … \emph{bösartig} – und zwar in Großbuchstaben, fett und unterstrichen – und ein geborener Killer … ich denke, das war alles.“ 

Dracos Augen hatten sich zu Strichen verzogen. „McGonagall hat dir das erzählt, nicht wahr?“ 

„Nein, sie hat sich geweigert, mir danach \emph{irgendetwas} über Lucius zu sagen, außer, dass ich mich von ihm fernhalten solle. Also habe ich mir während einem Vorfall im Tränkeladen, als Professor McGonagall damit beschäftigt war, mit dem Inhaber zu reden und die Situation wieder unter Kontrolle zu bekommen, einen Kunden geschnappt und \emph{den} über Lucius ausgefragt.“ 

Draco machte wieder große Augen. „Hast du das \emph{wirklich} gemacht?“ 

Harry warf Draco einen zweifelnden Blick zu. „Falls ich beim ersten Mal gelogen hätte, dann würde ich dir nicht die Wahrheit sagen, bloß weil du nochmal nachfragst.“ 

Einen Moment war es still, während Draco das verarbeitete. 

„Du wirst sowas von garantiert nach Slytherin kommen.“ 

„Ich werde garantiert nach Ravenclaw kommen, vielen Dank. Ich will nur deswegen Macht, damit ich Bücher bekomme.“ 

Draco kicherte. „Ja, sicher. Auf jeden Fall … um deine Frage zu beantworten …“ Draco atmete tief durch und wurde wieder ernst. „Vater hat einmal eine Abstimmung im Zaubererrat meinetwegen verpasst. Ich war von einem Besen gefallen und hatte mir viele Rippen gebrochen. Es tat sehr weh. Ich hatte noch nie solche Schmerzen gespürt und dachte, dass ich sterben würde. Also hat Vater diese wichtige Abstimmung verpasst, weil er im St. Mungo an meinem Bett saß, meine Hand hielt und mir immer wieder gesagt hat, dass alles gut wird.“ 

Harry blickte verschämt weg und hatte einige Mühe, seinen Blick wieder auf Draco zu zwingen. „Warum erzählst du mir das? Es ist doch recht … persönlich …“ 

Draco sah Harry ernst an. „Einer meiner Tutoren sagte einmal, dass Menschen enge Freundschaften schließen, indem sie persönliche Dinge über einander wissen; und dass die meisten Leute deswegen keine engen Freunde haben, weil es ihnen unangenehm ist, wirklich private Informationen mit anderen zu teilen.“ Draco sah ihn auffordernd an. „Jetzt bist du dran.“ 

Obwohl er wusste, dass Dracos erwartungsvoller Blick das Resultat monatelanger Übung war, bemerkte Harry, dass der Blick nicht an Wirkung verlor. Nun, genau genommen verlor er \emph{an Wirkung}, er verlor jedoch nicht \emph{seine Wirkung}. Genauso erging es ihm mit dem Erwiderungsdruck, den Dracos klug gemachtes Geschenk auf ihn ausübte. (In einem Psychologiebuch hatte er von einem ähnlichen Experiment gelesen, bei dem ein bedingungsloses Geschenk von $5 doppelt so wirksam war wie eine Belohnung von $50, wenn es darum ging, Menschen zum Ausfüllen eines Fragebogens zu bewegen.) Draco hatte Harry jetzt einen Vertrauensvorschuss geschenkt und Harry dann darum gebeten, ihm ebenfalls eine vertrauliche Information anzubieten … und Harry spürte den Druck. Wenn er sich jetzt weigerte, so war er überzeugt, würde er von Draco einen enttäuschten Blick ernten und womöglich etwas in seiner Achtung sinken. 

„Draco“, sagte Harry, „nur zur Information: Ich weiß genau, was du gerade versuchst. In meinen Büchern nennt man das \emph{Reziprozität}, also Erwiderung, und dort stehen Beispiele drin, wie etwa, dass man mit einem Geschenk von zwei Sickeln doppelt so gute Chancen hat, Leute zu etwas zu bewegen, wie mit einer Belohnung von zwanzig Sickeln …“ Harry verstummte. 

Draco sah traurig und enttäuscht aus. „Ich will dich nicht austricksen, Harry. Ich will mich wirklich mit dir anfreunden.“ 

Harry hob eine Hand. „Ich habe nicht gesagt, dass ich darauf nicht eingehen würde. Ich brauche nur einen Moment, um eine Information auszuwählen, die privat genug ist, aber niemandem schaden kann. Sagen wir es so … ich will dir zeigen, dass ich mich nicht zu etwas drängen lasse.“ Bereits eine kurze Denkpause konnte die Wirkung vieler Überzeugungstechniken eindämmen, wenn man sie einmal erkannt hatte. 

„In Ordnung“, sagte Draco. „Ich warte ab, bis dir etwas einfällt. Ach, und nimm doch bitte den Schal ab, während du es erzählst.“ 

\emph{Einfach, aber wirkungsvoll.} 

Harry konnte nicht umhin zu bemerken, wie offensichtlich, unbeholfen und tollpatschig seine Versuche, der Manipulation zu widerstehen / sein Gesicht zu wahren / anzugeben verglichen mit Dracos doch waren. \emph{Ich brauche unbedingt solche Tutoren.} 

„Also gut“, sagte Harry nach einer Weile. „Meine Information:“ Er sah um sich und schob den Schal dann hoch, so dass er nur noch die Narbe verdeckte. „Ähm … es klingt so, als ob du dich wirklich auf deinen Vater verlassen kannst. Ich meine … wenn du mit ihm über wichtige Dinge redest, dann hört er dir immer zu und nimmt dich ernst.“ 

Draco nickte. 

„Manchmal“, sagte Harry und schluckte. Es fiel ihm erstaunlich schwer, aber andererseits lag das in der Natur der Sache. „Manchmal wünschte ich, mein Papa wäre wie deiner.“ Harrys Augen wandten ihren Blick ab und Harry musste sich zwingen, Draco wieder anzusehen. 

Dann bemerkte Harry, \emph{was zum Teufel er da gerade gesagt hatte} und er fügte schnell hinzu: „Nicht, dass ich mir wünschen würde, dass mein Papa so ein makelloses Tötungsinstrument wäre – ich wünschte nur, er würde mich ernst nehmen …“ 

„Ich verstehe“, sagte Draco lächelnd. „So … fühlst du dich jetzt nicht auch schon etwas freundschaftlicher?“ 

Harry nickte. „Ja. Ehrlich gesagt schon. Ähm … nimm es mir bitte nicht übel, aber ich werde mich jetzt wieder maskieren, ich habe wirklich keine Lust, schon wieder –“ 

„Ich verstehe.“ 

Harry rollte den Schal wieder um seinen Kopf. 

„Mein Vater nimmt alle seine Verbündeten ernst“, sagte Draco. „Darum hat er so viele Verbündete. Vielleicht solltest du ihn mal kennen lernen.“ 

„Ich werde darüber nachdenken“, sagte Harry in einer wertungsfreien Stimme. Er schüttelte den Kopf. „Also bist du tatsächlich seine einzige Schwachstelle. Hm.“ 

Jetzt warf Draco Harry einen \emph{äußerst} irritierten Blick zu. „Wollen wir etwas zu trinken holen und uns dann irgendwo hinsetzen?“ 

Harry bemerkte erst jetzt, dass er zu lange auf der Stelle gestanden hatte und streckte sich. „Klar.“ 

Der Bahnsteig füllte sich allmählich, doch am hinteren Ende gab es noch einen ruhigen Bereich. Auf dem Weg dorthin kamen sie an einem Verkäufer vorbei, einem bärtigen Mann mit Glatze, der auf einem Karren Zeitungen, Comicbücher und neongrüne Getränkedosen anbot. 

Dieser hatte sich gerade an den Karren gelehnt und einen Schluck aus einer der neongrünen Dosen getrunken, als er sah, wie der elegant herausgeputzte Draco Malfoy zusammen mit einem mysteriösen Jungen, der einen Schal um den Kopf gewickelt hatte und unglaublich bescheuert aussah, näher kam. Der Verkäufer bekam plötzlich einen Hustenanfall und die grüne Flüssigkeit tropfte in seinen Bart. 

„Tschuldigung“, sagte Harry, „aber \emph{was genau} ist das?“ 

„Seltsaft“, sagte der Verkäufer. „Wenn du ihn trinkst, passiert irgendwas so Überraschendes, dass du dich oder jemanden in der Nähe vollprustest. Aber er ist verzaubert, so dass er nach einigen Sekunden wieder verschwindet.“ Und noch während er sprach, verschwanden die Flecken in seinem Bart. 

„Wie lächerlich“, sagte Draco. „Äußerst lächerlich. Komm, Mr Bronze, suchen wir uns einen anderen –“ 

„Warte mal“, sagte Harry. 

„\emph{Ach komm schon!} Das ist … das ist einfach \emph{kindisch}!“ 

„Nein, tut mir Leid, Draco, ich \emph{muss} das untersuchen. Was passiert, wenn ich Seltsaft trinke, während ich mir alle Mühe gebe, über ein vollkommen ernstes Thema zu sprechen?“ 

Der Verkäufer lächelte und zuckte mit den Schultern. „Wer weiß? Vielleicht siehst du einen Freund in einem Froschkostüm vorbeilaufen? \emph{Irgendwas} Witziges oder Unerwartetes wird auf jeden Fall passieren.“ 

„Nein. Tut mir Leid. Das glaube ich einfach nicht. Das verstößt auf so vielen Ebenen gegen meine Überzeugungen, dass ich es gar nicht in Worte fassen kann. Es \emph{kann einfach nicht sein}, dass so ein verdammtes \emph{Getränk} die Realität plötzlich in eine Sitcom verwandeln kann, sonst schmeiß ich alles hin und fliehe auf die Bahamas – oder sonst wohin.“ 

Draco stöhnte. „Muss das wirklich sein?“ 

„Du brauchst es nicht zu trinken, aber ich \emph{muss} es untersuchen. Ich \emph{muss} einfach. Wieviel kostet das?“ 

„Fünf Knut die Dose“, sagte der Verkäufer. 

„\emph{Fünf Knut?} Sie verkaufen realitätsverändernde Softdrinks für \emph{fünf Knut pro Dose}?“ Harry griff in seinen Beutel, sagte „vier Sickel und vier Knut“ und klatschte sie auf den Tresen. „Zwei dutzend Dosen bitte.“ 

„Ich nehm auch eine“, seufzte Draco und griff in seine Tasche. 

Harry schüttelte schnell den Kopf. „Nein, das geht auf mich. Sieh es nicht als Gefallen an, ich will überprüfen, ob es bei dir auch wirkt.“ Er warf Draco eine Dose zu und füllte dann seinen Beutel, dessen dehnbare Öffnung die Dosen mit einem leisen Rülpser verspeiste, welcher Harrys Hoffnung, dass er irgendwann eine vernünftige Erklärung für all das finden würde, nicht gerade bestärkte. 

Zweiundzwanzig Rülpser später nahm Harry die letzte Dose in die Hand. Draco sah ihn erwartungsvoll an und beide öffneten ihre Dosen gleichzeitig. 

Harry schob seinen Schal hoch, damit sein Mund frei war, und sie tranken den Seltsaft. Er \emph{schmeckte} sogar leuchtend grün – besonders sprudelnd und zitroniger als Zitronen. 

Nichts passierte. 

Harry sah den Verkäufer an, der ihnen wohlwollend zusah. 

\emph{Okay, falls dieser Kerl gerade einen bloßen Zufall ausgenutzt hat, um mir vierundzwanzig Dosen grüne Limo zu verkaufen, dann werde ich ihm für seinen Geschäftssinn applaudieren und ihn anschließend umbringen …} 

„Es geschieht nicht immer sofort“, sagte der Verkäufer. „Aber es geschieht garantiert einmal pro Dose, sonst kriegst du das Geld zurück.“ 

Harry nahm noch einen großen Schluck. 

Wieder passierte nichts. 

\emph{Vielleicht sollte ich den Rest so schnell wie möglich runterkippen … und hoffen, dass mein Bauch von der ganzen Kohlensäure nicht platzt, und dass mir nicht von selbst alles wieder hoch kommt …} 

Nein, ein \emph{wenig} Geduld konnte er schon haben. Aber ganz ehrlich, Harry konnte sich nicht vorstellen, dass dieses Getränk funktioniert. Man konnte nicht einfach auf jemanden zugehen und ihm sagen „Ich werde dich jetzt überraschen!“ oder „Ich werde dir jetzt die Pointe von dem Witz erzählen, und sie wird wirklich lustig!“ Das machte die gesamte Überraschung kaputt. So erwartungsvoll, wie Harry im Moment war, hätte Lucius Malfoy in Ballettkleidung vorbeitänzeln können und Harry hätte keine Miene verzogen. Was für einen verrückten Mumpitz sollte das Universum sich jetzt bloß ausdenken? 

„Naja, setzen wir uns erstmal“, sagte Harry. Er ging auf eine Bank zu und während er einen weiteren Schluck nahm, warf er noch einen kurzen Blick zurück – sein Blick streifte den Karren des Verkäufers und fiel auf eine Zeitung namens \emph{Der Klitterer}, die folgende Schlagzeile trug: 

\headline{JUNGE-DER-LEBT\\ SCHWÄNGERT DRACO MALFOY}

„\emph{Uargh!}“, schrie Draco, als Harry ihn mit einer knallgrünen Flüssigkeit vollprustete. Draco drehte sich wutentbrannt um. „Du Sohn eines Schlammbluts! Soll ich \emph{dich} mal vollspucken?“ Draco wollte das gerade in die Tat umsetzen, als auch sein Blick auf die Schlagzeile fiel. 

Reflexartig versuchte Harry sein Gesicht vor dem neongrünen Sprühnebel zu schützen. Leider riss er dabei genau die Hand hoch, in der er den Seltsaft hielt, so dass sich der Rest der grünen Flüssigkeit auf seine Schulter ergoss. 

Selbst während Harry noch keuchte und hustete, starrte er auf die leere Dose in seiner Hand und die grünen Flecken auf Dracos Umhang, die in dem Moment verschwanden. 

Dann sah er auf und las die Schlagzeile erneut: 

\headline{JUNGE-DER-LEBT\\ SCHWÄNGERT DRACO MALFOY}

Harrys Mund öffnete sich und sagte: „Ab– ab– … aber … ab– … aber … ab– bab…bab…bab…“ 

Er hatte einfach zu viele Einwände gleichzeitig. Jedes Mal, wenn er versuchte, „Aber wir sind erst elf!“ zu sagen, verlangte der Einwand „Aber Männer können nicht schwanger werden!“ höchste Priorität und wurde dann wiederum von „Aber zwischen uns läuft überhaupt nichts!“ überholt. 

Dann starrte Harry wieder auf die Dose in seiner Hand. 

Er spürte ein tiefes Verlangen, laut schreiend wegzurennen, bis er irgendwann wegen Sauerstoffmangels umgekippt wäre; und das einzige, was ihn davon abhielt, war, dass er einmal gelesen hatte, dass vollkommene Panik ein Anzeichen eines wirklich bedeutenden wissenschaftlichen Problems sei. 

Harry knurrte, warf die Dose wütend in einen nahen Mülleimer und schritt dann wieder zum Verkäufer rüber. „Eine Ausgabe vom \emph{Klitterer}, bitte.“ Er zahlte nochmal vier Knut, nahm eine weitere Dose Seltsaft aus seinem Beutel und ging dann wieder zurück zur Bank, wo Draco seine Limodose bewundernd ansah. 

„Ich nehm alles zurück“, sagte Draco. „Das war verdammt gut.“ 

„Draco, weißt du, ich könnte schwören, dass ein gemeinsamer Mord noch engere Freundschaften schließt als der Austausch von irgendwelchen Geheimnissen …“ 

„Ich habe einen Tutor, der das mal gesagt hat“, stimmte Draco zu. Er fasste unter seinen Umhang und kratzte sich gemächlich den Rücken. „An wen hattest du da gedacht?“ 

Harry schmiss den \emph{Klitterer} auf den Tisch. „Den Kerl, der sich diese Schlagzeile ausgedacht hat.“ 

Draco stöhnte. „Das ist kein Kerl. Das ist ein Mädchen. Ein \emph{zehn Jahre altes} Mädchen, kannst du dir das vorstellen? Sie ist verrückt geworden als ihre Mutter starb und ihr Vater, dem die Zeitung gehört, ist vollkommen überzeugt, dass sie eine Seherin ist. Wenn er also keine Ahnung hat, fragt er Luna Lovegood und glaubt ihr \emph{jedes Wort}.“ 

Ohne darüber nachzudenken öffnete Harry die nächste Dose Seltsaft und setzte zum Trinken an. „Machst du Witze? Das ist noch schlimmer als Muggeljournalismus und das habe ich für völlig unmöglich gehalten.“ 

Draco knurrte. „Außerdem hat sie eine perverse Vorliebe für die Malfoys und da ihr Vater unsere politischen Ansichten nicht teilt, druckt er jedes Wort davon. Sobald ich alt genug bin, werde ich sie vergewaltigen.“ 

Grüne Flüssigkeit spritzte aus Harrys Nasenlöchern und durchtränkte den Schal davor. Seltsaft und Lungen vertrugen sich nicht, so dass Harry einen heftigen Hustenanfall bekam. 

Draco sah ihn scharf an. „Stimmt was nicht?“ 

In dem Moment bemerkte Harry plötzlich (a), dass die Geräusche vom Rest des Bahnsteigs seit dem Zeitpunkt, wo Draco unter seinen Umhang gefasst hatte, nur noch ein dumpfes Murmeln waren und (b), dass in dem Moment, wo sie über Mord zur Freundschaftsvertiefung gesprochen hatten, nur einer der beiden Gesprächsteilnehmer von einem Witz ausgegangen war. 

\emph{Super. Er schien ja auch} so ein normales \emph{Kind zu sein. Und er} ist \emph{auch ein normales Kind; so normal, wie man es eben von einem Jungen erwarten würde, wenn er Darth Vader als fürsorglichen Vater hätte.} 

„Passt schon.“ Harry hustete. Wie sollte er bloß aus diesem Gespräch heil wieder rauskommen? „Ich war bloß überrascht, dass du bereit warst, so offen darüber zu sprechen; dass du gar keine Angst hattest, erwischt zu werden oder so etwas.“ 

Draco schnaubte. „Machst du Witze? \emph{Luna Lovegoods} Aussage gegen meine?“ 

\emph{Heilige Scheiße …} „Es gibt also keine magischen Lügendetektoren, nehme ich an?“ \emph{Oder DNA-Tests … zumindest noch nicht.} 

Draco sah sich um. Seine Augen verengten sich. „Ach stimmt, du weißt ja von nichts. Gut, ich erkläre dir das jetzt. Ich meine, wie es wirklich läuft; so als ob du schon in Slytherin wärst und mir dann die gleiche Frage stellst. Aber du musst schwören, dass du niemandem davon erzählst.“ 

„Ich darf aber über das Thema an sich sprechen, solange ich nicht erwähne, dass ich es von dir weiß, oder? Ich meine, angenommen ein anderer junger Slytherin stellt mir eines Tages dieselbe Frage …“ 

Draco überlegte. „Sag das nochmal.“ 

Harry tat es. 

„Okay, das klingt nicht so, als ob du mich irgendwie austricksen willst; also klar. Aber denk dran, dass ich jederzeit alles abstreiten kann. Schwöre es!“ 

„Ich schwöre“, sagte Harry. 

„Vor Gericht wird Veritaserum benutzt, aber das ist eigentlich ein schlechter Witz: Du musst bloß auf dich selbst einen Vergessenszauber sprechen bevor du aussagst und dann behaupten, dass der anderen Person per Gedächtniszauber eine falsche Erinnerung eingepflanzt wurde. Wenn du ein Denkarium hast, und wir haben eines, dann kannst du die Erinnerung danach sogar zurückholen. Naja, üblicherweise geht das Gericht eher von einem Vergessenszauber aus als von einem viel komplizierteren Gedächtniszauber. Aber da gibt es einen großen Entscheidungsspielraum. Und wenn \emph{ich} in sowas verwickelt bin, dann ist die Ehre eines Uralten Hauses betroffen und der Fall kommt vor den Zaubererrat, wo Vater genug Stimmen hat. Nachdem ich dort freigesprochen werde, müssten die Lovegoods mir Entschädigung zahlen weil sie meinen Ruf beschädigt haben. Und da sie von Anfang an wissen, dass es so laufen wird, werden sie gar nicht erst den Mund aufmachen.“ 

Eine Gänsehaut lief Harry über den Rücken. Eine Gänsehaut, die ihn eindringlich daran erinnerte, die Stimme und den Gesichtsausdruck so ruhig wie möglich zu belassen. \emph{Memo an mich selbst: Stürze die britische Zauberregierung bei der erstbesten Gelegenheit!} 

Harry räusperte sich laut. „Draco, versteh das jetzt bitte, bitte, \emph{bitte} nicht falsch, mein Schwur gilt natürlich, aber wie du sagtest, ich würde gut nach Slytherin passen und ich wollte das auch nur rein interessehalber fragen … also was würde \emph{rein theoretisch} passieren, falls ich aussagen \emph{würde}, dass ich deine Pläne mitgehört habe?“ 

„Nun, wenn ich kein Malfoy wäre, wäre ich in Schwierigkeiten“, antwortete Draco süffisant. „Aber da ich ein Malfoy bin … naja, Vater hat die Stimmen. Und anschließend würde er dich vernichten … vermutlich nicht ganz so leicht, da du der Junge-der-lebt bist, aber Vater ist ziemlich gut in solchen Sachen.“ Draco runzelte die Stirn. „Wo wir gerade dabei sind – \emph{du} hast doch vorgeschlagen sie umzubringen, warum hast du dir keine Sorgen gemacht, dass \emph{ich} aussage, wenn sie tot auftaucht? Ich bin zwar nicht ganz so berühmt wie du, aber deine, ähm, Anhänger werden wohl kaum allzu stark zu dir halten, wenn du irgendwas anstellst. Und Mord mit gefundener Leiche und all dem ist um einiges schlimmer als eine Vergewaltigung.“ 

Wenn das Gespräch nicht mehr vorwärts gehen kann, schlage einfach einen Haken. „Das ist bei Muggeln anders, in der Muggelwelt ist es ein riesiger Unterschied ob man mit Mord oder mit der Vergewaltigung eines kleinen Mädchens davonkommt.“ 

„Wirklich? Komisch. Wieso ist Mord nicht schlimmer? Also heißt das, wenn du sie vergewaltigst, wäre das für dich richtig klasse? Dann lasse ich dir gerne den Vortritt. Mensch, stell dir mal vor, wie Loony Lovegood behauptet, sie wäre von Draco Malfoy \emph{und} vom Jungen-der-lebt vergewaltigt worden … das würde ihr nicht mal \emph{Dumbledore} glauben.“ 

Zum Glück hatte Harry in dem Moment \emph{keinen} Seltsaft getrunken. \emph{Meine Güte, wann hat dieser Tag bloß angefangen, so verdammt schief zu laufen?} Harrys Gehirn überlegte und dachte sich den nächsten Haken aus. 

„Ehrlich gesagt wär's mir noch lieber, wenn du damit noch eine Weile abwartest. Nachdem ich mitgekriegt hab, dass diese Schlagzeile von einem Mädchen stammt, das ein Jahr jünger ist als ich, da dachte ich nicht mehr an Mord.“ 

„Ach? Woran sonst?“ Draco nahm einen weiteren Schluck Seltsaft. 

Harry wusste nicht, ob der Zauber mehr als einmal pro Dose wirken konnte, aber er wusste jetzt genau, wie er die Verantwortung von sich weisen konnte, daher gab er sich Mühe, genau den richtigen Zeitpunkt zu erwischen: 

„Ich habe mir gedacht, \emph{eines Tages werde ich dieses Mädchen heiraten}!“ 

Draco gab ein grässliches spritzendes Geräusch von sich und die grüne Flüssigkeit tropfte aus seinen Mundwinkeln wie aus einem undichten Tank. „\emph{Bist du wahnsinnig?}“ 

„Ja, bei wahnsinnig klarem Verstand.“ 

Draco kicherte; ein glockenhelles Geräusch. „Du hast einen noch verrückteren Geschmack als ein Lestrange. Aber du könntest sie trotzdem vergewaltigen, sie ist vermutlich verrückt genug, dass es ihr Spaß macht. Und falls nicht, könntest du immer noch einen Vergessenszauber benutzen und es in der nächsten Woche nochmal tun.“ 

\emph{Ich werde euer armseliges kleines magisches Überbleibsel aus dem Mittelalter so sehr in Stücke reißen, dass kein Molekül mehr auf dem anderen bleibt.} „Würdest du das bitte \emph{mir} überlassen? Wenn du sie tatsächlich vergewaltigen wolltest, dann kann ich dir einen Gefallen –“ 

Draco winkte ab. „Geschenkt.“ 

Harry starrte auf die Getränkedose in seiner Hand, während die eisige Kälte sich in seinem Körper breitmachte. Draco war charmant, gut gelaunt und Freunden gegenüber großzügig – er war kein Psychopath. Das war das traurige und zugleich schreckliche daran: Harry hatte genug über menschliche Psychologie gehört um zu wissen, dass Draco kein Monster war. In tausenden anderen Kulturen in der Menschheitsgeschichte hätte dieses Gespräch genau so stattfinden können. Nein, die Welt wäre ein vollkommen anderer Ort, wenn man ein bösartiger Freak sein müsste, um zu sagen, was Draco gesagt hatte. Es war sehr einfach, sehr menschlich; es war normal, solange niemand explizit gegenwirkte: Für Draco waren seine Feinde keine Menschen. 

Und in der zurückgebliebenen Zeit dieser zurückgebliebenen Gesellschaft – ebenso wie in der Dunkelheit vor dem Zeitalter der Erleuchtung – ging der Sohn eines ausreichend machtvollen Edelmanns schlicht und einfach davon aus, dass er über dem Gesetz stand. Zumindest, solange es um eine kleine Vergewaltigung hier oder da ging. 

Auch in der Muggelwelt gab es noch Länder, wo solche \emph{Edelmänner} existierten und genau so dachten; ganz zu schweigen von Ländern, wo breite Bevölkerungsschichten so dachten. Fast an jedem Ort und zu jeder Zeit, die nicht direkt vom Zeitalter der Erleuchtung abstammten, war es so. Eine Abstammungslinie, der das magische Großbritannien offenbar nicht angehörte, obwohl Dinge wie Getränkedosen es herüber geschafft hatten. 

\emph{Falls Draco von der Rache nicht ablässt und ich meine eigenen Lebenspläne nicht opfere um ein armes, verrücktes Mädchen zu heiraten, dann habe ich jetzt bloß etwas Zeit geschunden, und nicht mal all zu viel …} 
Für ein Mädchen. Nicht für andere. 

\emph{Ich frage mich, wie schwer es wohl wäre, eine Liste der bedeutenden Reinblut-Verfechter zu erstellen und sie alle umzubringen.} 

Genau das wurde in der Französischen Revolution versucht – erstelle eine Liste der \emph{Feinde gesellschaftlichen Fortschritts} und entferne alles oberhalb ihrer Schultern – und es hatte, soweit Harry sich erinnern konnte, nicht besonders gut funktioniert. Vielleicht sollte er in einigen der Geschichtsbücher, die sein Vater ihm gekauft hatte, nachschlagen, ob sich das, was damals falsch gelaufen war, auf einfache Weise beheben ließ. 

Harry sah zum Himmel auf und betrachtete die blasse Form des Mondes, die am wolkenlosen Morgenhimmel zu sehen war. 

\emph{Die Welt ist also kaputt und fehlerhaft und verrückt und grausam und blutbeschmiert und düster. Ist dir das neu? Das wusstest du doch alles schon längst …} 

„Du siehst so ernst aus“, sagte Draco. „Lass mich raten: Deine Muggeleltern haben dir erzählt, dass solche Sachen böse sind.“ 

Harry nickte, da er seiner Stimme nicht ganz vertrauten. 

„Naja, Vater sagt immer, es gibt zwar vier Häuser, aber letztendlich gehört jeder entweder nach Slytherin oder nach Hufflepuff. Und ehrlich gesagt bist du kein Hufflepuff. Wenn du dich insgeheim für die Malfoys entscheidest … unsere Macht und dein Ansehen … dann könntest du dir Sachen erlauben, die selbst \emph{ich} mich nicht trauen würde. Willst du es eine Zeit lang ausprobieren? Willst du wissen, wie sich das anfühlt?“ 

\emph{Was für eine schlaue kleine Schlange. Elf Jahre alt, aber du versuchst bereits, das Opfer aus dem Versteck zu locken. Ob es wohl schon zu spät ist, dich zu retten, Draco?} 

Harry überlegte, wog ab und wählte seine Waffe sorgfältig. „Draco, kannst du mir diese ganze Blutreinheits-Angelegenheit mal erklären? Das ist mir alles neu.“ 

Ein breites Lächeln lag auf Dracos Gesicht. „Du solltest wirklich Vater treffen und ihn fragen, weißt du, schließlich ist er unser Anführer.“ 

„Gib mir den \emph{Elevator Pitch}. Die Dreißig-Sekunden-Version, meine ich.“ 

„Okay“, sagte Draco. Er atmete tief durch und fing in einem etwas tieferen Tonfall an zu erklären. „Unsere Kräfte werden von Generation zu Generation durch den Einfluss der Schlammblüter schwächer. Während Salazar und Godric und Rowena und Helga einst Hogwarts aufgebaut und das Medaillon und das Schwert und das Diadem und den Becher und den Hut geschaffen haben, hat kein heutiger Zauberer Vergleichbares erreicht. Wir schwinden dahin, werden zu Muggeln werden, wenn wir uns mit ihrer Brut einlassen und unsere Squibs tolerieren. Wenn dieser Makel nicht ausgeräumt wird, dann werden unsere Zauberstäbe brechen und unsere Künste verfallen. Die Nachkommen Merlins werden entschwinden und das Blut von Atlantis versiegen. Unsere Kinder werden im Dreck kriechen, um wie bloße Muggel zu überleben, und Dunkelheit wird auf ewig über alle Länder fallen.“ Draco nahm einen Schluck aus der Dose und sah zufrieden aus. Soweit es ihn anging, waren das die wesentlichen Argumente. 

„Überzeugend“, sagte Harry, meinte jedoch die Beschreibung, nicht den Inhalt. Ganz klassische Muster: Der Fall in Ungnade; die Notwendigkeit, die verbliebene Reinheit gegen Verschmutzung zu schützen; die verklärte Vergangenheit und die unweigerlich ins Verderben führende Zukunft. Allerdings hatten diese Muster auch einen Gegensatz … „In einer Sache muss ich dir allerdings widersprechen. Dein Wissen über Muggel ist etwas veraltet. Wir kriechen längst nicht mehr im Dreck.“ 

Draco zuckte zusammen. „\emph{Was?} Wen meinst du mit \emph{wir}?“ 

„Wir. Die Wissenschaftler. Die Nachfolger von Francis Bacon, die Nachkommen des Zeitalters der Erleuchtung. Muggel haben nicht nur rumgesessen und geheult, dass sie keine Zauberstäbe haben; wir haben längst \emph{eigene} Kräfte, auch ohne Magie. Wenn all eure Kräfte schwinden, dann hätten wir etwas ungeheuer Wertvolles verloren, da die Zauberei der einzige Hinweis darauf ist, wie unser Universum \emph{tatsächlich} funktioniert – doch ihr würdet bei weitem nicht im Dreck kriechen. Eure Häuser wären weiterhin im Sommer kühl und im Winter warm, es gäbe weiterhin Ärzte und Medizin. Die Wissenschaft kann euch am Leben halten, selbst wenn eure Magie versagt. Es wäre eine Tragödie, und wir alle sollten versuchen, sie zu verhindern, doch es würde uns nicht in endgültige Dunkelheit stürzen. Das wollte ich nur gesagt haben.“ 

Draco war einige Schritte zurückgewichen und in seinem Gesicht mischten sich Angst und Unglaube. „\emph{Wovon bei Merlin sprichst du da, Potter?}“ 

„Hey, ich habe deiner Geschichte zugehört, willst du nicht auch meine anhören?“ \emph{Tölpelhaft}, tadelte Harry sich selbst, doch Draco wich nicht mehr zurück und schien ihm tatsächlich zuzuhören. 

„Was ich damit meine“, sagte Harry, „ist, dass du vermutlich nicht genau beachtest, was in der Muggelwelt vor sich geht.“ Vermutlich, weil die gesamte Zaubererwelt den Rest der Erde für eine Art Slum hält und ebensowenig beachtet wie die \emph{Financial Times} das alltägliche Elend in Burundi. „Na gut, kurzer Test: Waren Zauberer schon mal auf dem Mond? Du weißt schon, dort oben?“ Harry zeigte auf die riesige, weit entfernte Kugel am Himmel. 

„\emph{Was?}“, sagte Draco. Es war offensichtlich, dass der Gedanke ihm noch nie gekommen war. „Auf den – das ist doch –“ Sein Finger deutete auf die kleine blasse Sichel am Himmel. „Man kann nur an Orte apparieren wo man schonmal war – aber wie sollte man dort bitte jemals hinkommen?“ 

„Einen Moment“, sagte Harry zu Draco, „ich würde dir gerne ein Buch zeigen, das ich dabei habe. Ich glaube, ich erinnere mich, wo ich es hingetan habe.“ Harry kniete sich nieder, klappte den Zugang zum untersten Geschoss seines Koffers aus, eilte dann die Stufen herab und räumte einige Bücherkisten so schnell aus dem Weg wie es ihm möglich war, ohne die Bücher respektlos zu behandeln. Von einem Stapel hob er mehrere Kisten runter, legte den Deckel der darunterliegenden Kiste beiseite und überflog die Bücherrücken. 

(Harry hatte die nahezu magische Verres-sche Fähigkeit geerbt, sich sofort zu erinnern wo ein Buch war, auch wenn er es nur ein einziges Mal gesehen hatte – eine seltsame Fähigkeit, wenn man bedachte, dass gar keine genetische Verbindung bestand.) 

Harry eilte die Stufen wieder hoch, schloss den Eingang mit einem Stoß seiner Ferse und blätterte keuchend in dem Buch, bis er das Bild gefunden hatte, das er Draco zeigen wollte. 

Das Bild mit dem weißen, vertrockneten, kraterübersähten Land, den Menschen im Raumanzug und der blau-weißen Kugel, die über allem thronte. 

Das Bild. 

\emph{Das} Bild – wenn von allen Bildern auf der Erde nur ein einziges überdauern würde … 

„\emph{So}“, sagte Harry mit vor Stolz zitternder Stimme, „sieht die Erde aus, wenn man auf dem Mond steht.“ 

Draco beugte sich langsam rüber. Ein seltsamer Ausdruck lag auf seinem jungen Gesicht. „Wenn das ein \emph{echtes} Foto ist, warum bewegt es sich dann nicht?“ 

\emph{Bewegt?} Oh. „Muggel können auch bewegte Bilder aufnehmen, aber sie brauchen eine größere Kiste dafür, sie können die noch nicht auf eine Buchseite packen.“ 

Dracos Finger zeigte auf die Raumanzüge. „Was ist das?“ Seine Stimme begann zu flattern. 

„Das sind Menschen. Sie tragen Anzüge, die ihren ganzen Körper einschließen, um Luft zu bekommen, weil es auf dem Mond keine Luft gibt.“ 

„Das ist unmöglich“, flüsterte Draco. Tiefe Furcht war in seinen Augen zu sehen, gemischt mit vollkommener Verwirrung. „Kein Muggel könnte das jemals tun. \emph{Wie …}“ 

Harry nahm das Buch wieder und blätterte kurz, bis er fand, was er suchte. „Das ist eine Rakete, die gerade gestartet ist. Das Feuer schiebt sie höher und höher, bis sie zum Mond kommt.“ Er blätterte weiter. „Das ist eine Rakete auf dem Boden. Der winzige Fleck neben ihr ist ein Mensch.“ Draco keuchte. „Auf den Mond zu gelangen kostete ungefähr so viel wie … schätzungsweise tausend Millionen Galleonen.“ Draco würgte. „Und mitgewirkt haben … vermutlich mehr Menschen als in der gesamten britischen Zaubererwelt leben.“ \emph{Und als sie ankamen, haben sie eine Plakette hinterlassen, auf der stand: 'Wir kamen in Frieden, für die Menschheit.' Du bist noch nicht bereit, diese Worte zu hören, Draco, aber ich hoffe, dass es eines Tages soweit sein wird …} 

„Du sagst die Wahrheit“, sagte Draco zögernd. „Du würdest dafür nicht ein gesamtes Buch fälschen – und ich kann es in deiner Stimme hören. Aber … aber …“ 

„Aber wie, ohne Zauberstab und Magie? Es ist eine lange Geschichte, Draco. Wissenschaft funktioniert nicht, indem du den Zauberstab schwingst und Zaubersprüche aufsagst; sie funktioniert, indem du so genau weißt, wie das Universum funktioniert, dass du genau weißt, was du tun musst, damit das Universum tut, was du von ihm willst. Wenn Zauberei so ist, als ob du einen \emph{Imperius} auf eine Person sprichst, damit sie macht, was du willst, dann ist Wissenschaft so, als ob du die Person so genau kennst, dass du weißt, was du sagen musst, damit diese Person das Gefühl hat, sich selbst dazu entschlossen zu haben. Es ist sehr viel schwieriger als den Zauberstab zu schwingen, aber es funktioniert auch dann, wenn Zauberstäbe scheitern – genau wie du, wenn ein \emph{Imperius} scheitert, immer noch versuchen kannst, die Person zu überzeugen. Und Wissenschaft baut von Generation zu Generation immer aufeinander auf. Um Wissenschaft zu betreiben, musst du wirklich genau \emph{wissen}, was du tust – und wenn du etwas wirklich verstanden hast, dann kannst du es jemand anderem erklären. Die Mächte der größten Wissenschaftler vergangener Jahrhunderte, deren Namen man noch heute mit Bewunderung ausspricht, sind \emph{nichts}, verglichen mit der Macht der größten Wissenschaftler unserer Tage. So etwas wie die verlorenen Künste, die Hogwarts erschaffen haben, gibt es in der Wissenschaft nicht. Unsere Mächte wachsen von Jahr zu Jahr. Wir haben begonnen, die Geheimnisse von Leben und Vererbung zu erkunden und aufzuschlüsseln. Wir werden uns das Blut, von dem du sprachst, genau ansehen können, und schauen, was einen Zauberer ausmacht. Ein oder zwei Generationen später werden wir es überzeugen, all eure Kinder zu mächtigen Zauberern zu machen. Du siehst, dein Problem ist längst nicht so schlimm wie es aussieht, denn in wenigen Jahrzehnten werden die Wissenschaften es gelöst haben.“ 

„Aber …“, sagte Draco. Seine Stimme bebte. „Wenn \emph{Muggel} diese Mächte haben … was … was haben \emph{wir} dann noch?“ 

„Nein, Draco, das ist eine ganz falsche Frage, verstehst du nicht? Die Wissenschaften nutzen die Kraft des menschlichen Geistes, um die Welt anzusehen und sie zu erforschen. Sie können nicht vergehen, ohne dass die Menschheit selbst vergeht. Deine Magie könnte verschwinden, und so unerträglich es auch für dich wäre, du wärst immer noch \emph{du}. Du würdest dennoch leben und ihr nachtrauern. Doch weil die Wissenschaften auf meinem menschlichen Verstand aufbauen, kann ich ihre Macht nicht verlieren, ohne \emph{mich selbst} zu verlieren. Selbst wenn die Naturgesetze sich ändern und all mein bisheriges Wissen unnütz würde, werde ich die neuen Gesetze herausfinden, wie es schon einmal gelungen ist. Das ist keine Muggel-Sache – es ist \emph{menschlich}. Es baut auf all dem auf, was du jedes Mal benutzt, wenn du etwas nicht verstehst und dich fragst: 'Warum?' Du bist ein Slytherin, Draco; erkennst du nicht, was daraus folgt?“ 

Draco sah vom Buch auf. Das Verständnis erhellte plötzlich sein Gesicht: „Auch Zauberer können diese Macht nutzen.“ 

\emph{Der Köder ist ausgelegt. Obacht, jetzt … abwarten, bis er anbeißt …} „Wenn du lernst, dich selbst als \emph{Mensch} zu sehen statt als \emph{Zauberer}, dann kannst du diese menschlichen Fähigkeiten trainieren und verbessern.“ 

Dass dieser Satz üblicherweise nicht Teil des Lehrplans war, musste man Draco ja nicht erzählen, oder? 

Draco sah sehr nachdenklich aus. „Du … hast das bereits hinter dir?“ 

„Teilweise“, gab Harry zu. „Ich bin noch nicht fertig ausgebildet. Nicht mit elf Jahren. Aber – auch \emph{mein} Vater hat Tutoren bezahlt.“ Klar, es waren arme Studenten gewesen, und es hatte hauptsächlich an Harrys 26-stündigem Schlafzyklus gelegen — aber das lassen wir jetzt erstmal beiseite … 

Langsam nickte Draco. „Du glaubst also, du kannst \emph{beide} Künste erlernen, die Mächte kombinieren und …“ Draco starrte Harry an. „Herrscher beider Welten werden?“ 

Harry gab ein böses Gelächter von sich, weil es an dieser Stelle einfach so gut passte. „Du musst dir klarmachen, Draco, dass die ganze Welt, die du kennst – das gesamte magische Großbritannien – nur ein Feld auf einem viel größeren Spielbrett ist. Zum Spielbrett gehören auch der Mond und die Sterne am Nachthimmel, die ebenso leuchten wie unsere Sonne, aber unvorstellbar viel weiter entfernt sind, und Dinge wie Galaxien, die unermesslich größer sind als die Erde und die Sonne; derart große Dinge, dass nur Wissenschaftler sie sehen können und du nicht einmal von ihrer Existenz weißt. Aber ich bin wirklich ein Ravenclaw, weißt du, kein Slytherin. Ich will das Universum nicht beherrschen. Ich denke bloß, dass es etwas besser gestaltet werden könnte.“ 

Draco sah ihn erfürchtig an. „Warum erzählst du \emph{mir} davon?“

„Nun … es gibt nicht viele Menschen, die \emph{wahre} Wissenschaft betreiben können – etwas dazulernen, selbst wenn es dich höllisch verwirrt. Etwas Hilfe könnte ich gut gebrauchen.“ 

Draco starrte Harry mit offenem Mund an. 

„Aber versteh das nicht falsch, Draco. Wahre Wissenschaft ist mit Zauberei nicht vergleichbar, du kannst sie nicht betreiben und dann wieder damit aufhören, als ob du einen neuen Zauberspruch gelernt hast. Diese Macht hat ihren Preis – einen Preis, der so hoch ist, dass die meisten Leute ihn nicht zahlen wollen.“ 

Draco nickte beifällig, als ob er endlich etwas hörte, das er verstehen konnte. „Welcher Preis ist das?“ 

„Du musst lernen, Fehler einzugestehen.“ 

„Ähm“, sagte Draco nach einer dramatischen Pause. „Kannst du das näher erklären?“ 

„Wenn du versuchst herauszufinden, wie etwas funktioniert, dann sind die ersten neunundneunzig Erklärungen, die dir einfallen, höchstwahrscheinlich falsch. Erst die hundertste ist richtig. Du musst also wieder und wieder lernen, Fehler einzugestehen. Das hört sich leicht an, doch es ist so schwer, dass die meisten Leute es nicht können. Hinterfrage dich immer wieder aufs Neue, hinterfrage die Dinge, die du als selbstverständlich ansiehst“ – zum Beispiel, dass es im Quidditch einen Schnatz gibt – „und jedes Mal, wenn du deine Ansichten änderst, veränderst du dich selbst. Doch ich greife schon weit voraus. Ich greife viel zu weit voraus. Ich möchte dir nur sagen … ich biete dir an, mein Wissen zu teilen. Wenn du willst. Es gibt nur eine Bedingung.“ 

„Ah ja“, sagte Draco. „Vater sagt immer, dass es niemals ein gutes Zeichen ist, wenn jemand das zu dir sagt.“ 

Harry nickte. „Nun, versteh mich nicht falsch, ich will keinen Keil zwischen dich und deinen Vater treiben. Darum geht es mir nicht. Ich will es einfach mit jemandem in meinem Alter zu tun haben, nicht mit Lucius. Dein Vater wäre damit einverstanden, nehme ich an, er weiß auch, dass du irgendwann erwachsen werden musst. Doch deine Züge in unserem Spiel müssen deine eigenen sein. Das ist meine Bedingung, Draco – dass ich es mit dir zu tun habe, nicht mit deinem Vater.“ 

„Es reicht“, sagte Draco. Er stand auf. „Das war zuviel. Ich muss darüber nachdenken. Ganz zu schweigen davon, dass wir bald in den Zug einsteigen sollten.“ 

„Lass dir Zeit“, sagte Harry. „Aber denk daran, dass es kein Exklusivangebot ist, selbst wenn du es annimmst. In der Wissenschaft kommt man alleine manchmal nicht weit.“ 

Die Geräusche vom Bahnsteig wurden wieder deutlicher, als Draco davonging. 

Harry sah auf seine Armbanduhr, eine sehr einfache, mechanische Uhr, die sein Vater ihm in der Hoffnung gekauft hatte, dass sie selbst inmitten von Magie noch funktionieren würde. Sie tickte immer noch und falls sie richtig ging, dann war es gerade kurz vor elf Uhr. Es war wohl Zeit einzusteigen und sich auf die Suche nach wie-hieß-sie-doch-gleich zu machen, doch es erschien ihm sinnvoll, erstmal einige Atemübungen zu machen und abzuwarten, ob ihm dadurch nach diesem Gespräch wieder wärmer wurde. 

Doch als Harry von seiner Uhr aufsah, kamen zwei Personen auf ihn zu, die Schals um ihre Gesichter gewickelt hatten und absolut lächerlich aussahen. 

„Hallo, Mr Bronze“, sagte eine der maskierten Figuren, „hätten Sie wohl Interesse, sich dem Orden des Chaos anzuschließen?“ 


\later 

\emph{Nachspiel:} 
Etwas später, nachdem die Hektik des Tages endlich abgeflaut war, beugte sich Draco mit einer Feder in der Hand über den Schreibtisch. Er hatte in den Slytherin-Kerkern ein eigenes Zimmer, mit einem eigenen Schreibtisch und einem eigenen Kamin – zwar hatte selbst \emph{er} keinen Anschluss ans Flohnetzwerk, aber immerhin vertrat man in Slytherin nicht die \emph{blödsinnige} Idee, alle in einem Schlafsaal unterzubringen. Es gab nicht viele Einzelzimmer, man musste zu den Allerbesten im Haus gehören, doch bei einem Malfoy wurde das vorausgesetzt. 

\emph{Lieber Vater,} schrieb Draco. 

Dann setzte er ab. 

Tinte tropfte langsam von seiner Feder und hinterließ Flecken auf dem Pergament. 

Draco war nicht dumm. Er war jung, doch seine Tutoren hatten ihn darauf trainiert, bestimmte Verhaltensmuster automatisch zu erkennen. Draco wusste, dass Potter sich vermutlich Dumbledores Seite sehr viel verbundener fühlte, als es den Anschein hatte … doch Draco dachte, dass Potter in Versuchung geführt werden konnte. Allerdings war es kristallklar, dass Potter ebenso versuchte, Draco in Versuchung zu führen. 

Es war außerdem klar, dass Potter brilliant und sehr viel mehr als nur ein bisschen verrückt war – und, dass er ein großangelegtes Spiel spielte, dass er selbst kaum verstand; stattdessen improvisierte er bei rasender Geschwindigkeit mit der Subtilität eines randalierenden Nundu. Doch Potter hatte eine Taktik gewählt, der Draco sich nicht so einfach entziehen konnte: Er hatte Draco einen Teil seiner Macht angeboten, überzeugt davon, dass Draco diese Macht nicht benutzen konnte, ohne ihm ein Stück ähnlicher zu werden. Sein Vater hatte Draco gewarnt, dass diese äußerst fortgeschrittene Technik oft nicht funktionierte. 

Draco wusste, dass er nicht alles verstanden hatte, was passiert war … doch Potter hatte \emph{ihm} die Chance gegeben mitzuspielen, also war es jetzt \emph{seine} Chance. Falls er es jedoch rausposaunte, würde es zu einer Angelegenheit seines Vaters werden. 

Letzlich war es so einfach. Die simplen Techniken setzten voraus, dass die Zielperson sie nicht erkannte oder sich zumindest nicht sicher war. Schmeichelei musste den Anschein der Bewunderung erwecken. („Du wirst garantiert nach Slytherin kommen“ war ein uralter Klassiker; sehr wirkungsvoll auf unvorbereitete Personen und, bei Bedarf, unauffällig wiederholbar.) Doch wenn man den ultimativen Angriffspunkt bei einer Person herausfindet, dann macht es nichts aus, wenn es der Person bewusst ist. Potter hatte in seiner wahnwitzigen Eile einen Schlüssel zu Dracos Innerstem gefunden. Und dass Draco sich dessen bewusst war – schließlich war es ein offensichtlicher Angriffspunkt – änderte daran nicht das geringste. 

Zum ersten Mal in seinem Leben hatte er also wirkliche Geheimnisse. Er spielte sein eigenes Spiel. Es trug einen seltsamen Schmerz mit sich, doch da er wusste, dass Vater stolz sein würde, machte ihm das nichts aus. 

Draco ließ die Tintenflecken stehen – sie hatten eine gewisse Aussage, die sein Vater verstehen würde, da sie mehr als einmal solche Subtilitäten ausgetauscht hatten – und schrieb die Frage nieder, die schon die ganze Zeit an ihm nagte – das Detail, welches er eigentlich verstehen sollte, jedoch absolut nicht tat. 

\emph{Lieber Vater,}

\emph{angenommen, ich erzählte, dass ich auf Hogwarts einen Schüler kennengelernt habe, der noch nicht Mitglied unseres Bekanntenkreises ist, Dich ein 'makelloses Tötungsinstrument' nannte und ergänzte, dass ich Deine 'einzige Schwachstelle' sei. Was würdest Du von ihm halten?}

Es dauerte nicht lange, bis eine Eule Draco die Antwort brachte. 

\emph{Mein geliebter Sohn:}

\emph{ich würde sagen, dass du so das Glück hattest, jemanden kennenzulernen, der das vollkommene Vertrauen unseres Freundes und wertvollen Verbündeten Severus Snape genießt.}

Draco blickte eine Weile auf den Brief und warf ihn schließlich ins Feuer. 
