\chapter{Rationalisierung}

\lettrine{H}{ermine} Granger hatte Angst davor gehabt, böse zu werden. 

Der Unterschied zwischen gut und böse war normalerweise leicht zu erkennen; sie hatte nie verstanden, warum es anderen Leuten so schwer fiel. Auf Hogwarts waren Professor Flitwick und Professor McGonagall und Professor Sprout gut. Professor Snape und Professor Quirrell und Draco Malfoy waren böse. Harry Potter … war eine der wenigen Personen, denen man es \emph{nicht} einfach so ansehen konnte. Sie versuchte immer noch herauszufinden, wo er dazu gehörte. 

Aber was \emph{sie selbst} anging … 

Es machte Hermine \emph{zu viel Spaß}, Harry Potter fertigzumachen. 

Sie war in jedem einzelnen Unterrichtsfach besser als er. (Außer im Fliegen, aber das war wie Sportunterricht, das zählte nicht.) Sie hatte an fast jedem Tag der ersten Woche \emph{richtige} Hauspunkte bekommen. Nicht für komische heldenhafte Dinge, sondern für \emph{kluge} Dinge, wie Zaubersprüche schnell lernen und anderen Schülern helfen. Sie wusste, dass diese Hauspunkte besser waren – und das Beste daran war, dass Harry Potter es auch wusste. Sie konnte es jedes Mal, wenn sie wieder einen \emph{richtigen} Hauspunkt bekam, in seinen Augen sehen. 

Wenn man zu den Guten gehörte, dann sollte man seine Siege nicht so sehr auskosten. 

Es hatte schon am Tag der Zugfahrt begonnen, obwohl es eine Weile gedauert hatte, bis all diese Aufregung sich gelegt hatte. Erst spät am Abend war Hermine allmählich klar geworden, \emph{wie sehr} dieser Junge sie überrumpelt hatte. 

Bevor sie Harry Potter getroffen hatte, gab es niemanden, den sie fertigmachen wollte. Wenn jemand in der Schule nicht so gut wie sie war, dann war es ihre Aufgabe gewesen, zu helfen; nicht, noch darauf herumzureiten. So machte man das, wenn man zu den Guten gehörte. 

Und jetzt … 

… jetzt \emph{gewann} sie, Harry Potter zuckte jedes Mal zusammen, wenn sie wieder einen Hauspunkt bekam, und sie hatte \emph{so viel} Spaß dran; ihre Eltern hatten sie vor Drogen gewarnt und sie vermutete, dass sie hieran \emph{mehr} Spaß hatte als an Drogen. 

Sie hatte sich immer gefreut, wenn Lehrer ihr ein Lächeln geschenkt hatten, weil sie etwas richtig gemacht hatte. Sie hatte sich immer gefreut, wenn sie in einem Test volle Punktzahl bekommen und die vielen Häkchen auf dem Blatt gesehen hatte. Doch wenn sie jetzt im Unterricht gut war, drehte sie sich beiläufig um, sah wie Harry Potter mit den Zähnen knirschte und wollte sogleich ein Lied anstimmen, wie in einem Disney-Film. 

Das war böse, oder? 

Hermine Granger hatte Angst davor gehabt, böse zu werden. 

Und dann war ihr ein Gedanke gekommen, der alle ihre Sorgen beiseite gewischt hatte. 

Sie und Harry waren verliebt! Natürlich! Jeder wusste, was es bedeutet, wenn ein Junge und ein Mädchen anfingen, sich ständig zu streiten. Sie \emph{warben} umeinander! \emph{Daran} war nichts Böses. 

Es konnte nicht sein, dass sie ihre \emph{Freude} daran hatte, den berühmtesten Schüler der Schule in jedem Fach vernichtend zu schlagen; den Jungen, \emph{über den} Bücher geschrieben wurden, und der wie ein Buch \emph{redete}; den Jungen, der irgendwie den Dunklen Lord besiegt hatte und der sogar \emph{Professor Snape} zerquetscht hatte wie ein jämmerliches, kleines Insekt; den Jungen, der, wie Professor Quirrell es formuliert hätte, jedem anderen Ravenclaw aus seinem Jahr überlegen war, \emph{außer} Hermine Granger, die den Jungen-der-lebt in allen Fächern außer den Flugstunden übertrumpfte. 

Denn das wäre böse. 

Nein. Es war Liebe. \emph{Das} war es. \emph{Deswegen} stritten sie sich. 

Hermine war froh, dass sie das rechtzeitig verstanden hatte, denn heute würde Harry ihren Lesewettstreit verlieren, von dem \emph{die ganze Schule} wusste, und vor lauter überquellender Freude wollte sie anfangen zu \emph{tanzen}. 

Es war Samstag, 14:45~Uhr, und Harry Potter hatte noch die Hälfte von Bathilda Bagshots \emph{Geschichte der Zauberei} zu lesen und sie fixierte ihre Taschenuhr, die fürchterlich langsam die letzten zwei Minuten abzählte. 

Und alle Ravenclaws im Gemeinschaftsraum sahen zu. 

Es waren nicht nur die Erstklässler; die Neuigkeit hatte sich schnell rumgesprochen und halb Ravenclaw war im Raum versammelt, hatte sich auf den Sofas zusammengedrängt oder lehnte an Bücherschränken oder saß auf Sessellehnen. Alle sechs Vertrauensschüler waren da, auch die Schulsprecherin. Jemand musste einen Lufterfrischungszauber sprechen, damit genug Sauerstoff anwesend war. Und der Lärm der Gespräche war zu einem Geflüster geworden, das nun vollkommener Stille gewichen war. 

14:46~Uhr. 

Die Spannung war mit Händen zu greifen. Wenn es jemand anders gewesen wäre, \emph{irgendjemand} anders, dann wäre seine Niederlage eine ausgemachte Sache gewesen. 

Doch es war Harry Potter und es war nicht auszuschließen, dass er irgendwann in den nächsten Sekunden eine Hand heben und mit den Fingern schnippen würde. 

Mit einem plötzlichem Schrecken wurde ihr klar, dass Harry Potter womöglich genau das tun würde. Es wäre \emph{so typisch für ihn}, wenn er die zweite Hälfte des Buches schon vorher \emph{fertig gelesen hätte} … 

Hermines Blick verschwamm. Sie versuchte tief einzuatmen und stellte fest, dass es ihr nicht gelang. 

Noch zehn Sekunden, und er hatte immer noch nicht die Hand gehoben. 

Noch fünf Sekunden. 

14:47~Uhr. 

Harry Potter platzierte sorgfältig ein Lesezeichen in seinem Buch, schloss es und legte es beiseite. 

„Ich möchte zu Protokoll geben“, sagte der Junge-der-lebt mit deutlicher Stimme, „dass mir nur ein halbes Buch fehlte und dass ich eine Reihe unerwarteter Verzögerungen erlitten habe –“ 

\emph{„Du hast verloren!“}, kreischte Hermine. „Du \emph{hast es getan}! Du \emph{hast unseren Wettstreit verloren}!“ 

Ein kollektives Ausatmen ertönte, als alle wieder anfingen zu atmen. 

Harry Potter warf ihr einen empörten Blick zu, doch sie schwebte in einer Sphäre aus reiner weißer Freude und ließ sich davon nicht stören. 

\emph{„Hast du irgendeine Ahnung, was für eine Woche das für mich war?“}, sagte Harry Potter. „Jedes andere Wesen hätte es kaum geschafft, acht Kinderbücher zu lesen!“ 

„\emph{Du} hast das Zeitlimit gesetzt.“ 

Harrys Blick wurde noch empörter. „Ich konnte absolut nicht wissen, dass ich die ganze Schule vor Professor Snape retten oder im Verteidigungsunterricht verprügelt werden würde. Und wenn ich dir erzählen würde, was ich am Donnerstag von 17~Uhr bis zum Abendessen gemacht habe, würdest du mich für verrückt halten –“ 

„Ohhh, mir scheint, dass \emph{jemand} dem \emph{Planungstrugschluss} zum Opfer fiel.“ 

Auf Harry Potters Gesicht zeichnete sich purer Schock ab. 

„Oh, da fällt mir ein, ich habe den ersten Stapel Bücher gelesen, den du mir geliehen hast“, sagte Hermine mit dem unschuldigsten Gesichtsausdruck, den sie aufsetzen konnte. Es waren auch einige \emph{komplizierte} Bücher dabei gewesen. Sie fragte sich, wie lange \emph{er} wohl gebraucht hatte, um sie durchzulesen. 

„Eines Tages“, sagte der Junge, der lebt, „wenn die entfernten Verwandten von \emph{Homo sapiens} auf die Geschichte der Galaxie zurückblicken und sich fragen, warum alles so schief gelaufen ist, werden sie feststellen, dass es der allererste Fehler war, dass jemand Hermine Granger das Lesen beigebracht hat.“ 

„Aber du hast trotzdem verloren“, sagte Hermine. Sie strich mit der Hand über ihr Kinn und sah nachdenklich aus. „Ich frage mich, was wohl dein Wetteinsatz sein soll?“ 

\emph{„Was?“} 

„Du hast die Wette verloren“, erklärte Hermine, „also musst du zur Strafe etwas tun.“ 

„Ich kann mich nicht erinnern, dass wir so etwas ausgemacht haben!“ 

„Tatsächlich?“, sagte Hermine Granger. Sie setzte einen nachdenklichen Gesichtsausdruck auf. Dann, als ob die Idee ihr gerade erst gekommen sei: „Dann lassen wir darüber abstimmen. Alle Ravenclaws, die finden, dass Harry einen Wetteinsatz schuldig ist, meldet euch!“ 

\emph{„Was?“}, kreichte Harry Potter wieder. 

Er blickte sich um und sah, dass er sich inmitten eines Meeres aus erhobenen Händen befand. 

Und wenn Harry Potter \emph{genau hingeschaut} hätte, wäre ihm aufgefallen, dass schrecklich viele Schaulustige Mädchen waren, und dass fast jede weibliche Person im Raum die Hand gehoben hatte. 

„Stop!“, jammerte Harry Potter. „Ihr wisst nicht, was sie von mir verlangen will! \emph{Versteht} ihr denn nicht, was sie tut? Sie will, dass ihr euch jetzt schon festlegt, damit ihr euch später nicht mehr umentscheiden könnt und allem zustimmt, was sie verlangt!“ 

„Keine Sorge“, sagte die Vertrauensschülerin Penelope Clearwater. „Wenn sie etwas Übertriebenes verlangt, können wir uns einfach umentscheiden. Habe ich Recht?“ 

Und all jene Mädchen, denen Penelope Clearwater von Hermines Plan erzählt hatte, nickten eifrig. 

\later 

Eine stumme Gestalt schlich sich durch die kühlen Gänge in den Kerkern von Hogwarts. Er sollte um 18~Uhr in einem ganz bestimmten Raum sein, um eine ganz bestimmte Person zu treffen, und falls irgend möglich sollte er früher da sein, um Respekt zu zeigen. 

Doch als seine Hand den Türknauf drehte und die Tür zu dem dunklen, stillen, unbenutzten Klassenzimmer aufstieß, stand dort bereits eine Silhouette zwischen den Reihen von alten, verstaubten Tischen. Eine Silhouette, die einen dünnen, grün leuchtenden Stab hielt, von dem ein blasses Licht ausging, welches kaum seinen Halter erleuchtete – vom Raum ganz zu schweigen. 

Das Licht aus dem Korridor schwand, als die Tür hinter ihm zu fiel und Dracos Augen begannen, sich an das schummrige Glimmen zu gewöhnen. 

Die Silhouette wandte sich langsam ihm zu und in den Schatten erkannte er ein Gesicht, das nur teilweise vom gespenstischen grünen Licht erhellt wurde. 

Dieses Treffen gefiel Draco jetzt schon. Das kühle grüne Licht passte schon; wenn sie beide jetzt noch etwas größer wären, Kapuzen und Masken tragen und sich auf einem Friedhof statt in einem Klassenzimmer treffen würden, dann wäre es genau wie in den Todessergeschichten, die die Freunde seines Vaters immer erzählen. 

„Ich will, dass du weißt, Draco Malfoy“, sagte die Silhouette in gefährlich ruhigem Ton, „dass ich dir keine Schuld an meiner jüngsten Niederlage gebe.“ 

Draco öffnete unwillkürlich seinen Mund um zu widersprechen, es gab absolut keinen Grund, weshalb er dafür die Schuld – 

„Es lag vor allem an meiner eigenen Dummheit“, fuhr die Figur in den Schatten fort. „Ich hätte mich bei mehreren Gelegenheiten anders verhalten können. Du hast mich nicht gebeten, \emph{genau das} zu tun, was ich getan habe. Du hast nur um Hilfe gebeten. Ich war derjenige, der eine unkluge Art zu helfen gewählt hat. Doch es bleibt dabei, dass ich den Wettstreit um ein halbes Buch verloren habe. Das Handeln deines idiotischen Lakaien und der Gefallen, um den du mich gebeten hast, und, ja, mein eigenes unkluges Handeln in jener Angelegenheit haben mich \emph{Zeit gekostet}. Mehr Zeit als dir bewusst ist. Zeit, die sich schließlich als entscheidend herausstellte. Es bleibt dabei, Draco Malfoy, dass ich gewonnen \emph{hätte}, wenn du mich nicht um diesen Gefallen gebeten hättest. Stattdessen … habe ich nun … \emph{verloren}.“ 

Draco hatte bereits von Harrys Niederlage gehört und von dem Wetteinsatz, den Granger verlangt hatte. Die Neuigkeit hatte sich schneller rumgesprochen als es per Eule möglich gewesen wäre. 

„Ich verstehe“, sagte Draco. „Es tut mir Leid.“ Er konnte nichts anderes sagen, wenn er noch mit Harry Potter befreundet sein wollte. 

„Ich bitte nicht um dein Verständnis oder Mitgefühl“, sagte die dunkle Silhouette, immer noch in jenem gefährlich ruhigen Ton. „Doch ich habe soeben zwei volle Stunden in der Gegenwart von Hermine Granger verbracht, die mir zur Verfügung gestellte Kleidung getragen und so faszinierende Orte in Hogwarts besucht wie einen klitzekleinen, plätschernden Wasserfall, der aussah als ob Nasenschleim draus hervorsprudelte. Und das alles in Begleitung von einigen anderen Mädchen, die sich auf so wundervolle Tätigkeiten verlegt haben wie unseren Weg mit verwandelten Rosenblüten zu bestreuen. Ich war auf einem Date, junger Malfoy. Auf meinem \emph{ersten} Date. \emph{Und wenn ich dafür eine Wiedergutmachung verlange, dann wirst du sie erbringen.}“ 

Draco nickte ernsthaft. Vor diesem Treffen hatte er in weiser Voraussicht so viele Details wie möglich über Harrys Date in Erfahrung gebracht. So konnte er \emph{vor} ihrem Treffen in hysterisches Gelächter ausbrechen und beging nicht den Fauxpas, nun so lange zu kichern, bis ihm schwarz vor Augen wurde. 

„Glaubst du“, sagte Draco, „dass dem Granger-Mädchen irgend ein tragisches Unglück zustoßen –“ 

„Erzähle den Slytherins, dass \emph{ich} mich um Granger kümmern werde und dass ich die Überreste eines jeden, der sich in \emph{meine} Angelegenheiten einmischt, über ein dutzend verschiedene Länder verteilen werde. Und da ich nicht in Gryffindor bin und \emph{mit List} vorgehen werde, statt einfach draufzuhauen, sollen sie sich keine Sorgen machen, falls sie mich dabei erwischen, dass ich Hermine anlächle.“ 

„Oder wenn sie dich auf einem zweiten Date sehen?“, sagte Draco mit einem zweifelnden Unterton. 

\emph{„Es wird kein zweites Date geben“}, sagte die grün beleuchtete Figur in einer so furchteinflößenden Stimme, dass sie nicht nur wie irgendein Todesser klang, sondern wie Amycus Carrow damals, kurz bevor Vater ihm gesagt hatte, er solle damit aufhören; er sei nicht der Dunkle Lord. 

Aber natürlich \emph{war} es immer noch die hohe Stimme eines kleinen Jungen und wenn man das mit den \emph{gesprochenen Wörtern} kombinierte – nein, es passte einfach nicht. Falls Harry Potter eines Tages \emph{doch} zum nächsten Dunklen Lord geworden wäre, dann würde Draco eine Kopie dieser Erinnerung an einem sicheren Ort aufbewahren und Harry Potter würde es nie wagen, ihn zu hintergehen. 

„Doch sprechen wir über schönere Dinge“, sagte die Figur inmitten der grünlichen Schatten. „Sprechen wir über Wissen und über Macht. Draco Malfoy, sprechen wir über Wissenschaft.“

„Ja“, sagte Draco. „Reden wir darüber.“ 

Draco fragte sich, wie viel von seinem eigenen Gesicht man in diesem spukhaften grünen Licht erkennen konnte und wie viel in den Schatten lag. 

Und obwohl Dracos Gesichtsausdruck ernst blieb, trug er ein Lächeln in seinem Herzen. 

\emph{Endlich} führte er ein erwachsenes Gespräch. 

„Ich biete dir Macht“, sagte die Figur in den Schatten, „und ich werde dir von dieser Macht und ihrem Preis erzählen. Die Macht ist, das Wesen der Realität zu erkennen und so Einfluss über sie zu gewinnen. Was du verstehst, das kannst du steuern. Diese Macht ist groß genug, dass sie uns auf den Mond bringen kann. Der Preis dieser Macht ist, dass du lernen musst, der Natur Fragen zu stellen und, viel schwieriger noch, ihre Antworten hinzunehmen. Du wirst Experimente durchführen, Fragen stellen, und beobachten, was passiert. Und du musst diese Antworten akzeptieren, wenn sie dir mitteilen, dass du dich geirrt hast. Du musst \emph{lernen zu verlieren} – nicht gegen mich, sondern gegen die Natur. Wenn du merkst, dass du gegen die Realität ankämpfst, musst du die Realität gewinnen lassen. Es wird dir wehtun, Draco Malfoy, und ich weiß nicht, ob du stark genug dafür bist. Wenn dies der Preis ist – ist es dennoch dein Wunsch, diese Macht zu erlernen?“ 

Draco atmete tief ein. Er dachte darüber nach. Und er wusste nicht, wie er anders antworten könnte. Er hatte die Anweisung, jede Möglichkeit zu ergreifen um Freundschaft mit Harry Potter zu schließen. Es ging nur um’s \emph{Lernen}, er versprach nicht, irgendetwas zu \emph{tun}. Er konnte diesen Unterricht jederzeit abbrechen … 

Es sah in vielerlei Hinsicht nach einer Falle aus, doch ehrlich gesagt sah Draco nicht, wie es schief gehen könnte. 

Außerdem wollte Draco schon irgendwie die Welt beherrschen. 

„Ja“, sagte Draco. 

„Exzellent“, sagte die Figur in den Schatten. „Ich hatte eine recht \emph{stressige Woche} und werde einige Zeit brauchen, um einen Lehrplan zusammenzustellen –“ 

„Ich habe selbst viel zu tun, um meine Macht in Slytherin zu festigen“, sagte Draco, „von Hausaufgaben ganz zu schweigen. Vielleicht sollten wir einfach im Oktober anfangen?“ 

„Klingt vernünftig“, sagte die Figur in den Schatten, „aber worauf ich hinauswollte ist, dass ich wissen muss, was ich dir beibringen werde, um den Lehrplan zu entwickeln. Mir sind drei Ideen gekommen. Die erste ist, dass ich dir etwas über das menschliche Gehirn und Bewusstsein beibringe. Die zweite ist, dass ich dir etwas über das Universum beibringe; jene Künste, die nötig sind, um zum Mond zu reisen. Dafür würden wir uns mit vielen Zahlen beschäftigen, aber für manche Menschen sind gerade jene Zahlen schöner als alles andere, was die Wissenschaften uns lehren können. Magst du Zahlen, Draco?“ 

Draco schüttelte den Kopf. 

„Soviel dazu. Du wirst irgendwann dennoch Mathematik lernen, aber noch nicht jetzt, würde ich sagen. Die dritte Möglichkeit ist, dass ich dich Genetik und Evolution und Vererbbarkeit lehre; die Dinge, die du ‚Blut‘ nennst –“ 

„Das nehmen wir“, sagte Draco. 

Die Figur nickte. „Ich ahnte, dass du dich so entscheiden würdest. Aber ich glaube, dass das für dich der schmerzhafteste Pfad sein wird, Draco. Was, wenn deine Familie und Freunde, die Reinblütler, das eine sagen und du feststellst, dass die Experimente etwas anderes sagen?“ 

„Dann werde ich herausfinden, wie ich die Experimente dazu kriege, die \emph{richtige} Antwort auszuspucken!“ 

Einen Moment lang war es still, während die Figur in den Schatten mit offenem Mund dastand. 

„Ähm“, sagte die Figur. „So funktioniert das nicht. Davor wollte ich dich warnen, Draco. Du \emph{kannst} nicht dafür sorgen, dass die Antwort rauskommt, die du dir wünschst.“ 

„Du kannst \emph{immer} dafür sorgen, dass die Antwort rauskommst, die du dir wünschst“, sagte Draco. Das war so ziemlich das Erste, was seine Tutoren ihm beigebracht hatten. „Man muss bloß die richtigen Argumente finden.“ 

„Nein“, sagte die Figur in vor Frust erhobener Stimme. „Nein, nein, nein! Dann hast du die \emph{falsche Antwort} und wirst damit nie zum Mond kommen! Die Natur ist keine Person, du kannst sie nicht durch einen Trick hereinlegen. Wenn du dem Mond erzählst, dass er aus Käse besteht, dann kannst du tagelang auf ihn einreden und es wird den Mond kein bisschen verändern! Du begehst \emph{Rationalisierung}, das ist, wie wenn du ein Blatt Papier nimmst und in die letzte Zeile schreibst ‚und folglich besteht der Mond aus Käse‘ und dann anfängst, alle möglichen schlauen Argumente auf das Blatt zu schreiben. Aber entweder der Mond besteht aus Käse oder er besteht nicht aus Käse. In dem Moment, in dem du dieses Fazit hingeschrieben hast, war es bereits richtig oder falsch. Ob du am Ende auf das richtige oder falsche Ergebnis kommst, steht fest, sobald du das Fazit hingeschrieben hast. Wenn du dich zwischen zwei teuren Koffern entscheiden sollst und dir der glitzernde besser gefällt, dann ist es egal, mit welchen schlauen Gründe du die Kaufentscheidung später begründest; die Regel, der du \emph{in Wirklichkeit} gefolgt bist, lautet ‚wähle den glitzernden Koffer‘. Und je nachdem, wie gut diese Regel darin ist, den besseren Koffer zu identifizieren, wirst du den besseren oder den schlechteren Koffer kaufen. Rationalität kann \emph{nicht} dazu verwendet werden, für eine bestimmte Seite zu streiten; sie kann nur helfen zu ermitteln, \emph{für welche Seite} du streiten solltest. Wissenschaft ist nicht dazu da, irgendwen zu \emph{überreden}, dass er für die Reinblütler sein soll. Das ist \emph{Politik}! Wissenschaft ist so mächtig, weil wir damit herausfinden können, \emph{wie die Natur sich tatsächlich verhält, egal was unsere Meinung ist!} Wissenschaft \emph{kann} uns aber beibringen, \emph{wie Blut tatsächlich funktioniert}, wie Zauberer ihre Fähigkeiten von ihren Eltern erben und ob Muggelgeborene schwächer oder stärker –“ 

\emph{„Stärker!“}, sagte Draco. Er hatte versucht, Harrys Erklärungen nachzuvollziehen. Sein Gesichtsausdruck war grübelnd; er hatte erkannt, dass es \emph{irgendwie} Sinn ergab, obwohl es völlig anders war als alles, was er zuvor gehört hatte. Und dann hatte Harry Potter etwas gesagt, was Draco unmöglich stehenlassen konnte. „Du findest, Schlammblüter sind \emph{stärker}?“ 

„Ich finde gar nichts“, sagte die Figur in den Schatten. „Ich weiß gar nichts. Ich glaube gar nichts. Ich habe noch nichts in die letzte Zeile geschrieben. Ich werde herausfinden, wie ich die magische Stärke von Muggelgeborenen und die magische Stärke von Reinblütern messen kann. Wenn meine Messungen herausfinden, dass Muggelgeborene schwächer sind, dann werde ich glauben, dass sie schwächer sind. Wenn meine Messungen herausfinden, dass Muggelgeborene stärker sind, dann werde ich glauben, dass sie stärker sind. Wenn ich diese und weitere Tatsachen herausfinde, dann werde ich eine gewisse Macht erlangen –“ 

„Und du erwartest von mir, dass ich einfach glaube, was immer du sagst?“, fragte Draco empört. 

„Ich erwarte, dass du die Messungen \emph{selbst} durchführst“, sagte die Figur in den Schatten ruhig. „Hast du Angst vor dem, was \emph{du} herausfinden wirst?“ 

Draco starrte die Figur in den Schatten eine Weile lang mit zusammengekniffenen Augen an. „Schöne Falle, Harry“, sagte er. „Ich muss sie mir merken, die kannte ich noch nicht.“ 

Die Figur in den Schatten schüttelte den Kopf. „Es ist keine Falle, Draco. Denk daran – ich \emph{weiß selbst nicht}, was wir herausfinden werden. Aber du wirst das Universum nicht verstehen, indem du dich mit ihm streitest oder ihm sagst, dass es gefälligst eine andere Antwort liefern soll. Wenn du den Umhang eines Wissenschaftlers überstreifst, dann musst du all die Politik und Diskussionen und Gruppen und Seiten vergessen, du musst das sehnsüchtige Verlangen in deinem Kopf zum Schweigen bringen und nur danach streben, die Antwort der Natur zu erfahren.“ Die Figur schwieg kurz. „Die meisten Leute können das nicht. Darum ist es so schwierig. Bist du dir sicher, dass du nicht doch lieber etwas über das Gehirn lernen möchtest?“ 

„Und wenn ich dir sage, dass ich lieber etwas über das Gehirn lerne“, sagte Draco mit harter Stimme, „dann wirst du herumerzählen, dass ich Angst davor hatte, was ich herausgefunden hätte.“ 

„Nein“, sagte die Figur in den Schatten. „Ich werde nichts dergleichen tun.“ 

„Aber du würdest die Messungen selbst durchführen und wenn du die falsche Antwort herausfindest, dann könnte ich nichts tun, bevor du jemand anderem davon erzählst.“ Dracos Stimme war immer noch hart. 

„Ich würde dich trotzdem vorher fragen, Draco“, sagte die Figur in den Schatten sanft. 

Draco schwieg. Er hatte das nicht erwartet; hatte gedacht, dass er die Falle entdeckt hätte, aber … „Das würdest du tun?“ 

„Natürlich. Woher sollte \emph{ich} sonst erfahren, wen wir damit erpressen oder was wir verlangen könnten? Draco, ich sage es nochmal: Das ist keine Falle, die ich für dich aufgestellt habe. Zumindest nicht für dich persönlich. Wenn du andere Ansichten hättest, dann hätte ich gefragt ‚Was, wenn die Messungen ergeben, dass Reinblütler stärker sind?‘“ 

„Wirklich.“ 

„\emph{Ja!} Das ist der Preis, den \emph{jeder} bezahlen muss, um Wissenschaftler zu werden!“ 

Draco hob eine Hand. Er musste nachdenken. 

Die grünlich schimmernde Figur in den Schatten wartete. 

Draco musste jedoch nicht lange darüber nachdenken. Wenn man all die verwirrenden Dinge mal beiseite ließ … dann plante Harry Potter, mit etwas herumzuexperimentieren, was ein riesiges politisches Erdbeben verursachen konnte und es wäre wahnsinnig, einfach wegzusehen und es ihm alleine zu überlassen. „Wir werden uns mit Blut beschäftigen“, sagte Draco. 

\emph{„Exzellent“}, sagte die Figur und lächelte. „Ich gratuliere dir dafür, dass du bereit bist, die Frage zu stellen.“ 

„Danke“, sagte Draco und schaffte es nicht ganz, die Ironie aus seiner Stimme zu verbannen. 

„Hey, hast du etwa geglaubt, es wäre \emph{einfach}, zum Mond zu fliegen? Sei froh, dass wir bloß ab und zu die Meinung ändern müssen und keine Menschenopfer darbringen brauchen!“ 

„Menschenopfer darbringen wäre \emph{viel} einfacher!“ 

Einen Moment lang war es still, dann nickte die Figur. „Auch wieder wahr.“ 

„Sag mal, Harry“, sagte Draco, ohne sich große Hoffnungen zu machen, „ich dachte, wir wollen uns das gesamte Wissen der Muggel aneignen, es mit dem Wissen der Zauberer verbinden und so Herrscher über beide Welten werden. Wäre es nicht viel einfacher, all das zu lernen, was die Muggel längst herausgefunden haben, wie das mit dem Mond, und \emph{diese} Macht zu nutzen –“ 

\emph{„Nein“}, sagte die Figur und schüttelte energisch den Kopf, wodurch grüne Schatten auf seinem Gesicht tanzten. Seine Stimme war nun sehr ernst. „Falls du diese Kunst der Wissenschaft nicht erlernen kannst, die Wirklichkeit nicht akzeptieren kannst, dann \emph{darf ich dir nicht verraten}, was die Wissenschaften dadurch herausgefunden haben. Das ist wie wenn dir ein mächtiger Zauberer von jenen Toren erzählt, die nicht geöffnet werden dürfen, von den Siegeln, die nicht gebrochen werden dürfen, bis du deine Intelligenz und Strebsamkeit unter Beweis gestellt hast, indem du die niederen Schrecken überlebst.“ 

Draco lief es kalt den Rücken herunter und ihm schauderte unwillkürlich. Er wusste, dass es auch im Halbdunkel sichtbar gewesen war. „Na gut“, sagte Draco. „Ich verstehe.“ Vater hatte ihm das viele Male gesagt: Wenn ein mächtigerer Zauberer dir sagt, dass du nicht bereit bist, etwas zu wissen, dann forschst du nicht weiter nach, wenn dein Leben dir etwas bedeutet. 

Die Figur senkte den Kopf. „Gut. Doch es gibt noch etwas, was du verstehen solltest. Die ersten Wissenschaftler waren Muggel, kannten diese Tradition also nicht. Anfangs verstanden sie einfach nicht, dass es gefährliches Wissen gibt, und glaubten, dass sie über alle Dinge offen reden sollten. Als ihre Forschung gefährliche Dinge zutage brachte, berichteten sie ihren Politikern von etwas, das besser geheim geblieben wäre – schau nicht so, Draco, es war nicht einfach aus Dummheit; sie waren klug genug, um diese Dinge überhaupt herauszufinden. Aber es waren Muggel, es war das erste Mal, dass sie etwas \emph{wirklich} Gefährliches herausgefunden hatten, und solche Geheimhaltung hatte noch keine Tradition. Es herrschte ein Krieg und die Wissenschaftler der einen Seite sorgten sich, dass – wenn sie \emph{nichts} sagten – die Wissenschaftler des feindlichen Landes es \emph{ihren} Politikern sagen würden …“ Die Stimme senkte sich bedeutungsschwer. „Sie haben die Welt nicht zerstört. Aber es war knapp. Und \emph{wir} werden diesen Fehler nicht wiederholen.“ 

„Genau“, sagte Draco, nun mit fester Stimme. „\emph{Wir} nicht. Wir sind Zauberer und werden nicht zu Muggeln, bloß weil wir Wissenschaft betreiben.“ 

„Du hast Recht“, sagte die vom grünen Licht umrissene Figur. „Wir werden unsere \emph{eigene} Wissenschaft begründen, eine magische Wissenschaft, und wir werden von Anfang an bessere Traditionen etablieren.“ Die Stimme wurde fester. „Neben dem Wissen, das ich mit dir teile, werde ich dich auch lehren, die Wahrheit zu akzeptieren. Neues beibringen werde ich dir nur, wenn du in jener Fähigkeit Fortschritte machst und du wirst all das Wissen an niemanden weitergeben, der jene Fähigkeit nicht entwickelt hat. Bist du damit einverstanden?“ 

„Ja“, sagte Draco. Was sollte er sonst tun? Nein sagen? 

„Gut. Und was du selbst entdeckst, wirst du für dich behalten, bis du glaubst, dass andere Wissenschaftler bereit sind, es zu erfahren. Und unsere gemeinsamen Erkenntnisse veröffentlichen wir erst dann, wenn wir uns einig sind, dass die Welt dafür bereit ist. Und egal welche Ansichten und Bündnisse wir haben – \emph{wir alle} werden \emph{jeden einzelnen} von uns bestrafen, der gefährliche Magie veröffentlicht oder gefährliche Waffen weitergibt, egal was für ein Krieg auch herrschen mag. Von heute an wird das Tradition und Gesetz der magischen Wissenschaft sein. Sind wir uns da einig?“ 

„Ja“, sagte Draco. Das klang allmählich wirklich recht attraktiv. Die Todesser hatten versucht, Macht zu erlangen, indem sie furchteinflößender als alle anderen waren, und sie hatten es noch nicht endgültig geschafft. Vielleicht war es an der Zeit, stattdessen mithilfe von Geheimnissen zu herrschen. „Und unsere Gruppe bleibt so lange wie möglich geheim und jeder, der teilnehmen will, muss unseren Regeln zustimmen.“ 

„Natürlich. Auf jeden Fall.“ 

Einen Moment lang war es still. 

„Wir brauchen bessere Umhänge“, sagte die Figur in den Schatten, „mit Kapuzen und so …“ 

„Darüber habe ich auch gerade nachgedacht“, sagte Draco. „Aber wir brauchen keine völlig neuen Umhänge, nur Kappen zum Überwerfen. Ich habe eine Freundin in Slytherin, sie wird uns maßgeschneiderte –“ 

„Aber sag ihr nicht, \emph{wofür} wir die brauchen –“ 

„Ich bin nicht \emph{blöd}!“ 

„Und wir brauchen erstmal keine Masken; nicht, solange es nur wir zwei sind“, sagte die Figur in den Schatten. 

„Okay! Aber später sollten wir uns irgendein Erkennungszeichen ausdenken, das alle unsere Anhänger haben, das Mal der Wissenschaft. Zum Beispiel eine Schlange, die den Mond frisst, auf dem rechten Unterarm –“ 

„Du meinst einen Doktortitel – und wäre es dann nicht viel zu einfach, unsere Leute zu identifizieren?“ 

„Hä?“ 

„Ich meine, was wenn jemand sagt ‚okay, jetzt krempelt alle mal euren rechten Ärmel hoch‘ und unser Anhänger sagt dann ‚ups, tja, sieht so aus, als ob ich ein Spion bin‘ –“ 

\emph{„Vergiss, dass ich was gesagt habe“}, sagte Draco, dem plötzlich am ganzen Körper der Schweiß ausbrach. Er brauchte eine Ablenkung, \emph{schnell} – „Und wie wollen wir uns nennen? Die Wissenschaftsesser?“ 

„Nein“, sagte die Figur in den Schatten langsam. „Das klingt falsch …“ 

Draco wischte sich mit dem Ärmel seines Umhangs den Schweiß von der Stirn. Was hatte der Dunkle Lord sich dabei bloß \emph{gedacht}? Vater hatte gesagt, der Dunkle Lord sei \emph{schlau} gewesen! 

„Ich hab’s“, sagte die Figur in den Schatten plötzlich. „Du wirst es noch nicht verstehen, aber glaub mir, es passt.“ 

In diesem Moment wäre Draco auch mit „Malfoymampfer“ einverstanden gewesen, solange sie bloß das Thema wechselten. „Was denn?“ 

Und inmitten der verstaubten Tische in einem ungenutzten Klassenzimmer in den Kerkern von Hogwarts breitete die grün beleuchtete Figur von Harry Potter dramatisch die Arme aus und sagte: „Am heutigen Tage gründen wir … den \emph{Bayes-Geheimbund}.“ 

\later 

Eine stumme Person schlurfte erschöpft durch die Flure von Hogwarts zum Ravenclaw-Turm. 

Harry war nach dem Treffen mit Draco sofort zum Essen gegangen, dort gerade lange genug geblieben, um rasch ein paar Bissen Essen runterzuwürgen, bevor er ins Bett ging. 

Es war nicht einmal 19~Uhr, doch für Harry war es längst Schlafenszeit. Er hatte \emph{letzte} Nacht festgestellt, dass er den Zeitumkehrer am Samstag erst nach Ende des Lesewettstreits wieder benutzen konnte. Doch er konnte den Zeitumkehrer immer noch am \emph{Freitagabend} nutzen und so Zeit gewinnen. Also hatte Harry sich dazu gezwungen, am Freitag wach zu bleiben, bis die Schutzhülle sich um 21~Uhr öffnete, um dann die vier verbleibenden Stunden zu nutzen, nach 17~Uhr zurückzureisen und erschöpft ins Bett zu fallen. Er war am Samstagmorgen wie geplant gegen zwei Uhr aufgewacht und hatte dann zwölf Stunden am Stück gelesen … und es hatte trotzdem nicht ausgereicht. Und jetzt würde Harry einige Tage lang recht früh ins Bett gehen, bis sein Schlafrhythmus sich wieder eingependelt hatte. 

Das Bild vor dem Eingang stellte Harry irgendein dummes Rätsel, das für Elfjährige gedacht war, und er antwortete, ohne über die Worte nachzudenken. Dann stolperte Harry die Treppe hoch zu seinem Schlafsaal, zog seinen Schlafanzug an und fiel ins Bett. 

Und bemerkte, dass sein Kissen recht hart war. 

Harry stöhnte auf. Er setzte sich widerwillig auf, drehte sich um und hob sein Kissen an. 

Er entdeckte einen Zettel, zwei goldene Galleonen und ein Buch namens \emph{Okklumentik: Geheime Kunst}. 

Harry griff nach dem Zettel und las: 

\begin{writtenNote}
Meine Güte, du bist ja gut darin, Probleme heraufzubeschwören. Dein Vater war nichts dagegen. 

Du hast dir einen mächtigen Feind gemacht. Ganz Slytherin vertraut, bewundert und fürchtet Snape. Du kannst nun keinem Angehörigen dieses Hauses mehr trauen, egal ob sie Freund oder Feind zu sein scheinen. 

Von nun an darfst du Snape nicht in die Augen sehen. Er ist ein Legilimens und kann sonst deine Gedanken lesen. Ich habe ein Buch beigelegt, mit dem du lernen kannst, dich zu schützen. Allzu weit wirst du ohne einen Lehrer zwar nicht kommen, es könnte dir aber gelingen, Angriffe zumindest zu erkennen. 

Damit du etwas Zeit findest um Okklumentik zu studieren, habe ich zwei Galleonen beigelegt – den Preis einer Musterlösung für alle Tests und Aufsätze im ersten Jahr des Geschichtsunterrichts. (Professor Binns hat seit seinem Tod jedes Jahr genau die selben Aufgaben gestellt.) Deine neuen Freunde, die Weasley-Zwillinge, sollten dir eine Kopie verkaufen können. Es versteht sich von selbst, dass du dich nicht damit erwischen lassen darfst. 

Über Professor Quirrell weiß ich wenig. Er ist ein Slytherin und ein Verteidigungslehrer; zwei Dinge, die gegen ihn sprechen. Prüfe jeden Ratschlag, den er dir gibt, sorgfältig und sprich mit ihm nicht über Dinge, die du geheim halten möchtest 

Dumbledore tut nur so, als wäre er verrückt. Er ist äußerst intelligent und falls du wieder in einen Wandschrank gehst und dann verschwindest, wird er sicherlich schlussfolgern, dass du einen Unsichtbarkeitsumhang besitzt, falls er es nicht längst getan hat. Meide ihn wann immer möglich, verstecke den Unsichtbarkeitsumhang an einem sicheren Ort (\emph{nicht} in deinem Beutel), wann immer du ihn nicht meiden kannst, und verhalte dich in seiner Gegenwart äußerst vorsichtig. 

Bitte sei in Zukunft vorsichtiger, Harry Potter. 

—Santa Claus
\end{writtenNote}

Harry starrte den Zettel an. 

Das \emph{schienen} ziemlich gute Ratschläge zu sein. Natürlich würde Harry im Geschichtsunterricht nicht schummeln, selbst wenn er ein totes Pferd als Lehrer hätte. Doch Snapes Legilimentik … wer auch immer ihm diesen Zettel geschrieben hatte, der wusste viele wichtige, geheime Dinge und war dazu bereit, sie Harry mitzuteilen. Der Zettel warnte ihn zwar immer noch davor, dass Dumbledore den Umhang stehlen würde, doch im Moment hatte Harry keine Ahnung, ob das ein schlechtes Zeichen war oder einfach ein nachvollziehbarer Fehler. 

Irgendeine Art Verschwörung schien in Hogwarts stattzufinden. Vielleicht könnte Harry die Erzählungen von Dumbledore und vom Autor dieses Zettels miteinander \emph{abgleichen} und so ein \emph{Gesamtbild} erhalten, das zutraf? Wenn \emph{beide} sich in einer Sache einig waren, dann … 

… egal … 

Harry stopfte alles in seinen Beutel, aktivierte den Stillezauber, zog die Decke über den Kopf und verlor das Bewusstsein. 

\later 

Es war ein Sonntagmorgen und Harry aß in der Großen Halle Eierkuchen, biss rasch davon ab und warf zwischendurch alle paar Sekunden nervöse Blicke auf seine Armbanduhr. 

Es war 8:02~Uhr und in genau zwei Stunden und einer Minute war es \emph{exakt eine Woche her}, dass er die Weasleys gesehen hatte und zum Bahnsteig Neundreiviertel gegangen war. 

Und ihm war plötzlich der Gedanke gekommen … Harry wusste nicht, ob es zulässig war, so zu argumentieren; er wusste gar nichts mehr, aber es erschien \emph{möglich} … 

Dass … 

\emph{… ihm in der letzten Woche noch nicht genug interessante Dinge zugestoßen waren.} 

Wenn er fertig gegessen hatte, wollte Harry geradewegs in sein Zimmer gehen, sich in dem Raum in seinem Koffer verstecken und bis 10:03~Uhr mit niemandem ein Wort wechseln. 

Und dann sah Harry die Weasley-Zwillinge auf sich zukommen. Einer der beiden versteckte etwas hinter seinem Rücken. 

Er sollte aufschreien und wegrennen. 

Er sollte aufschreien und wegrennen. 

Was auch immer das war, es könnte … 

… der \emph{krönende Abschluss} sein … 

Er sollte wirklich einfach aufschreien und wegrennen. 

Mit dem resignierten Gefühl, dass das Universum ihn \emph{trotzdem} aufspüren würde, schnitt Harry ein weiteres Stück vom Eierkuchen ab. Er konnte die nötige Energie nicht aufbringen. Das war die traurige Wahrheit. Harry wusste nun, wie Menschen sich fühlten, wenn sie des Wegrennens müde waren, wenn sie dem Schicksal nicht mehr entkommen konnten und einfach zu Boden fielen und sich von den grauenhaften, scharfzähnigen und tentakelbesetzten Dämonen der tiefsten Abgründe in ihr fürchterliches Schicksal ziehen ließen. 

Die Weasley-Zwillinge kamen näher. 

Und noch näher. 

Harry aß noch ein Stück Eierkuchen. 

Die Weasley-Zwillinge kamen an und grinsten fröhlich. 

„Hallo, Fred“, sagt Harry matt. Einer der Zwillinge nickte. „Hallo, George.“ Der andere Zwilling nickte. 

„Du klingst müde“, sagte George. 

„Du solltest fröhlicher dreinschauen“, sagte Fred. 

„Schau mal, was \emph{wir} dir mitgebracht haben!“ 

Und George zog hinter Freds Rücken – 

– einen Kuchen hervor. Mit zwölf brennenden Kerzen. 

Einen Moment lang war es still, als ganz Ravenclaw sie anstarrte. 

„Das stimmt nicht“, sagte jemand. „Harry Potter wurde am einunddreißigsten Jul–“ 

„ER WIRD KOMMEN“, ertönte eine laute, hallende Stimme, die alle Gespräche durchschnitt wie ein eisiges Schwert. „UND ZERSTÖREN WIRD ER SELBST DIE –“ 

Dumbledore war von seinem Platz aufgesprungen und über den Lehrertisch gesprungen und hatte die Frau ergriffen, die diese fürchterlichen Worte sprach, Fawkes war plötzlich erschienen und alle drei verschwanden in einer auflodernden Flamme. 

Es folgte eine schockierte Stille … 

… und viele Köpfe wandten sich Harry Potter zu. 

„Ich war’s nicht“, sagte Harry in einer erschöpften Stimme. 

„Das war eine \emph{Prophezeiung}!“, zischte jemand am Tisch. „Und ich wette, dass es um \emph{dich} ging!“ 

Harry seufzte. 

Er stand auf und versuchte mit lauter Stimme all die Gespräche zu übertönen, die nun anfingen. \emph{„Es ging nicht um mich! Natürlich nicht! Ich werde nicht kommen, ich bin schon hier!“} 

Harry setzte sich wieder. 

Die Schüler, die zu ihm geblickt hatten, wandten sich wieder ab. 

Jemand anders am Tisch fragte: „Aber um wen ging es dann?“ 

Und Harry spürte eine dumpfe, bleierne Schwere, als ihm klar wurde, wer \emph{noch nicht} in Hogwarts war. 

Es war bloß eine wilde Vermutung, aber Harry hatte das Gefühl, dass der untote Dunkle Lord in nächster Zeit hier auftauchen würde. 

Die Gespräche um ihn herum wurden fortgesetzt. 

„Und außerdem: \emph{Was} wird er zerstören?“ 

„Ich glaube, Trelawney hat begonnen, irgendwas mit ‚S‘ zu sagen, als der Schulleiter sie ergriffen hat.“ 

„Also … Seele? Sonne?“ 

„Wenn jemand die Sonne zerstören wird, dann haben wir ein echtes Problem!“ 

Das erschien Harry sehr unwahrscheinlich, es sei denn, auf dieser Welt gab es furchterregende Wesen, die von David Criswells Starlifting-Ideen gelesen hatten. 

„Sagt mal“, sagte Harry mit müder Stimme, „das passiert jeden Sonntag beim Frühstück, oder?“ 

„Nein“, sagte ein Schüler, der im siebten Schuljahr sein könnte, und runzelte die Stirn. „Normalerweise nicht.“ 

Harry zuckte mit den Schultern. „Egal. Mag jemand ein Stück Geburtstagskuchen?“ 

„Aber du hast heute nicht Geburtstag!“, sagte der gleiche Schüler, der vorhin schon widersprochen hatte. 

Fred und George begannen daraufhin natürlich schallend zu lachen. 

Selbst Harry gelang ein müdes Lächeln. 

Als er das erste Stück bekam, sagte Harry: „Ich hatte eine \emph{wirklich lange Woche}.“ 

\later 

Und Harry saß in dem Raum in seinem Koffer, der von innen geschlossen und abgesperrt war, sodass niemand reinkommen konnte. Er hatte sich eine Decke über den Kopf gezogen und wartete darauf, dass diese Woche zuende ging. 

10:01~Uhr. 

10:02~Uhr. 

10:03~Uhr. Aber sicherheitshalber … 

10:04~Uhr. Die erste Woche war vorbei. 

Harry atmete erleichtert auf und guckte vorsichtig unter der Decke hervor. 

Etwas später stand er in seinem hellen, lichtdurchfluteten Schlafsaal. 

Kurz darauf war er im Gemeinschaftsraum der Ravenclaws. Einige Schüler sahen auf, doch niemand sagte etwas oder versuchte, mit ihm zu sprechen. 

Harry suchte sich einen schön großen Schreibtisch, nahm sich einen bequemen Stuhl und setzte sich. Aus seinem Beutel zog er ein Blatt Papier und einen Bleistift hervor. 

Seine Eltern hatten ihm unmissverständlich gesagt, dass sie zwar Verständnis dafür hatten, wenn er gerne von zuhause und von seinen Eltern weg wäre, dass er ihnen aber \emph{ohne Ausnahme jede Woche} einen Brief schreiben sollte, damit sie wüssten, dass er lebte, dass es ihm gut gehe und er nicht im Gefängnis gelandet sei. 

Harry starrte auf das weiße Blatt. \emph{Also …} 

Seitdem er sich auf dem Bahnhof von seinen Eltern verabschiedet hatte … 

… hatte er einen Jungen kennengelernt, der von Darth Vader aufgezogen wurde. Er hatte sich mit drei berüchtigten Scherzbolden angefreundet, hatte Hermine getroffen, dann die Sache mit dem Sprechenden Hut … am Montag hatte er eine Zeitmaschine bekommen, um seine Schlafstörung zu behandeln, er hatte einen legendären Unsichtbarkeitsumhang von einem unbekannten Wohltäter erhalten, sieben Hufflepuffs gerettet, indem er fünf angsteinflößenden älteren Jungs gegenübergetreten war, von denen einer gedroht hatte, ihm die Finger zu brechen, hatte gemerkt, dass er eine mysteriöse dunkle Seite besaß, hatte im Zauberkunst-Unterricht \emph{Frigideiro} gelernt und eine Rivalität mit Hermine gestartet … am Dienstag hatten sie Astronomie bei Professor Aurora Sinistra, die nett war, und Geschichte der Zauberei, was von einem Geist unterrichtet wurde, der exorziert und durch einen Kassettenrekorder ersetzt werden sollte … am Mittwoch wurde er zum gefährlichsten Schüler im Klassenzimmer ernannt … am Donnerstag … damit fangen wir lieber gar nicht erst an … am Freitag das Ereignis im Zaubertränke-Unterricht, dann hatte er den Schulleiter erpresst, dann ließ der Verteidigungslehrer ihn im Unterricht verprügeln, dann stellte sich heraus, dass der Verteidigungslehrer der wundervollste Mensch war, der je auf dem Antlitz der Erde wandelte … am Samstag hatte er eine Wette verloren, sein erstes Date gehabt und angefangen, Draco auf die Seite des Guten zu holen … und dann heute Morgen Trelawneys Prophezeiung, die womöglich bedeutete, dass ein unsterblicher Dunkler Lord Hogwarts angreifen würde. 

Harry sammelte seine Gedanken und begann zu schreiben. 

\begin{writtenNote}
\letterAddress{Liebe Mama, lieber Papa!}

Hogwarts ist richtig toll. Ich habe im Zauberkunst-Unterricht gelernt, gegen den zweiten Hauptsatz der Thermodynamik zu verstoßen, und ich habe ein Mädchen namens Hermine Granger kennengelernt, die schneller liest als ich.

Ich belasse es besser dabei.

\letterClosing[In liebe]{Euer Sohn, Harry James Potter-Evans-Verres.}
\end{writtenNote}
