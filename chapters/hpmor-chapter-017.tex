\chapter{Die bevorzugte Hypothese}

\emph{„Man beginnt, Muster in der Welt zu sehen, ihren Rhythmus zu hören.“} 

\later

\lettrine{D}{onnerstag}.

\hplettrineextrapara
Um genau zu sein, Donnerstag Morgen um 7:24~Uhr. 

Harry saß auf seinem Bett, ein Lehrbuch starr in seinen unbewegten Händen. 

Harry war gerade ein \emph{wirklich brillantes} Experiment eingefallen. 

Er würde eine Stunde länger bis zum Frühstück warten müssen, aber dafür hatte er Müsliriegel. Nein, diese Idee musste unbestreitbar, zweifelsohne auf der Stelle ausprobiert werden, jetzt, sofort. 

Harry legte das Lehrbuch beiseite, sprang aus dem Bett, rannte ums Bett, zerrte den Eingang zu seinem Koffer auf, eilte die Treppe herunter und schob Kisten mit Büchern herum. (Irgendwann müsste er die wirklich mal auspacken und Bücherschränke kaufen, aber im Moment lag er im Lesewettstreit mit Hermine zurück und hatte noch keine Zeit gehabt.) 

Harry fand das Buch, das er gesucht hatte, und rannte die Treppe wieder hoch. 

Die anderen Jungen machten sich gerade fertig, um zum Frühstück in die Große Halle zu gehen und den Tag zu beginnen. 

„Entschuldige, kannst du mal was für mich tun?“, sagte Harry. Während er sprach, überflog er das Inhaltsverzeichnis des Buches, fand die Seite mit den ersten zehntausend Primzahlen, blätterte zu der Seite und drückte Anthony Goldstein das Buch in die Hand. „Wähle zwei dreistellige Zahlen aus der Liste. Sag mir nicht welche. Multipliziere sie einfach und sag mir das Ergebnis. Ach, und könntest du das zur Sicherheit nochmal nachrechnen? Pass bitte wirklich auf, dass du die richtige Zahl rausbekommst, ich weiß nämlich nicht, was mit mir oder mit dem Universum geschieht, falls du einen Rechenfehler machst.“ 

Es sagte viel über das Geschehen der letzten Tage in diesem Schlafsaal, dass Anthony nicht einmal Fragen stellte wie „Warum bist du plötzlich ausgeflippt?“ oder „Das klingt wirklich verrückt, warum fragst du nach sowas?“ oder „Was soll das heißen, du weißt nicht, was mit dem Universum geschehen wird?“ 

Anthony nahm das Buch wortlos entgegen und holte Pergament und Schreibfeder. Harry drehte sich um, schloss die Augen, damit er wirklich nichts sah, und trat ungeduldig von einem Bein aufs andere. Er holte einen Block Papier und einen Drehbleistift und machte sich bereit zum Schreiben. 

„Okay“, sagte Anthony. „Einhunderteinundachtzigtausendvierhundertundneunundzwanzig.“

Harry schrieb 181.429 auf. Er wiederholte, was er gerade aufgeschrieben hatte und Anthony bestätigte es. 

Dann eilte Harry zurück in das Kellerabteil seines Koffers, sah auf die Uhr (sie zeigte 4:28~Uhr an, es war also 7:28~Uhr) und schloss dann die Augen. 

Etwa dreißig Sekunden später hörte Harry Schritte, gefolgt von dem Geräusch des sich schließenden Eingangs zum Kellerabteil. (Harry machte sich keine Sorgen, dass er ersticken könnte. Ein eingebauter Luftaufbereitungszauber war im Lieferumfang enthalten, wenn man einen wirklich guten Koffer kaufte. War Zauberei nicht wundervoll, sie brauchte sich keine Gedanken um die Stromrechnung machen.) 

Als Harry die Augen wieder öffnete, sah er genau was er gehofft hatte: ein gefaltetes Stück Papier auf dem Boden, das Geschenk von sich selbst aus der Zukunft. 

Sei dieses Stück Papier „Zettel 2“. 

Harry riss ein Stück Papier vom Notizblock ab. 

Sei dies „Zettel 1“. Es war natürlich das selbe Stück Papier. Wenn man genau hinsah, konnte man sogar erkennen, dass die abgerissene Kante genau gleich aussah. 

Harry ging in seinem Kopf den Algorithmus durch, den er befolgen würde. 

Wenn Harry Zettel 2 entfaltete und dieser leer war, dann würde er „101~$\times$~101“ auf Zettel 1 schreiben, diesen zusammenfalten, eine Stunde lang lernen, dann eine Stunde zurück reisen, Zettel 1 auf dem Boden hinterlassen (wodurch dieser zu Zettel 2 würde) und dann das Kellerabteil des Koffers verlassen und seinen Mitbewohner zum Frühstück folgen. 

Wenn Harry Zettel 2 entfaltete und zwei Zahlen darauf standen, würde Harry diese Zahlen multiplizieren. 

Wenn ihr Produkt 181.429 betrug, würde Harry diese beiden Zahlen auf Zettel 1 schreiben und mit Zettel 1 eine Stunde zurück reisen. 

Anderenfalls würde Harry zur rechten Zahl zwei hinzuaddieren und das neue Zahlenpaar auf Zettel 1 schreiben – es sei denn, die rechte Zahl würde dadurch größer als 997; in diesem Fall würde Harry zur linken Zahl zwei hinzuaddieren und rechts 101 hinschreiben. 

Und wenn auf Zettel 2 „997~$\times$~997“ stand, dann würde Harry Zettel 1 leer lassen. 

Das bedeutete, dass die einzige mögliche \emph{stabile} Zeitschleife jene war, in der Zettel 2 die beiden Primfaktoren von 181.429 enthielt. 

Wenn dies funktionierte, dann könnte Harry jede Aufgabe lösen, deren Antwort leicht zu prüfen, aber schwer herauszufinden war. Er hätte nicht \emph{nur} gezeigt, dass $\mbox{P}=\mbox{NP}$ galt, wenn man einen Zeitumkehrer hatte; dieser Trick war \emph{noch allgemeiner} anwendbar. Harry konnte damit jedes Zahlenschloss oder kompliziertere Passwort knacken. Vielleicht sogar den Eingang zu Slytherins Kammer des Schreckens finden, wenn Harry eine systematische Beschreibung aller Orte in Hogwarte einfiel. Selbst für Harrys Verhältnisse war dies ein unglaublicher Trick. 

Harry hob Zettel 2 mit zitternden Händen auf und entfaltete ihn. 

Auf Zettel 2 stand in etwas zittriger Handschrift: 

\shout{Lege dich nicht mit der Zeit an}

Harry schrieb „\shout{Lege dich nicht mit der Zeit an}“ in etwas zittriger Handschrift auf Zettel 1, faltete ihn sauber zusammen und beschloss, nicht weiter mit Zeit herumzuexperimentieren, bis er mindestens fünfzehn war. 

Soweit es Harry bekannt war, war dies das gruseligste Experiment in der gesamten Geschichte der Wissenschaft gewesen. 

Während der nächsten Stunde fiel es Harry nicht ganz leicht, sich auf sein Lehrbuch zu konzentrieren. 

So begann der Donnerstag für Harry. 

\later 

Donnerstag. 

Um genau zu sein, Donnerstag Nachmittag um 15:32~Uhr. 

Harry und alle anderen Jungen im ersten Schuljahr waren mit Madam Hooch draußen auf einer Wiese und standen neben den Schulbesen. Die Mädchen lernten in einer eigenen Gruppe fliegen. Anscheinend wollten Mädchen aus irgendeinem Grund nicht in Anwesenheit von Jungen auf Besen fliegen lernen. 

Harry war schon den ganzen Tag lang etwas unruhig gewesen. Er konnte nicht aufhören darüber nachzudenken, wie ausgerechnet \emph{diese} stabile Zeitschleife ausgewählt wurde – im Nachhinein betrachtet gab es doch eine Vielzahl von Möglichkeiten. 

Außerdem: \emph{Besen?} Tatsächlich? Er sollte also, im Grunde genommen, auf einer geraden Linie fliegen? War das nicht so ziemlich die instabilste Form, die man sich vorstellen konnte – wenn man nicht gerade versuchen wollte, sich an einem einzigen Punkt festzuhalten? Wer hatte sich den ausgerechnet \emph{diesen} Entwurf für ein Fluggerät ausgedacht? Harry hatte gehofft, dass es nur eine Redensart war, aber nein, er stand tatsächlich vor etwas, das wie ein haushaltsüblicher hölzerner Besen aussah. War jemand einfach auf diese Idee gekommen und hatte dann gar nichts anderes mehr in Betracht gezogen? So musste es gewesen sein. Wenn man unvoreingenommen darüber nachdachte, war es vollkommen undenkbar, dass die \emph{optimalen} Werkzeuge für’s Fegen und für’s Fliegen zufällig identisch wären. 

Es war ein schöner Tag mit hellblauem Himmel und einer strahlenden Sonne, die nur zu gerne offene Augen blendete und das Sehen unmöglich machte, wenn man versuchte, bei diesen Bedingungen zu fliegen. Der Boden war vollkommen trocken, roch nach gebrannter Erde und fühlte sich unter Harrys Füßen sehr, sehr hart an. 

Harry erinnerte sich daran, dass alle Elfjährigen das lernten, es konnte also nicht so schwer sein. 

„Streckt die rechte Hand über euren Besen aus – oder die linke, wenn ihr Linkshänder seid“, rief Madam Hooch, „und sagt ‚\shout{Hoch!}‘“ 

„\shout{Hoch!}“, riefen alle. 

Der Besen sprang eifrig in Harrys Hand. 

Damit gehörte er dieses eine Mal zu den Besten der Klasse. „\shout{Hoch!}“ zu sagen war offenbar sehr viel schwieriger als es schien und so wanden die meisten Besen sich auf dem Boden oder versuchten, von ihren Möchtegern-Reitern wegzurollen. 

(Harry hätte natürlich Geld darauf verwettet, dass Hermine mindestens ebensogut gewesen war, als sie es am Vormittag probiert hatte. Nichts, was \emph{er} beim ersten Versuch schaffte, konnte für Hermine schwer sein; und wenn es doch so etwas gab und es \emph{auf einem Besen fliegen} war, statt irgendetwas Intellektuellem, dann würde Harry vor Scham sterben.) 

Es dauerte eine Weile, bis jeder einen Besen in der Hand hielt. Madam Hooch zeigte ihnen, wie man aufsteigt, und ging dann durch die Reihen, um die Körperhaltung und den Griff zu überprüfen. Selbst die Kinder, die daheim schon fliegen durften, hatten es offenbar nicht richtig beigebracht bekommen. 

Madam Hooch blickte die Jungen an und nickte. „Wenn ich jetzt pfeife, dann stoßt ihr euch fest vom Boden ab.“ 

Harry schluckte und versuchte, das mulmige Gefühl in seinem Bauch zu verdrängen. 

„Haltet die Besen gut fest, steigt ein paar Meter hoch und kommt dann sofort wieder runter, indem ihr euch leicht nach vorn beugt. Auf meinen Pfiff! Drei, zwei, –“ 

Einer der Besen schoss empor und ein Junge schrie – vor Schrecken, nicht vor Freude. Der Junge drehte sich beängstigend schnell, während er hochflog; sein weißes Gesicht war nur Sekundenbruchteile zu sehen – 

Wie in Zeitlupe sprang Harry von seinem eigenen Besen und griff hektisch nach seinem Zauberstab, obwohl er nicht genau wusste, was er damit vorhatte, er hatte erst zwei Unterrichtsstunden Zauberkunst gehabt und in der zweiten \emph{hatten} sie den Schwebezauber gelernt, aber Harry hatte ihn bei drei Versuchen nur einmal erfolgreich gesprochen und er konnte sicherlich keinen ganzen Menschen schweben lassen – 

\emph{Wenn in mir irgendwelche geheimen Kräfte stecken – zeigt euch JETZT!} 

„Komm zurück, Junge!“, rief Madam Hooch. (Das war wohl die nutzloseste nur denkbare Empfehlung zum Umgang mit einem außer Kontrolle geratenen Besen – und das von einer \emph{Fluglehrerin.} Ein Teil von Harrys Gehirn setzte Madam Hooch auf die Liste der Dummköpfe.) 

Und der Junge fiel vom Besen. 

Er schien sich erst nur sehr langsam durch die Luft zu bewegen. 

\emph{„Wingardium Leviosa!“}, schrie Harry. 

Der Zauberspruch schlug fehl. Er konnte es spüren. 

Ein dumpfes WUMM, ein leises Knacken und der Junge lag zusammengekrümmt auf dem Boden, das Gesicht im Gras. 

Harry steckte seinen Zauberstab weg und rannte so schnell er konnte. Er kam gleichzeitig mit Madam Hooch beim Jungen an und Harry griff in seinen Beutel und versuchte, sich zu erinnern, ohje, wie hieß es bloß, egal, er würde einfach „Erste-Hilfe-Kasten!“ versuchen und es erschien in seiner Hand und – 

„Gebrochenes Handgelenk“, sagte Madam Hooch. „Beruhige dich, Junge, es ist nur ein gebrochenes Handgelenk!“ 

Mit einer Art mentalem Ruck verließ Harrys Gehirn den Panikmodus. 

Das Notheilpaket Plus lag offen vor ihm auf dem Boden und in seiner Hand war eine Spritze mit flüssigem Feuer, welches das Gehirn des Jungen mit Sauerstoff versorgt hätte, falls dieser sich das Genick gebrochen hätte. 

„Äh…“, sagte Harry in einer etwas flatternden Stimme. Sein Herz schlug so laut, dass er kaum hören konnte, wie er selbst nach Luft japste. „Gebrochener Knochen … okay … Gips?“ 

„Das ist nur für Notfälle“, herrschte Madam Hooch ihn an. „Pack es weg, ihm geht’s gut.“ Sie beugte sich über den Jungen und hielt ihm die Hand hin. „Komm hoch, Junge, es ist alles gut, steh’ schon auf!“ 

„Sie wollen ihn doch nicht tatsächlich jetzt wieder auf einen Besen lassen?“, fragte Harry schockiert. 

Madam Hooch sah Harry empört an. „Natürlich nicht!“ Sie zog den Jungen an seinem gesunden Arm hoch – Harry stellte erschrocken fest, dass es \emph{schon wieder} Neville Longbottom war; was war mit ihm bloß los? – und wandte sich den zusehenden Schülern zu. „Niemand von euch bewegt sich, während ich diesen Jungen in den Krankenflügel bringe! Ihr lasst die Besen wo sie sind oder ihr müsst Hogwarts schneller verlassen als ihr ‚Quidditch’ sagen könnt. Komm mit, mein Kleiner.“ 

Und Neville, der sein Handgelenk festhielt und versuchte, sein Schluchzen zu unterdrücken, ging mit Madam Hooch davon. 

Sobald sie außer Hörweite waren, fing ein Slytherin an zu kichern. 

Die Anderen steckten sich an. 

Harry drehte sich um und sah sie an. Das war wohl eine gute Gelegenheit, sich einige Gesichter einzuprägen. 

Und Harry sah, wie Draco auf ihn zukam, begleitet von Mr Crabbe und Mr Goyle. Mr Crabbe grinste nicht. Mr Goyle hingegen sehr deutlich. Draco selbst trug einen sehr neutralen Gesichtsausdruck, der gelegentlich erzitterte. Harry folgerte daraus, dass Draco die Ereignisse höchst amüsant fand, aber keinen politischen Vorteil erwartete, wenn er jetzt in Gelächter ausbrach, statt später in den Slytherin-Kerkern. 

„Nun, Potter“, sagte Draco in einer ruhigen Stimme, immer noch mit dem erzitternden, neutralen Gesichtsausdruck, „ich wollte dir nur sagen, wenn du eine Notsituation ausnutzen willst, um Führungsstärke zu zeigen, dann möchtest du den Eindruck erwecken, dass du die Situation vollkommen unter Kontrolle hast – und nicht, dass du in Panik verfällst.“ Mr Goyle kicherte, aber unterdrückte das nach einem Blick von Draco. „Aber du hast vermutlich trotzdem ein paar Punkte gesammelt. Brauchst du Hilfe beim Zusammenpacken der Heilerausrüstung?“ 

Harry sah zum Notheilpaket und konnte sein Gesicht so von Draco abwenden. „Ich schaff’s schon so“, sagte Harry. Er legt die Spritze wieder an ihren Platz zurück, schnallte sie fest und stand auf. 

Ernie Macmillan kam an, als Harry seinen Beutel gerade mit dem Paket fütterte. 

„Im Namen von Hufflepuff – vielen Dank, Harry Potter“, sagte Ernie Macmillan gewichtig. „Es war ein guter Versuch und eine gute Idee.“ 

„Eine gute Idee, in der Tat“, sagte Draco. „Warum hatte kein anderer Hufflepuff den Zauberstab in der Hand? Wenn ihr \emph{alle} geholfen hättet, statt nur Potter, hättet ihr ihn auffangen können. Ich dachte, Hufflepuffs sollten zueinander halten?“ 

Ernie sah aus, als sei er zwischen Empörung und Scham zwiegespalten. „Wir haben nicht rechtzeitig daran gedacht –“ 

„Ach“, sagte Draco, „ihr habt nicht daran \emph{gedacht.} Ich nehme an, deswegen ist es besser, einen einzigen Ravenclaw zum Freund zu haben, als ganz Hufflepuff.“ 

Ach verdammt, wie sollte Harry das wohl geradebiegen? „Das hilft uns nicht weiter“, sagte Harry in einem sanften Tonfall. Hoffentlich verstand Draco das als \emph{du durchkreuzt meine Pläne, sei bloß still.} 

„Hey, was ist das?“, sagt Mr Goyle. Er kniete im Gras nieder und hob eine größere Murmel auf; einen Glasball, der mit verwirbeltem weißem Nebel gefüllt schien. 

Ernie blinzelte. „Nevilles Erinnermich!“ 

„Was ist ein Erinnermich?“, fragte Harry. 

„Es färbt sich rot, wenn du etwas vergessen hast“, sagte Ernie. „Aber es verrät dir nicht, was du vergessen hast. Gib es bitte her, dann werde ich’s Neville nachher zurückgeben.“ Ernie streckte die Hand aus. 

Plötzlich erschien ein Grinsen auf Mr Goyles Gesicht, er drehte sich um und rannte weg. 

Ernie blieb einen Moment lang überrascht stehen, rief dann „Hey!“ und rannte Mr Goyle hinterher. 

Und Mr Goyle griff zu einem Besen, sprang mit einer flüssigen Bewegung auf und flog empor. 

Harrys Mund stand offen. Hatte Madam Hooch nicht gesagt, dass er dafür \emph{von der Schule fliegen} würde? 

\emph{„Dieser Idiot!“}, zischte Draco. Er öffnete den Mund, um zu rufen – 

\emph{„Hey!“}, schrie Ernie. „Das gehört Neville! \emph{Gib es zurück!}“ 

Die Slytherins begannen zu rufen und zu jubeln. 

Dracos Mund klappte zu. Harry entdeckte plötzlich Verunsicherung in seinem Gesicht. 

„Draco“, sagte Harry leise, „wenn du diesen Idioten nicht zurück auf den Boden holst, wird die Lehrerin zurückkommen und –“ 

\emph{„Komm doch und hol es dir, du Hufflepuffle!“}, schrie Mr Goyle und unter den Slytherins brauste erneut Jubel auf. 

„Ich \emph{kann nicht!“}, flüsterte Draco. „Alle Slytherins würden mich für \emph{schwach} halten!“ 

„Und wenn Mr Goyle rausfliegt“, zischte Harry, „wird \emph{dein Vater} dich für einen \emph{Idioten} halten!“ 

Draco verzog gequält das Gesicht. 

Im gleichen Moment – 

„Hey, du \emph{schleimiger Slytherin}“, schrie Ernie, „hat dir noch niemand erzählt, dass Hufflepuffs zusammenhalten? \emph{Zauberstäbe raus, Hufflepuff!}“ 

Plötzlich zeigten ziemlich viele Zauberstäbe auf Mr Goyle. 

Drei Sekunden später – 

\emph{„Zauberstäbe raus, Slytherin!“,} riefen etwa fünf verschiedene Slytherins. 

Und ziemlich viele Zauberstäbe zeigten auf die Hufflepuffs. 

Zwei Sekunden später – 

\emph{„Zauberstäbe raus, Gryffindor!“} 

\emph{„Mach etwas, Potter!“}, flüsterte Draco. \emph{„Ich kann das nicht aufhalten, das musst du machen! Du hast dann was gut bei mir, denk dir irgendwas aus, du bist doch angeblich so brillant!“} 

In ungefähr fünfeinhalb Sekunden, ahnte Harry, würde irgendwer den Simplen Sumerischen Schlagzauber sprechen und wenn alles vorbei wäre und die Lehrer alle Beteiligten der Schule verwiesen hätten, dann würde der Jahrgang nur noch aus den Ravenclaws bestehen. 

\shout{„Zauberstäbe raus, Ravenclaw!“}, schrie Michael Corner, der offenbar auch an diesem Desaster beteiligt sein wollte. 

„\scream{Gregory Goyle}!“, schrie Harry. \shout{„Ich fordere dich zu einem Wettstreit um den Besitz von Nevilles Erinnermich heraus!“} 

Plötzlich war es still. 

„Ach, tatsächlich?“, sagte Draco besonders laut und affektiert. „Das klingt interessant. Was für ein Wettstreit, Potter?“ 

Ähm … 

Weiter als „Wettstreit“ hatte Harry noch nicht gedacht. Was für ein Wettstreit … er konnte nicht „Schach“ sagten, da es seltsam aussehen würde, wenn Draco das akzeptierte; er konnte nicht „Armdrücken“ sagen, weil Mr Goyle ihn darin schlagen würde – 

„Wie wär’s hiermit?“, sagte Harry laut. „Gregory Goyle und ich stehen in einiger Entfernung voneinander und niemand darf sich einem von uns nähern. Wir benutzen keine Zauberstäbe und ihr alle auch nicht. Ich bewege mich nicht von der Stelle und er auch nicht. Und wenn ich an Nevilles Erinnermich komme, dann verzichtet Gregory Goyle auf das Erinnermich, das er in der Hand hält, und gibt es mir.“ 

Es war abermals still, während die Erleichterung auf den Gesichtern durch Verwirrung abgelöst wurde. 

„Ha, Potter!“, sagte Draco laut. „Ich möchte sehen, wie du \emph{das} schaffst! Mr Goyle ist einverstanden!“ 

„Los geht’s!“, sagte Harry. 

„Potter, \emph{was zum Teufel?}“, flüsterte Draco, ohne die Lippen zu bewegen. 

Harry konnte ihm nicht antworten, ohne seine Lippen zu bewegen. 

Die Schüler steckten ihre Zauberstäbe ein und Mr Goyle landete elegant am Boden, sah jedoch recht verwirrt aus. Einige Hufflepuffs machten Anstalten, auf Mr Goyle loszugehen, doch ein flehender Blick von Harry genügte, um sie aufzuhalten. 

Harry ging auf Mr Goyle zu und blieb einige Meter entfernt stehen; weit genug, dass sie einander nicht erreichen konnten. 

Langsam und betont steckte Harry seinen Zauberstab ein. 

Alle anderen hielten Abstand. 

Harry schluckte. Er wusste ungefähr, was er tun \emph{wollte,} aber er musste es so tun, dass niemand verstehen würde, \emph{was} er getan hatte – 

„Also gut“, sagte Harry laut. „Und nun …“ Er atmete tief durch und erhob eine Hand, die Finger bereit zum Schnippen. Alle, die die Geschichte mit den Kuchen gehört hatten – also fast alle Anwesenden – schnappten nach Luft. \emph{„Ich rufe den Wahnsinn von Hogwarts zur Hilfe! Fröhlich, fröhlich, bumm bumm, Sumpf, Sumpf, Sumpf!“} Und Harry schnippte mit den Fingern. 

Viele Schüler zuckten zusammen. 

Und nichts geschah. 

Harry ließ die Stille etwas andauern, wartete ab, bis … 

„Ähm“, sagte jemand. „Das war’s?“ 

Harry sah den Jungen an, der gesprochen hatte. „Schau vor dich. Siehst du den dunklen Fleck am Boden, wo kein Gras ist?“ 

„Ähm, ja“, sagte der Junge, ein Gryffindor. (Dean irgendwas?) 

„Grabe dort.“ 

Jetzt erntete Harry viele verwirrte Blicke. 

„Ähm, warum?“, sagte Dean irgendwas. 

„Mach’s einfach“, sagte Terry Boot mit schwacher Stimme. „Nachfragen bringt nichts, glaub’ mir.“ 

Dean irgendwas kniete nieder und begann, die Erde wegzukratzen. 

Nach etwa einer Minute stand Dean wieder auf. „Da ist nichts“, sagte Dean. 

Hm. Harry hatte eigentlich vorgehabt, in der Zeit zurückzureisen und eine Schatzkarte zu vergraben, die zu einer weiteren Schatzkarte führte, die zu Nevilles Erinnermich führte, welches er in das Versteck tun würde, sobald er es von Mr Goyle zurück hatte … 

Dann fiel Harry ein, dass es eine einfachere Möglichkeit gab, die das Geheimnis des Zeitumkehrers nicht so sehr gefährden würde. 

„Danke, Dean!“, sagte Harry laut. „Ernie, kannst du an der Stelle, wo Neville hingefallen ist, mal schauen, ob du Nevilles Erinnermich am Boden findest?“ 

Die Leute sahen noch verwirrter aus. 

„Mach’s einfach“, sagte Terry Boot. „Er wird es solange versuchen, bis irgendwas funktioniert. Und das Beängstigende daran ist –“ 

\emph{„Bei Merlin!“}, entglitt es Ernie. Er hielt Nevilles Erinnermich hoch. „\emph{Hier} ist es! Genau wo er hingefallen ist!“ 

\emph{„Was?“}, rief Mr Goyle. Er blickte nieder und sah … 

… dass er immer noch Nevilles Erinnermich in der Hand hielt. 

Lange Zeit war es still. 

„Ähm“, sagte Dean irgendwas, „das ist nicht möglich, oder?“ 

„Es ist ein Plotloch“, sagte Harry. „Ich habe mich verrückt genug gemacht, dass das Universum einen Moment lang abgelenkt war und vergessen hatte, dass Goyle das Erinnermich bereits aufgehoben hatte.“ 

„Nein, Moment mal, ich meine, das ist \emph{vollkommen} unmöglich –“ 

„Entschuldigung, aber stehen wir hier alle gerade rum und warten darauf, dass wir auf Besen fliegen? Ja, tun wir. Also halt’ den Mund. Wie dem auch sei, sobald ich Nevilles Erinnermich in der Hand halte, ist der Wettstreit vorbei und Gregory Goyle muss mir das Erinnermich geben, das er in der Hand hält. So lauteten die Bedingungen, denkt dran.“ Harry streckte die Hand aus und nickte Ernie zu. „Da mir niemand zu nahe kommen soll, rolle es mir doch einfach zu, okay?“ 

„Warte mal!“, rief ein Slytherin – Blaise Zabini, diesen Namen würde Harry wohl nicht so schnell vergessen. „Woher wissen wir, dass das Nevilles Erinnermich ist? Du hättest irgendein \emph{anderes} Erinnermich dort hinlegen können –“ 

„Ein wahrer Slytherin“, sagte Harry lächelnd. „Aber du hast mein Wort, dass das Erinnermich in Ernies Hand Neville gehört. Was das in Gregory Goyles Hand angeht – kein Kommentar.“ 

Zabini drehte sich zu Draco. „\emph{Malfoy!} Du wirst doch nicht zulassen, dass er damit durchkommt –“ 

„Schweig, du“, polterte Mr Crabbe, der hinter Draco stand. „\emph{Du} hast Mister Malfoy nicht zu sagen, was er tun soll!“ 

\emph{Guter} Lakai. 

„Ich habe eine Wette mit Draco vom führnehmen und gar alten Haus Malfoy abgeschlossen“, sagte Harry. „Nicht mit dir, Zabini. Ich habe getan, was Mr Malfoy von mir verlangt hat und überlasse das endgültige Urteil nun Mr Malfoy.“ Harry nickte Draco zu und hob die Augenbrauen. Das sollte ausreichen, um Draco das Gesicht zu wahren. 

Es war kurz still. 

„Du versprichst, dass das \emph{tatsächlich} Nevilles Erinnermich ist?“, sagte Draco. 

„Ja“, sagte Harry. „Dieses bekommt Neville zurück; es war von Anfang an seines. Und dasjenige in Gregory Goyles Hand bekomme ich.“ 

Draco nickte und sah entschlossen aus. „Ich werde am Wort des führnehmen Hauses Potter nicht zweifeln, so seltsam das alles auch war. Und das führnehme und gar alte Haus Malfoy hält sein Wort ebenso. Mr Goyle, gib das Mr~Potter –“ 

„Hey!“, sagte Zabini. „Er hat \emph{noch} nicht gewonnen, er hat das Erinnermich noch –“ 

„Fang, Harry!“, sagte Ernie und warf das Erinnermich. 

Harry fing es mühelos, er hatte schon immer gute Reflexe gehabt. „So“, sagte Harry, „ich habe gewonnen …“ 

Harry sprach nicht weiter. Alle Gespräche brachen ab. 

Das Erinnermich in seiner Hand leuchtete in kräftigem Rot, strahlte wie eine winzige Sonne, die im hellen Tageslicht Schatten auf den Boden warf. 

\later 

Donnerstag. 

Um genau zu sein, Donnerstag Nachmittag um 17:09~Uhr nach der Flugstunde in Professor McGonagalls Büro. (Nachdem Harry eine zusätzliche Stunde verbracht hatte.) 

Professor McGonagall saß hinter ihrem Schreibtisch, Harry vor dem Schreibtisch auf der Anklagebank. 

„Professor“, sagte Harry brüsk, „die Slytherins hatten ihre Zauberstäbe auf die Hufflepuffs gerichtet, die Gryffindors hatten ihre Zauberstäbe auf die Slytherins gerichtet, irgendein \emph{Idiot} schrie, dass die Ravenclaws ihre Zauberstäbe rausholen sollten, und mir blieben ungefähr fünf Sekunden, bis die ganze Sache explodiert wäre! Was Besseres ist mir nicht eingefallen!“ 

Professor McGonagalls Gesicht war verkniffen und wütend. „\emph{Sie dürfen den Zeitumkehrer nicht für so etwas benutzen, Mr~Potter!} Haben Sie denn keine Ahnung, was Geheimhaltung bedeutet?“ 

„Die \emph{wissen} nicht, wie ich es geschafft habe! Sie denken einfach, dass ich seltsame Sachen tun kann, indem ich mit den Fingern schnippe! Ich habe schon andere seltsame Sachen gemacht, die nicht mal mit einem Zeitumkehrer möglich sind, und ich werde noch \emph{mehr} solches Zeug machen. \emph{Dieses} eine Mal wird dann niemandem mehr auffallen! Ich \emph{musste es tun,} Professor!“ 

„Sie mussten das \emph{nicht} tun!“, schimpfte Professor McGonagall. „Sie hätten nur \emph{einen Slytherin} wieder zurück auf den Boden holen und dann dafür sorgen müssen, dass die Zauberstäbe weggesteckt werden. Sie hätten ihn zu einer Runde Zauberschnippschnapp herausfordern können, aber nein, sie mussten den Zeitumkehrer auf eine schamlose und völlig unnötige Weise verwenden!“ 

„Mir ist nichts Besseres eingefallen! Ich weiß nicht mal, was Zauberschnippschnapp \emph{ist,} bei Schach hätten die nicht mitgemacht und Armdrücken hätte ich verloren!“ 

\emph{„Dann hätten Sie Armdrücken nehmen sollen!“} 

Harry stutzte. „Aber dann hätte ich \emph{verloren} –“ 

Harry brach ab. 

Professor McGonagall sah \emph{sehr} wütend aus. 

„Es tut mir Leid, Professor McGonagall“, sagte Harry leise. „Ich habe wirklich nicht daran gedacht, aber sie haben Recht, ich hätte das tun sollen, es wäre großartig gewesen, wenn ich das getan hätte, aber ich habe wirklich nicht mehr daran gedacht …“ 

Harry verstummte. Plötzlich wurde ihm klar, dass er \emph{viele} andere Möglichkeiten gehabt hätte. Er hätte \emph{Draco} um einen Vorschlag bitten können, er hätte die Mitschüler fragen können … dass er den Zeitumkehrer verwendet hatte, war \emph{tatsächlich} völlig schamlos und unnötig gewesen. Er hätte so viele andere Möglichkeiten gehabt; warum hatte er bloß \emph{diese} gewählt? 

Weil er \emph{gewinnen} konnte. Einen unwichtigen Gegenstand gewinnen, den die Lehrer Mr Goyle ohnehin wieder weggenommen hätten. 

Die Absicht, zu gewinnen. Daran war er gescheitert. 

„Es tut mir Leid“, sagte Harry wieder. „Dass ich so stolz und dumm war.“ 

Professor McGonagall wischte sich mit der Hand über die Stirn. Ein Teil ihres Ärgers schien zu verfliegen. Doch ihre Stimme klang immer noch eisig. „Noch eine solche Vorführung, Mr~Potter, und Sie werden den Zeitumkehrer zurückgeben. Habe ich mich klar ausgedrückt?“ 

„Ja“, sagte Harry. „Ich habe es verstanden und es tut mir Leid.“ 

„Dann, Mr~Potter, werden Sie den Zeitumkehrer bis auf Weiteres behalten können. Und angesichts des Debakels, was sie zugegebenermaßen vermieden haben, werde ich Ravenclaw keine Punkte abziehen.“ 

\emph{Und Sie hätten niemandem erklären können, warum Sie mir die Punkte abgezogen hätten.} Aber Harry war nicht so dumm, das laut auszusprechen. 

„Eine wichtige Frage noch“, sagte Harry, „warum hat das Erinnermich so geleuchtet? Bedeutet das, dass ich einen Vergessenszauber abbekommen habe?“ 

„Das verwundert mich auch“, sagte Professor McGonagall langsam. „Wenn es so einfach wäre, würde man die Erinnermiche wohl vor Gericht verwenden, aber das tut man nicht. Ich werde das prüfen, Mr~Potter.“ Sie seufzte. „Sie können jetzt gehen.“ 

Harry begann sich aus dem Stuhl zu erheben, setzte sich dann wieder. „Ähm, es tut mir Leid, ich wollte Ihnen noch etwas Anderes erzählen –“ 

Man konnte das Zusammenzucken kaum sehen. „Was gibt es, Mr~Potter?“ 

„Es geht um Professor Quirrell –“ 

„Ich bin mir sicher, Mr~Potter, dass es nichts Wichtiges ist.“ Professor McGonagall sprach nun sehr schnell. „Sicherlich haben Sie gehört, dass der Schulleiter allen Schülern gesagt hat, sie sollen uns nicht mit irgendwelchen unwichtigen Beschwerden über den Verteidigungslehrer stören?“ 

Harry war verwirrt. „Aber das \emph{könnte} wichtig sein; gestern habe ich dieses bedrückende Gefühl verspürt, als –“ 

„Mr~Potter! Ich verspüre auch ein bedrückendes Gefühl! Und dieses bedrückende Gefühl sagt mir, dass \emph{Sie kein weiteres Wort sagen sollten!}“ 

Harrys Mund stand offen. Professor McGonagall war es gelungen: Harry war sprachlos. 

„Mr~Potter“, sagte Professor McGonagall, „wenn Sie irgendetwas über Professor Quirrell herausgefunden haben, was möglicherweise von Belang sein könnte, dann seien Sie so nett es weder mir noch irgendwem anders mitzuteilen. Und ich denke, Sie haben nun genug meiner wertvollen Zeit in Anspruch –“ 

\emph{„Was ist mit Ihnen los?“}, entfuhr es Harry. „Es tut mir Leid, aber das erscheint mir so \emph{unvorstellbar} unverantwortlich! Ich habe gehört, dass die Stelle des Verteidigungslehrers verflucht ist, und wenn Sie schon \emph{wissen,} dass irgendetwas Schlimmes passieren wird, dann sollten Sie doch eigentlich auf der Hut –“ 

„Etwas \emph{Schlimmes,} Mr~Potter? \emph{Das glaube ich kaum.}“ Professor McGonagalls Gesicht war regungslos. „Nachdem Professor Blake im Februar mit nicht weniger als drei Slytherins aus dem fünften Schuljahr in einem Wandschrank erwischt wurde, und Professor Summers im Jahr zuvor so vollkommen versagt hat, dass ihre Schüler einen Irrwicht für ein Möbelstück hielten, wäre es \emph{katastrophal,} wenn ich nun von irgendeinem Problem mit dem außergewöhnlich kompetenten Professor Quirrell erfahren würde. Ich fürchte, dann würden die meisten Schüler ihre ZAGs und UTZs in Verteidigung nicht bestehen.“ 

„Ich verstehe“, sagte Harry langsam, während er das verarbeitete. „Also anders gesagt, egal was mit Professor Quirrell los ist, vor Ende des Schuljahres wollen Sie auf keinen Fall etwas davon wissen. Und da wir erst September haben, könnte er vor laufenden Kameras den Premierminister umbringen, ohne dass es Sie stören würde.“ 

Professor McGonagall sah ihn mit starrem Blick an. „Ich bin mir sicher, dass Sie nie hören werden, wie ich so einer Aussage zustimme, Mr~Potter. Hogwarts setzt sich aktiv gegen \emph{alles} ein, was die schulischen Leistungen unserer Schüler gefährdet.“ 

\emph{Zum Beispiel Ravenclaw-Erstklässler, die ihren Mund nicht halten können.} „Ich glaube, dass ich Sie vollkommen verstehe, Professor McGonagall.“ 

„Oh, das bezweifle ich, Mr~Potter. Das bezweifle ich sehr.“ Professor McGonagall beugte sich vor, ihr Gesicht wurde wieder strenger. „Da Sie und ich schon weit ernstere Dinge besprochen haben, will ich offen mit Ihnen sein. Sie, und nur Sie, haben von diesem mysteriösen bedrückenden Gefühl berichtet. Sie, und nur Sie, ziehen das Chaos so an, wie ich es noch nie zuvor gesehen habe. Nach unserer kleinen Shoppingtour in der Winkelgasse \emph{und} der Geschichte mit dem Sprechenden Hut \emph{und} der heutigen kleinen Angelegenheit, sehe ich es schon kommen, dass ich im Büro des Schulleiters sitzen werde und eine unglaubliche Geschichte über Professor Quirrell hören werde, in der Sie, und nur Sie, eine Hauptrolle spielen, woraufhin wir keine andere Wahl haben als ihn zu feuern. Ich habe mich schon damit abgefunden, Mr~Potter. Und wenn dieses traurige Ereignis vor den Iden des Mai stattfindet, dann werde ich Sie an ihren eigenen Eingeweiden an den Toren von Hogwarts aufhängen und Feuerkäfer in Ihre Nase kippen. Haben Sie mich \emph{jetzt} vollkommen verstanden?“ 

Harry nickte mit weit offenen Augen. Dann, eine Sekunde später: „Was bekomme ich, wenn ich es schaffe, dass es am letzten Tag des Schuljahres passiert?“ 

\emph{„Raus aus meinem Büro!“} 

\later 

Donnerstag. 

Es musste an den Donnerstagen in Hogwarts liegen. 

Es war Donnerstag Nachmittag, 17:23~Uhr, und Harry stand neben Professor Flitwick vor dem großen steinernen Wasserspeier, der den Eingang zum Büro des Schulleiters bewachte. 

Sobald er aus Professor McGonagalls Büro in die Lernräume der Ravenclaws zurückgekehrt war, hatte ein Schüler ihm gesagt, er solle sich in Professor Flitwicks Büro melden. Dort hatte Harry dann erfahren, dass Dumbledore ihn sprechen wollte. 

Harry, der recht nervös war, hatte Professor Flitwick gefragt, ob der Schulleiter gesagt habe, worum es gehe. 

Professor Flitwick hatte hilflos mit den Schultern gezuckt. 

Offenbar hatte Dumbledore gesagt, dass Harry viel zu jung war, um die Worte der Macht und des Wahnsinns auszusprechen. 

\emph{Fröhlich, fröhlich, bumm bumm, Sumpf, Sumpf, Sumpf?}, hatte Harry gedacht, aber nicht laut gesagt. 

„Sorgen Sie sich bitte nicht zu sehr, Mr. Potter“, quiekte Professor Flitwick ungefähr auf Schulterhöhe von Harry. (Harry war froh, dass Professor Flitwick einen großen bauschigen Bart hatte. Es war schwer, sich an einen Professor zu gewöhnen, der nicht nur kleiner als er war, sondern auch eine höhere Stimme hatte.) „Schulleiter Dumbledore mag ein wenig seltsam erscheinen, oder sehr seltsam, oder gar außerordentlich seltsam, aber er hat noch nie einem Schüler auch nur ein Haar gekrümmt und ich glaube nicht, dass er das jemals tun wird.“ Professor Flitwick lächelte Harry beruhigend zu. „Behalten Sie das jederzeit im Kopf, dann haben Sie keinen Grund, in Panik auszubrechen!“ 

Das machte es nicht besser. 

„Viel Glück!“, quiekte Professor Flitwick, beugte sich zum Wasserspeier und sagte etwas, das Harry aus irgendeinem Grund vollkommen überhörte. (Das Passwort würde natürlich nicht viel bringen, wenn man hören könnte, wie jemand es sagt.) Und der Wasserspeier trat mit einer sehr natürlichen und gewöhnlichen Bewegung beiseite, was Harry recht schockierend fand, da der Wasserspeier immer noch wie fester, unbeweglicher Stein aussah.

Hinter dem Wasserspeier befand sich eine langsam rotierende Wendeltreppe. Sie war verstörend hypnotisch – und noch verstörender war, dass eine rotierende Spirale sich nicht nach oben bewegen sollte. 

„Hoch mit Ihnen!“, quiekte Flitwick. 

Nervös setzte Harry einen Fuß auf die Treppe und wurde, aus irgendeinem Grund, den sein Gehirn absolut nicht nachvollziehen konnte, aufwärts bewegt. 

Hinter ihm kehrte der Wasserspeier an seinen Platz zurück und die Wendeltreppe drehte sich weiter und Harry befand sich in immer größerer Höhe. Nach einer recht verwirrenden Weile befand Harry sich vor einer eichenen Tür mit einem bronzenen Türklopfer in Gestalt eines Greifen. 

Harry griff nach dem Türknauf und drehte ihn. 

Die Tür schwang auf. 

Und Harry sah den interessantesten Raum, den er je in seinem Leben gesehen hatte. 

Er sah kleine metallene Gerätschaften, die surrten oder tickten, langsam die Form änderten oder kleine Rauchwölkchen ausstießen. Er sah dutzende mysteriöse Flüssigkeiten in dutzenden seltsam geformten Behältnissen; sie blubberten, kochten, sickerten, änderten die Farbe oder nahmen interessante Formen an, die sich eine halbe Sekunde nachdem man sie erblickt hatte auflösten. Er sah Dinge, die wie Uhren mit vielen Zeigern aussahen, beschriftet mit Zahlen oder in unidentifizierbaren Sprachen. Er sah einen Armreif, in den ein in tausend Farben funkelnder, linsenförmiger Kristall eingearbeitet war, und einen Vogel, der sich auf einer goldenen Platte niedergelassen hatte, und eine hölzerne Tasse, gefüllt mit etwas, das wie Blut aussah, und eine Statue eines Falken, die mit schwarzer Emaille überzogen war. Überall an der Wand hingen Bilder von schlafenden Menschen und der Sprechende Hut befand sich auf einem Hutständer, der auch zwei Regenschirmen und drei linken, roten Hausschuhen Platz bot. 

Inmitten all dieses Chaos befand sich ein sauberer, schwarzer, eichener Schreibtisch. Vor dem Tisch stand ein eichener Stuhl. Und hinter dem Tisch stand ein gut gepolsterter Thron, auf dem Albus Percival Wulfric Brian Dumbledore saß, mit seinem langen, silbernen Bart, einem Hut, der wie ein riesiger, eingedellter Champignon aussah, und in Kleidung, die für Muggelaugen wie drei Schichten kreischend pinker Pyjamas aussah. 

Dumbledore lächelte und seine hellen Augen funkelten wie verrückt. 

Beklommen setzte Harry sich auf den Stuhl vor dem Tisch. Die Tür hinter ihm ging mit einem lauten Knall zu. 

„Hallo, Harry“, sagte Dumbledore. 

„Hallo, Schulleiter“, antwortete Harry. Sie waren also schon per du? Würde Dumbledore ihm als nächstes anbieten – 

„Aber bitte, Harry!“, sagte Dumbledore. „Schulleiter klingt so formell. Nenn mich einfach Schu.“ 

„Das werde ich, Schu“, sagte Harry. 

Es war kurz still. 

„Wusstest du“, sagte Dumbledore, „dass du die erste Person bist, die dieses Angebot jemals angenommen hat?“ 

„Ähm …“, sagte Harry. Er versuchte, seine Stimme gleichmäßig zu halten, obwohl er plötzlich ein übles Gefühl im Bauch verspürte. „Es tut mir Leid, ich, äh, Schulleiter, Sie haben mich darum gebeten, also habe ich –“ 

„Schu, bitte!“, sagte Dumbledore fröhlich. „Und es gibt keinen Grund zur Panik, ich werde dich nicht aus dem Fenster werfen, bloß weil du einen Fehler machst. Wenn du etwas falsch machst, werde ich dich erst oft genug warnen! Außerdem ist es nicht so wichtig, wie die Menschen mit dir reden; wichtig ist, was sie von dir denken.“ 

\emph{Er hat noch nie einem Schüler auch nur ein Haar gekrümmt. Behalten Sie das jederzeit im Kopf, dann haben Sie keinen Grund, in Panik auszubrechen.} 

Dumbledore zog eine kleine Metalldose hervor, öffnete sie und eine kleine, gelbe Klumpen kamen zum Vorschein. „Zitronenbrausebonbon?“, sagte der Schulleiter. 

„Ähm, nein danke, Schu“, sagte Harry. \emph{Zählt es als} ein Haar krümmen, \emph{wenn man einem Schüler LSD anbietet, oder fällt das in die Kategorie harmloser Spaß?} „Sie … ähm … sagten, dass ich zu jung sei, um die Worte der Macht und des Wahnsinns auszusprechen?“ 

„Das bist du auf jeden Fall!“, sagte Dumbledore. „Glücklicherweise sind die Worte der Macht und des Wahnsinns vor sieben Jahrhunderten verschollen und niemand hat mehr die geringste Ahnung, wie sie lauten. Das war nur eine Randbemerkung.“ 

„Ähm …“, sagte Harry. Ihm fiel auf, dass sein Mund offen stand. „Warum haben Sie mich dann hergebeten?“ 

\emph{„Warum?“}, wiederholte Dumbledore. „Ach, Harry, wenn ich den ganzen Tag darüber nachdenken würde, \emph{warum} ich Dinge tue, dann hätte ich nie die Zeit, auch nur irgendetwas zu tun! Ich bin ein vielbeschäftigter Mensch, weißt du?“ 

Harry nickte lächelnd. „Ja, das war eine äußerst beeindruckende Liste. Schulleiter von Hogwarts, Großmeister des Zaubergamots und Ganz hohes Tier der Internationalen Vereinigung der Zauberer. Entschuldigen Sie die Frage, ich hab nur überlegt – ist es möglich, mehr als sechs Stunden zu bekommen, wenn man mehr als einen Zeitumkehrer benutzt? Denn es ist ziemlich beeindruckend, falls sie das in nur dreißig Stunden am Tag schaffen.“ 

Es war erneut einen Moment lang still und Harry lächelte weiterhin. Er war ein wenig nervös – sehr nervös sogar –, aber sobald ihm klar geworden war, dass Dumbledore ihn absichtlich verwirrte, hatte ein Teil von ihm sich \emph{vollkommen geweigert,} stumm dazusitzen und das einfach so über sich ergehen zu lassen. 

„Ich fürchte, Zeit mag nicht zu sehr gedehnt werden“, sagte Dumbledore nach einem kurzen Moment, „und dennoch scheinen wir etwas zu groß für sie zu sein. Es ist also ein ewiges Mühen, unser Leben in die Zeit reinzupressen.“ 

„In der Tat“, sagte Harry in ernstem und ehrwürdigem Ton. „Darum ist es am Besten, rasch das Wichtige anzusprechen.“ 

Einen Moment lang fragte Harry sich, ob er zu weit gegangen war. 

Dann gluckste Dumbledore. „Dann will ich rasch zum Kern der Sache kommen.“ Der Schulleiter lehnte sich vor, sein eingedellter Hut rutschte etwas zur Seite und der Bart fiel auf den Schreibtisch. „Harry, an diesem Montag hast du etwas getan, was selbst mit einem Zeitumkehrer nicht möglich sein sollte. Oder genauer, etwas, was nicht \emph{alleine} mit einem Zeitumkehrer zu erklären ist. Ich frage mich, wo diese zwei Kuchen wohl herkamen.“ 

Harry spürte einen Adrenalinschub. Er hatte das mit dem Unsichtbarkeitsumhang gemacht; mit dem Umhang, den er im Geschenkkarton zusammen mit einem Zettel gefunden hatte. Und auf dem Zettel hatte gestanden: \emph{Wenn Dumbledore eine Chance sähe, eines der Heiligtümer des Todes zu besitzen, dann würde er bis zum Tag seines Todes nicht davon ablassen.} 

„Eine naheliegende Vermutung wäre“, fuhr Dumbledore fort, „dass keiner der anwesenden Erstklässler einen solchen Spruch beherrscht. Folglich war eine weitere Person dort anwesend, wurde aber nicht gesehen. Wenn niemand sie sehen konnte, nun, dann wäre es ein Leichtes gewesen, die Kuchen zu werfen. Man könnte außerdem vermuten, dass du, da du einen Zeitumkehrer besitzt, diese unsichtbare Person warst; und da der Desillusionierungszauber weit jenseits deiner derzeitigen Möglichkeiten liegt, hattest du einen Unsichtbarkeitsumhang dabei.“ Dumbledore lächelte ihm verschwörerisch zu. „Bin ich soweit auf der richtigen Spur, Harry?“ 

Harry war erstarrt. Er hatte das Gefühl, dass eine faustdicke Lüge wirklich keine gute Idee war und ihm vermutlich auch nichts nützen würde, doch er wusste nicht, was er sonst sagen sollte. 

Dumbledore winkte lächelnd ab. „Keine Sorge, Harry, du hast nichts Schlimmes getan. Unsichtbarkeitsumhänge verstoßen nicht gegen die Regeln – ich nehme an, sie sind so selten, dass niemand jemals daran gedacht hat, sie zu verbieten. Aber ich habe mich eigentlich etwas ganz anderes gefragt.“ 

„Ach?“, sagte Harry in der normalsten Stimme, die er hinbekam. 

Dumbledores Augen glänzten vor Begeisterung. „Weißt du, Harry, wenn man ein paar Abenteuer durchstanden hat, bekommt man mit, wie so etwas abläuft. Man beginnt, Muster in der Welt zu sehen, ihren Rhythmus zu hören. Man beginnt, Verdacht zu schöpfen, noch \emph{bevor} die Auflösung präsentiert wird. Du bist der Junge, der überlebt hat, und irgendwie bekommst du vier Tage, nachdem du die britische Zaubererwelt kennengelernt hast, einen Unsichtbarkeitsumhang in die Hände. Solche Umhänge gibt es nicht in der Winkelgasse zu kaufen, doch es gibt \emph{einen,} der seinen Weg in die Hände eines vorherbestimmten Besitzers finden könnte. Darum kann ich mir die Frage nicht verkneifen, ob du durch irgendeinen seltsamen Zufall nicht nur \emph{einen} Unsichtbarkeitsumhang, sondern \emph{den} Unsichtbarkeitsumhang gefunden hast: eines der drei Heiligtümer des Todes, von dem es heißt, dass es seinen Träger vor dem Tod selbst verstecken kann.“ Dumbledores Augen strahlten eifrig. „Darf ich ihn sehen, Harry?“ 

Harry schluckte. Sein Adrenalinpegel war hochgeschnellt, doch es war völlig umsonst; dies war der mächtigste Zauberer der Welt und er würde unmöglich bis zur Tür kommen und selbst wenn, könnte er sich nirgendwo in Hogwarts verstecken, er würde den Umhang verlieren, der seit wer weiß wie vielen Generationen von einem Potter zum nächsten weitergereicht wurde – 

Langsam lehnte Dumbledore sich in seinem hohen Stuhl zurück. Das Glitzern in seinen Augen war verschwunden, er sah verwundert und etwas besorgt aus. „Harry“, sagte Dumbledore, „wenn du das nicht willst, kannst du einfach nein sagen.“ 

„Kann ich?“, krächzte Harry. 

„Ja, Harry“, sagte Dumbledore. Seine Stimme klang nun traurig und besorgt. „Mir scheint, dass du Angst vor mir hast, Harry. Darf ich fragen, womit ich dein Misstrauen verdient habe?“ 

Harry schluckte. „Können Sie irgendwie einen magischen Eid schwören, dass Sie mir den Umhang nicht wegnehmen werden?“ 

Dumbledore schüttelte langsam den Kopf. „Unbrechbare Schwüre sollten nicht so leichtfertig benutzt werden. Außerdem, Harry, hättest du, wenn du den Spruch nicht kennst, nur mein Wort dafür, dass er bindend ist. Dabei ist dir sicherlich bewusst, dass ich deine Erlaubnis nicht \emph{brauche,} um den Umhang zu sehen. Ich bin mächtig genug, um ihn selbst hervorzuziehen; egal, ob er in dem Eselsfellbeutel steckt oder nicht.“ Dumbledores Gesicht war sehr ernst. „Aber ich werde das nicht tun. Der Umhang gehört dir, Harry. Ich werde ihn dir nicht nehmen. Nicht einmal, um ihn einen Moment lang zu betrachten; es sei denn, du erlaubst es mir. Das ist ein Versprechen und ein Schwur. Müsste ich dir verbieten, ihn auf dem Schulgelände zu benutzen, dann würde ich dich zu deinem Verlies in Gringotts schicken, um ihn dort abzulegen.“ 

„Ähm …“, sagte Harry. Er schluckte fest; versuchte, den Adrenalinpegel unter Kontrolle zu bekommen und klar zu denken. Er nahm den Eselsfellbeutel von seinem Gürtel. „Wenn Sie meine Erlaubnis wirklich \emph{nicht} brauchen … dann haben sie diese.“ Harry hielt Dumbledore den Beutel hin und biss sich fest auf die Lippen, um sich selbst das Zeichen zu geben, falls er gleich einen Vergessenszauber abbekommen würde. 

Der alte Zauberer griff in den Beutel und holte, ohne ein Wort zu sagen, den Unsichtbarkeitsumhang hervor. 

„Ah“, sagte Dumbledore, „ich hatte Recht …“ Das schimmernd schwarze, samtige Gewebe floss ihm durch die Finger. „Jahrhunderte alt, aber doch so perfekt wie an dem Tag, als er geschaffen wurde. Wir haben viele unserer Künste in all den Jahren verloren und heutzutage kann ich einen solchen Gegenstand nicht mehr schaffen; niemand kann es mehr. Ich kann seine Macht in meinem Geist widerhallen hören, wie ein Lied, das ewig gesungen wird, ohne dass jemand es hört …“ Der Zauberer sah vom Umhang auf. „Verkaufe ihn nicht“, sagte er, „übergib ihn niemand anderem. Denke zwei Mal darüber nach, bevor du ihn jemandem zeigst, und denke drei Mal darüber nach, bevor du verrätst, dass er ein Heiligtum des Todes ist. Behandle ihn sorgsam, denn er ist wahrlich ein machtvolles Objekt.“ 

Einen Moment lang sah man die Sehnsucht in Dumbledores Gesicht … 

… dann gab er Harry den Umhang zurück. 

Harry steckte ihn zurück in den Beutel. 

Dumbledores Gesicht wurde wieder ernst. „Darf ich nochmals fragen, Harry, weshalb du mir so misstraust?“ 

Plötzlich schämte Harry sich. 

„Dem Umhang lag ein Zettel bei“, sagte Harry kleinlaut. „Darauf stand, Sie würden mir den Umhang wegnehmen, wenn Sie davon wüssten. Ich weiß aber nicht, von wem der Zettel stammte, wirklich nicht.“ 

„Ich … verstehe“, sagte Dumbledore langsam. „Nun, Harry, ich will es demjenigen, der dir diesen Zettel hinterlassen hat, nicht übel nehmen. Wer weiß, vielleicht steckten ja beste Absichten dahinter? Immerhin hat diese Person dir auch den Umhang überlassen.“ 

Harry nickte, beeindruckt von Dumbledores Güte und beschämt angesichts seines so gegenteiligen Verhaltens. 

Der alte Zauberer fuhr fort. „Aber du und ich, wir sind Spielfiguren der selben Farbe, denke ich. Der Junge, der Voldemort endlich besiegte, und der alte Mann, der ihn lange genug aufgehalten hat, dass du uns retten konntest. Ich will dir deine Vorsicht nicht übelnehmen, Harry, wir alle versuchen, so gut es geht weise zu sein. Ich möchte dich nur bitten, beim nächsten Mal doppelt und dreifach darüber nachzudenken, wenn jemand dir sagt, du solltest mir misstrauen.“ 

„Es tut mir Leid“, sagte Harry. Er fühlte sich elendig, immerhin hatte er gerade quasi Gandalf zurückgewiesen, und Dumbledores Güte machte es ihm nur noch schwerer. „Ich hätte Ihnen nicht misstrauen sollen.“ 

„Ach, Harry, in dieser Welt …“ Der alte Zauberer schüttelte mit dem Kopf. „Ich kann nicht einmal behaupten, dass es unklug war. Du kanntest mich nicht. Und um ehrlich zu sein, gibt es jene auf Hogwarts, denen du besser nicht vertrauen solltest. Möglicherweise sogar welche, die du als Freunde bezeichnen würdest.“ 

Harry schluckte. Das klang unheilvoll. „Wen denn?“ 

Dumbledore stand auf und begann, eines seiner Instrumente zu betrachten, eine Uhr mit acht Zeigern verschiedener Länge. 

Nach einigen Momenten sprach der alte Zauberer weiter. „Er erscheint dir vermutlich sehr charmant“, sagte Dumbledore. „Höflich – zumindest dir gegenüber. Wortgewandt, womöglich freigiebig mit Komplimenten. Stets hilfsbereit mit Rat und Tat –“ 

„Ach, \emph{Draco Malfoy!}“, sagte Harry, erleichtert, dass es nicht Hermine oder jemand anders war. „Oh nein, nein nein nein, Sie haben das vollkommen falsch verstanden. Er zieht mich nicht auf die schiefe Bahn – ich ziehe ihn auf die gerade Bahn.“ 

Dumbledore erstarrte zur Uhr gewandt. „Du \emph{was?}“ 

„Ich werde Draco Malfoy von der dunklen Seite abwenden“, sagte Harry. „Sie wissen schon, ihn zu einem der Guten machen.“ 

Dumbledore richtete sich kerzengerade auf und drehte sich zu Harry. Einen so erstaunten Gesichtsausdruck hatte Harry selten gesehen, schon gar nicht bei jemandem mit einem langen, silbernen Bart. „Bist du dir sicher“, sagte der alte Zauberer nach einem Moment, „dass das Gute, was du in ihm zu sehen glaubst, nicht nur Wunschdenken ist, Harry? Ich fürchte, das, was du da siehst, ist nur ein Trick, ein Köder –“ 

„Ähm, unwahrscheinlich“, sagte Harry. „Ich meine, wenn er versucht, so zu tun, als ob er ein guter Kerl ist, dann macht er das unglaublich schlecht. Wir reden hier nicht davon, dass Draco mich anspricht, total charmant ist und ich beschließe, dass er irgendwo tief in seinem Inneren ein guter Mensch sein muss. Ich habe ihn auserwählt, gerade weil er der Erbe des Hauses Malfoy ist. Wenn ich nur eine Person auswählen dürfte, die ich auf die gute Seite rüberhole, dann ist es selbstverständlich er.“ 

Dumbledores linkes Auge zuckte. „Du versuchst, Samen der Liebe und Freundschaft in Dracos Herz zu säen, weil du glaubst, dass der Erbe der Malfoys dir nützlich sein kann?“ 

„Nicht nur \emph{mir!}“, sagte Harry empört. „Der ganzen britischen Zaubererwelt, wenn alles gut geht! \emph{Und} er selbst wird ein glücklicheres und mental gesünderes Leben führen! Verstehen Sie, ich habe nicht genug Zeit, \emph{alle} von der dunklen Seite abzuwenden, also muss ich mir überlegen, wo das Gute am schnellsten den größten Gewinn erzielen kann –“ 

Dumbledore begann zu lachen. Er lachte viel lauter, als Harry es erwartet hatte; brüllte fast. Es erschien geradezu \emph{würdelos.} Ein alter und mächtiger Zauberer sollte in tiefen, donnernden Tönen glucksen, nicht so schallend lachen, dass er nach Atem rang. Harry war einmal wortwörtlich vor Lachen vom Stuhl gefallen, als er den Film \emph{Die Marx Brothers im Krieg} gesehen hatte, und Dumbledore lachte jetzt genauso laut. 

„Das ist nicht \emph{so} witzig“, sagte Harry nach einer Weile. Er begann wieder, sich um Dumbledores Verstand zu sorgen. 

Dumbledore bekam sich mit sichtbarer Mühe wieder ein. „Ach, Harry, eines der Symptome der Krankheit namens Weisheit ist, dass man beginnt, über Dinge zu lachen, die niemand sonst lustig findet, denn wenn man weise ist, Harry, versteht man den Witz!“ Der alte Zauberer wischte Tränen aus seinen Augen. „Hach ja, hach ja. Oft wird böser Wille Böses vereiteln. In der Tat. In der Tat.“ 

Harry brauchte einen Moment, bis er die Worte einordnen konnte … „Hey, das ist ein \emph{Tolkien}-Zitat! \emph{Gandalf} sagt das!“ 

„Theoden, genau genommen“, sagte Dumbledore. 

„Sie sind \emph{muggelgeboren?}“, sagte Harry schockiert. 

„Ich fürchte nicht“, sagte Dumbledore lächelnd. „Ich wurde siebzig Jahre vor Erscheinen des Buchs geboren, mein Lieber. Aber anscheinend haben meine muggelgeborenen Schüler oft ähnliche Gedanken. Ich habe über die Jahre nicht weniger als zwanzig Ausgaben von \emph{Der Herr der Ringe} angesammelt, zudem drei Ausgaben von Tolkiens gesammelten Werken, und ich hüte jede von ihnen wie einen Schatz.“ Dumbledore zog seinen Zauberstab, hielt ihn hoch und posierte. „\emph{Du kommst nicht vorbei!} Wie sehe ich aus?“ 

„Ähm“, sagte Harry, dessen Gehirn einem vollkommenen Aussetzer nahe war, „ich glaube, Ihnen fehlt der Balrog.“ Und die pinken Pyjamas und der eingedellte Champignonhut waren beileibe keine Hilfe. 

„Ich verstehe.“ Dumbledore seufzte und steckte den Zauberstab enttäuscht wieder ein. „Ich fürchte, mein Leben war zuletzt äußerst arm an Balrogs. Heutzutage ist es voller Sitzungen des Zaubergamots, bei denen ich verzweifelt versuchen muss, jegliches Vorankommen zu blockieren, und feierlicher Abendessen, bei denen ausländische Politiker darum wetteifern, wer der starrköpfigste Narr sein kann. Dazu etwas mysteriöses Benehmen; Dinge wissen, die ich gar nicht wissen kann; kryptische Bemerkungen, die erst im Nachhinein verständlich sind, und all die anderen Kleinigkeiten, mit denen mächtige Zauberer sich amüsieren, nachdem sie selbst die Rolle des Helden abgegeben haben. Apropos: Harry, ich möchte dir etwas ganz bestimmtes geben; etwas, das deinem Vater gehört hat.“ 

„Wirklich?“, sagte Harry. „Meine Güte, wer hätte das erwartet.“ 

„In der Tat“, sagte Dumbledore. „Es ist ein wenig vorhersehbar, nicht wahr?“ Sein Gesicht wurde feierlich. „Nichtsdestoweniger …“ 

Dumbledore ging wieder an seinen Schreibtisch und setzte sich, wobei er eine Schublade aufzog. Er griff mit beiden Armen hinein, hob mit einiger Mühe einen recht großen und schwer aussehenden Gegenstand aus der Schublade und legte ihn mit einem lauten Knall auf den eichenen Schreibtisch. 

„Das“, sagte Dumbledore, „war der Stein deines Vaters.“ 

Harry starrte darauf. Der Stein war hellgrau, verfärbt, unregelmäßig geformt, scharfkantig und ein vollkommen normaler großer Stein. Dumbledore hatte ihn auf die flachste Stelle gelegt, dennoch wippte er auf dem Tisch ein wenig hin und her. 

Harry sah auf. „Das ist ein Witz, oder?“ 

„Es ist kein Witz“, sagte Dumbledore, schüttelte den Kopf und sah sehr ernst aus. „Ich habe ihn aus den Ruinen von James’ und Lilys Haus in Godrics Hollow mitgenommen, wo ich auch dich gefunden habe, und ich habe ihn seitdem aufbewahrt, bis zu dem Tag, an dem ich ihn dir geben kann.“ 

Unter den vielen Hypothesen, die Harrys Modell der Welt darstellten, gewann jene, wonach Dumbledore verrückt war, rasch an Wahrscheinlichkeit. Doch die anderslautenden Hypothesen vereinten immer noch eine große Wahrscheinlichkeit auf sich … „Ähm, ist das ein \emph{magischer} Stein?“ 

„Nicht, dass ich wüsste“, sagte Dumbledore. „Aber ich rate empfehle dir mit höchster Dringlichkeit, ihn jederzeit bei dir zu haben.“ 

Okay. Dumbledore war \emph{wahrscheinlich} verrückt, aber wenn er es \emph{nicht} war … nun, es wäre einfach zu \emph{peinlich,} in Schwierigkeiten zu geraten, weil man den Rat des unergründlichen alten Zauberers nicht befolgte. Das musste ungefähr Platz 4 auf der Liste der 100 offensichtlichsten Fehler einnehmen. 

Harry trat vor und fasste den Stein an. Er suchte nach einer Stelle, an der er ihn anfassen und hochheben konnte, ohne sich selbst zu schneiden. „Dann werde ich ihn in meinen Beutel packen.“ 

Dumbledore runzelte die Stirn. „Ich weiß nicht, ob das nahe genug bei dir ist. Und was, wenn du den Beutel verlierst oder er gestohlen wird?“ 

„Sie sagen, ich sollte jederzeit einen großen Stein mit mir rumtragen, wo immer ich auch hingehe? 

Dumbledore sah Harry ernst an. „Das könnte sich als weise herausstellen.“ 

„Äh …“, sagte Harry. Der Stein sah recht schwer aus. „Ich glaube, die anderen Schüler würden mir deswegen einige Fragen stellen.“ 

„Sag ihnen, ich hätte es dir befohlen“, sagte Dumbledore. „Niemand wird daran zweifeln, weil alle mich für verrückt halten.“ Sein Gesichtsausdruck war immer noch vollkommen ernst. 

„Ähm, ehrlich gesagt, wenn Sie Schülern befehlen, große Steine mit sich rumzutragen, dann kann ich irgendwie nachvollziehen, warum Leute so etwas denken.“ 

„Ach, Harry“, sagte Dumbledore. Der alte Zauberer wedelte mit der Hand; eine Geste, die auf all die rätselhaften Gegenstände im Raum zu verweisen schien. „Wenn wir jung sind, glauben wir, dass wir alles wissen, und darum denken wir, wenn wir keine Erklärung für etwas kennen, dass keine Erklärung existiert. Wenn wir älter werden, wird uns bewusst, dass das Universum einem Muster folgt, einen Grund hat, auch wenn wir selbst ihn nicht kennen. Es ist unser eigenes Unwissen, dass uns als Wahnsinn erscheint.“ 

„Die Welt folgt stets Gesetzen“, sagte Harry, „auch wenn wir die Gesetze nicht kennen.“ 

„Exakt, Harry“, sagte Dumbledore. „Das zu verstehen – und ich merke, \emph{dass} du es verstehst – ist der Kern aller Weisheit.“ 

„Also … \emph{warum} genau soll ich diesen Stein mit mir rumtragen?“

„Ich wüsste ehrlich gesagt keinen Grund“, sagte Dumbledore. 

„Sie … wissen keinen Grund.“ 

Dumbledore nickte. „Aber nur, weil ich keinen Grund wüsste, heißt das noch längst nicht, dass es keinen Grund \emph{gibt.}“ 

Die rätselhaften Gegenstände tickten weiter. 

„Okay“, sagte Harry, „ich weiß nicht mal, ob ich das sagen sollte, aber das ist einfach nicht die richtige Art, mit unserem zugegebenermaßen sehr großen Unwissen darüber, wie das Universum funktioniert, umzugehen.“ 

„Nicht?“, sagte der alte Zauberer und sah überrascht und enttäuscht aus. 

Harry ahnte, dass dieses Gespräch nicht zu seinen Gunsten ausgehen würde, aber er fuhr trotzdem fort. „Nein. Ich weiß nicht mal, ob dieser Fehlschluss eine offizielle Bezeichnung hat, aber wenn ich mir etwas ausdenken müsste, dann würde ich ihn ‚Bevorzugung der Hypothese‘ nennen, oder so etwas. Wie kann ich das formal beschreiben … ähm … angenommen, Sie hätten eine Million Schachteln und in nur einer davon ist ein Diamant drin. Und sie hätten eine Kiste voller Diamantendetektoren, die beim Diamanten jedes Mal aufleuchten und bei einer leeren Schachtel in der Hälfte der Fälle aufleuchten. Wenn Sie die Schachteln mit zwanzig Detektoren überprüfen würden, blieben durchschnittlich nur noch eine falsche Schachtel und die richtige Schachtel übrig. Und dann bräuchten Sie nur noch einen oder zwei Detektoren, um die richtige Schachtel zu identifizieren. Das heißt, wenn es viele Möglichkeiten gibt, dann werden \emph{die meisten} Hinweise nur dafür gebraucht, die richtige Hypothese unter Millionen anderer Hypothesen zu \emph{finden} – um überhaupt erst mal darauf aufmerksam zu werden. Die Zahl der Hinweise, die Sie brauchen, um zwischen zwei oder drei wahrscheinlichen Kandidaten zu unterscheiden, ist verglichen damit viel geringer. Wenn Sie sich also einfach auf ein Problem stürzen und sich auf irgendeine der möglichen Antworten konzentrieren, dann lassen Sie einen Großteil der Arbeit aus. Zum Beispiel, wenn Sie in einer Stadt mit einer Million Einwohnern leben, ein Mord geschieht und der Polizist sagt, naja, wir haben absolut keine Hinweise, haben wir also schon geprüft, ob Mortimer Snodgrass der Täter war?“ 

„War er es?“, sagte Dumbledore. 

„Nein“, sagte Harry. „Aber später stellt sich raus, dass der Mörder schwarze Haare hatte, und Mortimer hatte auch schwarze Haare, also sagen alle, ach, also war es tatsächlich Mortimer. Also ist es Mortimer gegenüber unfair, wenn die Polizei sich \emph{auf ihn konzentriert,} ohne dass sie bereits einen guten Grund hat, ihn zu verdächtigen. Wenn es viele Möglichkeiten gibt, dann ist der größte Teil der Arbeit, überhaupt erstmal die richtige Hypothese zu finden – auf sie aufmerksam zu werden. Man braucht keinen \emph{Beweis,} nicht so deutliche Belege, wie ein Wissenschaftler oder ein Richter sie fordern würde, aber man braucht irgendeinen \emph{Hinweis.} Und dieser Hinweis muss helfen, diese bestimmte Möglichkeit von den Millionen von anderen Möglichkeiten zu unterscheiden. Ansonsten kann man die richtige Antwort nicht einfach so aus der Luft greifen. Man kann nicht einmal eine wahrscheinliche Antwort aus der Luft greifen. Und ich könnte genauso gut eine Million andere Dinge tun, statt den Stein meines Vaters mit mir rumzutragen. Nur weil ich nicht weiß, wie das Universum funktioniert, heißt das noch lange nicht, dass ich nicht weiß, wie ich mit meinem Unwissen zurecht kommen soll. Die Gesetze der Wahrscheinlichkeitsrechnung stehen ebenso fest wie die Gesetze der altbekannten Logik, und was Sie da gerade getan haben, ist \emph{nicht erlaubt.}“ Harry unterbrach sich. „\emph{Außer,} natürlich, wenn Sie irgendeinen \emph{Hinweis} haben, den Sie nicht erwähnt haben.“ 

„Ah“, sagte Dumbledore. Er rieb sich nachdenklich das Kinn. „Ein interessantes Argument, zweifelsohne, doch bricht es nicht an dem Punkt zusammen, wo du eine Million mögliche Mörder, von denen nur einer den Mord begangen hat, mit den vielen verschiedenen Handlungsmöglichkeiten vergleichst, wo doch viele verschiedene Handlungen zugleich weise könnten? Ich sage nicht, dass es die bestmögliche Idee ist, den Stein deines Vaters mit dir rumzutragen; nur, dass es besser ist, es zu tun, als es nicht zu tun.“ 

Dumbledore griff erneut in die Schublade, aus der er den Stein geholt hatte, dieses Mal wühlte er darin rum – zumindest schien sein Arm sich zu bewegen. „Ich möchte bemerken“, sagte Dumbledore, während Harry immer noch überlegte, wie er auf diese völlig unerwartete Erwiderung reagieren sollte, „dass es unter Ravenclaws ein häufiges Missverständnis ist, dass alle klugen Kinder in ihr Haus kommen, so dass in den anderen Häuser keine sind. Dem ist nicht so; nach Ravenclaw zu kommen bedeutet, dass man vom Streben nach Wissen angetrieben wird, was keineswegs das gleiche ist wie Intelligenz.“ Der Zauberer lächelte, während er sich über die Schublade beugte. „Nichtsdestoweniger scheinst \emph{du} sehr intelligent zu sein. Nicht wie ein gewöhnlicher junger Held, eher wie ein junger mysteriöser uralter Zauberer. Ich denke, ich habe dich wohl falsch eingeschätzt, Harry, und dass du Dinge verstehen könntest, die kaum ein anderer begreifen kann. Darum will ich es wagen, dir ein gewisses \emph{anderes} Erbstück anzubieten.“ 

„Soll das heißen …“, keuchte Harry, „mein Vater … \emph{besaß noch einen Stein?}“ 

„Entschuldige“, sagte Dumbledore, „ich \emph{bin} immer noch älter und rätselhafter als du. Wenn es hier noch Dinge zu offenbaren gibt, dann werde \emph{ich} sie offenbaren, wenn ich bitten darf … ach, wo \emph{ist} es denn jetzt bloß!“ Dumbledore griff tiefer und tiefer in die Schublade hinein. Sein Kopf, seine Schultern und schließlich sein ganzer Oberkörper verschwanden darin, als würde die Schublade ihn aufessen. 

Harry fragte sich, wie viele Dinge nur da drin waren und wie es aussähe, wenn das alles auf einem Haufen wäre. 

Schließlich kam Dumbledore aus der Schublade hervor und legte das Ziel seiner Suche neben den Stein auf den Schreibtisch. 

Es war ein altes Lehrbuch mit abgegriffenen Rändern und geknicktem Buchrücken: \emph{Zaubertränke: Die Zwischenstufen} von Libatius Borage. Auf der Vorderseite war eine dampfende Phiole abgebildet. 

„Das“, betonte Dumbledore, „war das Zaubertränke-Lehrbuch deiner Mutter in ihrem fünften Schuljahr.“ 

„Und ich sollte es jederzeit mit mir rumtragen“, sagte Harry. 

„\emph{Und es verbirgt ein schreckliches Geheimnis.} Ein Geheimnis, dessen Bekanntwerden so desaströs wäre, dass ich dich bitte zu schwören – und egal was du von all dem hältst, Harry, ich verlange, dass du es in vollem Ernst schwörst –, dass du niemals irgendwem oder irgendetwas anderem davon erzählst.“ 

Harry betrachtete das Zaubertränke-Lehrbuch seiner Mutter, das offenbar ein schreckliches Geheimnis verbarg. 

Das Problem war, dass Harry solche Schwüre sehr ernst nahm. Jeder Schwur war ein Unbrechbarer Schwur, wenn die richtige Person ihn leistete. 

Und … 

„Ich habe Durst“, sagte Harry, „und das ist wirklich kein gutes Zeichen.“ 

Dumbledore reagierte gar nicht auf diese kryptische Aussage. „\emph{Schwörst} du, Harry?“, sagte Dumbledore. Seine Augen fixierten die von Harry. „Ansonsten kann ich es dir nicht erzählen.“ 

„Ja“, sagte Harry. „Ich schwöre.“ Das war das Problem, wenn man ein Ravenclaw war. Man konnte so ein Angebot nicht ausschlagen, sonst würde die eigene Neugier einen von innen auffressen. Jeder wusste das. 

„Und ich schwöre im Gegenzug“, sagte Dumbledore, „dass das, was ich dir nun erzählen werde, die Wahrheit ist.“ 

Dumbledore öffnete das Buch an einer offenbar zufälligen Stelle und Harry beugte sich vor, um es zu sehen. 

„Siehst du die Notizen“, sagte Dumbledore in einer leisen Stimme, fast einem Flüstern, „die am Seitenrand stehen?“ 

Harry kniff die Augen zusammen. Auf den vergilbten Seiten war der \emph{Trank der Adlerspracht} beschrieben, viele seiner Zutaten waren Harry nicht bekannt und ihre Namen schienen nicht aus dem Englischen abgeleitet zu sein. Auf den Seitenrand war eine handgeschriebene Bemerkung gekritzelt: \emph{Was würde wohl passieren, wenn man hier Thestralblut statt Blaubeeren nimmt?} Direkt darunter stand die Antwort in einer anderen Handschrift: \emph{Man wäre wochenlang krank und würde vielleicht sterben.} 

„Ich sehe sie“, sagte Harry. „Was ist mit ihnen?“ 

Dumbledore deutete auf den zweiten Satz. „Diejenigen in dieser Handschrift“, sagte er immer noch mit leiser Stimme, „wurden von deiner Mutter geschrieben. Und die in \emph{dieser} Handschrift“, sein Finger deutete nun auf den ersten Satz, „wurden von mir geschrieben. Ich habe mich unsichtbar gemacht und bin in ihren Schlafsaal geschlichen, während sie schlief. Lily dachte, dass eine Freundin von ihr das schrieb, und sie haben sich deswegen ganz schön gestritten.“ 

In genau diesem Moment wurde Harry klar, dass der Schulleiter von Hogwarts \emph{tatsächlich} verrückt war. 

Dumbledore sah ihn mit ernstem Gesichtsausdruck an. „Verstehst du die Bedeutung dessen, was ich dir gerade erzählt habe, Harry?“ 

„Äh …“, sagte Harry. Seine Stimme stockte. „Sorry … ich … nicht wirklich …“ 

„Ach ja“, sagte Dumbledore und seufzte. „Also hat wohl auch deine Intelligenz Grenzen. Es scheint, als wäre meine Begeisterung äußerst voreilig gewesen. Wollen wir davon ausgehen, dass ich nichts Belastendes zugegeben habe?“ 

Harry stand auf, ein starres Lächeln auf dem Gesicht. „Natürlich“, sagte Harry. „Wissen Sie, es wird schon spät und ich habe Hunger, also sollte ich wirklich runter zum Abendessen gehen.“ Und Harry eilte zur Tür. 

Der Türknauf drehte sich kein Stückchen. 

„Du verletzt mich, Harry“, sagte Dumbledores Stimme direkt hinter ihm leise. „Ist dir nicht zumindest klar, dass es ein Zeichen des Vertrauens ist, was ich dir erzählt habe?“ 

Harry drehte sich langsam um. 

Vor ihm stand ein sehr mächtiger und sehr verrückter Zauberer mit einem langen, silbernen Bart, einem Hut, der wie ein riesiger eingedellter Champignon aussah und Kleidung, die für Muggelaugen wie drei Schichten kreischend pinker Pyjamas aussah. 

Hinter ihm war eine Tür, die derzeit offenbar nicht funktionierte. 

Dumbledore sah recht traurig und müde aus, als ob er sich auf einen Stab lehnen wollte, den er nicht hatte. 

„Also wirklich“, sagte Dumbledore, „da probiert man einmal etwas Neues aus, statt dem selben Muster zu folgen wie die letzten einhundertundzehn Jahre, und schon laufen die Leute weg.“ Der alte Zauberer schüttelte trauernd den Kopf. „Ich hatte mehr von dir erhofft, Harry Potter. Ich hatte gehört, dass auch deine eigenen Freunde dich für verrückt halten. Ich weiß, dass sie sich irren. Willst du nicht das selbe über mich denken? 

„Bitte öffnen Sie die Tür“, sagte Harry mit zitternder Stimme. „Wenn Sie wollen, dass ich Ihnen jemals wieder vertraue, dann öffnen Sie die Tür.“ 

Hinter ihm erklang das Geräusch einer sich öffnenden Tür. 

„Ich wollte dir noch andere Dinge sagen“, sagte Dumbledore, „und wenn du jetzt gehst, wirst du nicht erfahren, welche.“ 

Manchmal \emph{hasste} Harry es wirklich, ein Ravenclaw zu sein. 

\emph{Er hat nie einem Schüler auch nur ein Haar gekrümmt}, sagte Harrys Gryffindorseite. \emph{Behalte das jederzeit im Kopf, dann hast du keinen Grund, in Panik auszubrechen! Du willst doch nicht etwa wegrennen, wenn die Dinge gerade versprechen, interessant zu werden?} 

\emph{Du kannst doch nicht einfach vor dem Schulleiter weglaufen!}, sagte seine Hufflepuffseite. \emph{Was, wenn er anfängt, dir Hauspunkte abzuziehen? Er könnte deine Schulzeit sehr unangenehm machen, wenn er beschließt, dass er dich nicht mag!} 

Und ein Teil von ihm selbst, den Harry nicht mochte, den er aber auch nicht ganz ruhiggestellt bekam, dachte an die möglichen Vorteile, die man als einer der wenigen Freunde dieses verrückten alten Zauberers, der zudem Schulleiter, Großmeister des Zaubergamot und Ganz hohes Tier war, genießen würde. Und sein innerer Slytherin war leider sehr viel besser als Draco darin, Leute auf die dunkle Seite zu ziehen; er sagte Dinge wie \emph{Der arme Kerl, er sieht so aus, als ob er jemanden zum Reden braucht, nicht wahr?} und \emph{Du willst doch nicht, dass so ein mächtiger Mann sich einer anderen, nicht so tugendhaften Person anvertraut?} und \emph{Ich frage mich, welche unglaublichen Geheimnisse Dumbledore dir verraten könnte, wenn du, du weißt schon, dich mit ihm anfreunden würdest} und sogar \emph{Wetten, dass er eine seeeehr interessante Büchersammlung hat?} 

\emph{Ihr seid doch alle verrückt}, dachte Harry zu ihnen allen, doch sämtliche Teile von ihm hatten sich einstimmig gegen ihn gestellt. 

Harry drehte sich um, ging einen Schritt auf die Tür zu, streckte die Hand aus und schloss die Tür wieder. Da er ohnehin hier bleiben würde, kostete ihn das nichts, Dumbledore könnte seinen Bewegungsspielraum ohnehin kontrollieren, aber vielleicht beeindruckte es Dumbledore. 

Als Harry sich wieder umdrehte, sah er, dass der mächtige verrückte Zauberer wieder lächelte und freundlich aussah. Das war gut – vielleicht. 

„Bitte tun Sie das nicht nochmal“, sagte Harry. „Ich mag es nicht, eingesperrt zu sein.“ 

„Es tut mir Leid, Harry“, sagte Dumbledore in einem reuevollen Tonfall. „Aber es wäre schrecklich unklug gewesen, dich ohne den Stein deines Vaters gehen zu lassen.“ 

„Natürlich“, sagte Harry. „Ich kann nicht ernsthaft erwarten, dass die Tür aufgeht, bevor ich die Quest-Gegenstände in mein Inventar gelegt habe.“ 

Dumbledore lächelte und nickte. 

Harry ging zum Schreibtisch, zog seinen Beutel am Gürtel nach vorne und schaffte es mit einiger Anstrengung, den Stein in seine elfjährigen Arme zu nehmen und an den Beutel zu verfüttern. 

Er konnte fühlen, wie das Gewicht langsam nachließ, während die magisch geweitete Beutelöffnung den Stein verschluckte. Der folgende Rülpser war recht laut und in einem anklagenden Tonfall gehalten. 

Das Zaubertränke-Lehrbuch seiner Mutter (das ein wirklich schreckliches Geheimnis verbarg) folgte kurz darauf. 

Und dann machte Harrys innerer Slytherin einen listigen Vorschlag, wie er sich mit dem Schulleiter gut stellen könnte, der leider genau so vorgetragen wurde, dass er die Unterstützung der Ravenclaw-Fraktion – und damit der Mehrheit – hatte. 

„Also“, sagte Harry. „Ähm. Wo ich gerade hier bin, würden Sie mir da vielleicht Ihr Büro etwas zeigen? Es würde mich schon interessieren, was einige dieser Dinge sind.“ \emph{Die Untertreibung des Monats September.} 

Dumbledore warf ihm einen Blick zu und nickte dann mit einem leichten Grinsen. „Dein Interesse schmeichelt mir“, sagte Dumbledore, „aber ich fürchte, es gibt nicht viel zu erzählen.“ Dumbledore ging einen Schritt auf die Wand zu und deutete auf ein Gemälde eines schlafenden Mannes. „Das sind Porträts vergangener Schulleiter von Hogwarts.“ Er drehte sich um und zeigte auf den Schreibtisch. „Das ist mein Schreibtisch.“ Er zeigte auf seinen Stuhl. „Das ist mein Stuhl –“ 

„Entschuldigen Sie“, sagte Harry, „aber ich meinte eher sowas.“ Harry deutete auf einen kleinen Würfel, der leise „blorp … blorp … blorp“ flüsterte. 

„Ach, die kleinen kniffligen Dinger?“, sagte Dumbledore. „Die gehören zum Büro dazu und ich habe absolut keine Ahnung, was die meisten davon machen. Aber \emph{dieses} Zifferblatt mit den acht Zeigern zählt die Anzahl der, sagen wir mal, Nieser von linkshändigen Hexen innerhalb Frankreichs; du glaubst gar nicht, wie aufwändig es war, das herauszufinden. Und \emph{dies} hier, mit den goldenen Fühlern, ist meine eigene Erfindung und Minerva wird nie und nimmer herausfinden, was es tut.“ 

Dumbledore ging einen Schritt auf die Hutablage zu, während Harry das immer noch verarbeitete. „Dies ist natürlich der Sprechende Hut, ich glaube, ihr habt einander schon kennengelernt. Er sagte mir, dass er auf keinen Fall noch einmal auf deinen Kopf gesetzt werden will. Du bist erst der vierzehnte Schüler in der Geschichte der Schule, über den er das gesagt hat. Baba Jaga war eine der anderen, und von den restlichen zwölf erzähle ich dir, wenn du älter bist. Das ist ein Regenschirm. Das ist noch ein Regenschirm.“ Dumbledore ging noch einige Schritte und drehte sich breit grinsend um. „Und die meisten Leute, die in mein Büro kommen, wollen natürlich Fawkes sehen.“ 

Dumbledore stand neben der goldenen Platte, auf der ein Vogel saß. 

Harry kam verwirrt näher. „Das ist Fawkes?“ 

„Fawkes ist ein Phönix“, sagte Dumbledore. „Sehr seltene, sehr mächtige magische Kreaturen.“ 

„Äh …“, sagte Harry. Er senkte den Kopf und blickte genau in die winzigen, perlenförmigen schwarzen Augen, die kein bisschen Macht oder Weisheit erkennen ließen. 

„Ähh …“, sagte Harry wieder. 

Er war sich ziemlich sicher, dass er die Form des Vogels erkannte. Das war kaum zu verwechseln. 

„Ähm …“ 

\emph{Sag etwas Intelligentes!}, schrie Harrys Gehirn sich selbst an. \emph{Steh’ nicht einfach so da wie ein stotternder Irrer!} 

\emph{Na, was zum Teufel} soll \emph{ ich denn sagen?}, brüllte Harrys Gehirn zurück. 

\emph{Irgendwas!} 

\emph{Du meinst, irgendwas außer „Fawkes ist ein Hühnchen“ –} 

\emph{Ja! Irgendwas, bloß nicht das!} 

„Also, äh, was für eine Art von Magie tun Phönixe denn?“ 

„Ihre Tränen haben heilende Kräfte“, sagte Dumbledore. „Sie sind Geschöpfe des Feuers und können sich von einem Ort zum nächsten bewegen, ebenso wie Feuer an einem Ort erlöschen und am nächsten entzündet werden kann. Die große Last ihrer eigenen Magie lässt ihren Körper schnell altern, und doch sind sie der Unsterblichkeit näher als jedes andere Geschöpf auf dieser Welt, denn wann immer ihr Körper sie im Stich lässt, gehen sie von selbst in Flammen auf und lassen ein Küken zurück, oder manchmal ein Ei.“ Dumbledore kam näher und betrachtete das Hühnchen stirnrunzelnd. „Hm … er sieht ein wenig mitgenommen aus, würde ich sagen.“ 

Noch bevor Harry diese Informationen ganz verarbeitet hatte, stand das Hühnchen schon in Flammen. 

Sein Schnabel öffnete sich, aber es konnte kein einziges Mal krächzen, bevor es zusammensank und verkohlte. Die Flammen loderten kurz und stark auf und waren vollkommen isoliert; es roch nicht verbrannt. 

Das Feuer erlosch nur wenige Sekunden, nachdem es begonnen hatte, und ließ ein kleines, armseliges Häufchen Asche auf der goldenen Platte zurück. 

„Schau nicht so entsetzt drein, Harry!“, sagte Dumbledore. „Fawkes ist unverletzt.“ Dumbledores Hand griff in eine Tasche, dann fuhr die selbe Hand durch die Asche und beförderte ein kleines gelbliches Ei hervor. „Schau, hier ist ein Ei!“ 

„Oh … wow … beeindruckend …“ 

„Aber jetzt sollten wir uns um die restlichen Dinge kümmern“, sagte Dumbledore. Er ließ das Ei in der Asche des Hühnchens zurück, kehrte zu seinem Platz zurück und setzte sich. „Es ist schließlich fast Zeit für das Abendessen, und wir wollen doch nicht unsere Zeitumkehrer benutzen.“ 

In Harrys innerer Regierung herrschte ein gewaltiger Machtkampf. Slytherin und Hufflepuff hatten die Seiten gewechselt, nachdem sie gesehen hatten, wie wie der Schulleiter von Hogwarts ein Hühnchen in Brand steckte. 

„Ja, Dinge“, sagten Harrys Lippen. „Und dann Abendessen.“ 

\emph{Du klingst wieder wie ein stotternder Irrer}, bemerkte Harrys innerer Kritiker. 

„Nun“, sagte Dumbledore, „ich fürchte, ich muss etwas gestehen, Harry. Etwas gestehen und um Entschuldigung bitten.“ 

„Entschuldigungen sind gut.“ \emph{Das ergibt überhaupt keinen Sinn! Was erzähle ich da?} 

Der alte Zauberer seufzte tief. „Du magst anders darüber denken, wenn du erfährst, was ich zu sagen habe. Ich fürchte, Harry, dass ich dich dein ganzes Leben lang manipuliert habe. Ich war es, der dich deinen bösen Stiefeltern anvertraut hat –“ 

„Meine Stiefeltern sind nicht böse!“, platzte es aus Harry heraus. „Meine \emph{Eltern,} meine ich!“ 

„Nicht?“, sagte Dumbledore, der überrascht und enttäuscht aussah. „Kein bisschen böse? Das passt nicht …“ 

Harrys innerer Slytherin schrie so laut er konnte: SEI LEISE, DU IDIOT, ER WIRD DICH IHNEN WEGNEHMEN! 

„Nein, nein“, sagte Harry, die Lippen zu einer starren Grimasse verzerrt, „ich wollte nur Ihre Gefühle nicht verletzen; in Wirklichkeit sind sie sehr böse …“ 

„Tatsächlich?“ Dumbledore beugte sich vor und sah ihn aufmerksam an. „Was machen sie?“ 

\emph{Rede schnell.} „Sie, äh, ich muss Geschirr lösen und Matheaufgaben abspülen und sie lassen mich nicht viele Bücher lesen und –“ 

„Ah, gut, das ist gut zu hören“, sagte Dumbledore und lehnte sich wieder zurück. Er lächelte traurig. „Dann bitte ich \emph{dafür} um Entschuldigung. Nun, wo war ich? Oh, ja. Ich muss dir leider sagen, Harry, dass ich für nahezu alles Schlechte verantwortlich bin, was dir je passiert ist. Ich weiß, dass dich das vermutlich sehr wütend macht.“ 

„Ja, ich bin sehr wütend!“, sagte Harry. „Grrr!“ 

Harrys innerer Kritiker verlieh ihm prompt den Hauptpreis für die schlechteste Schauspielleistung in der Geschichte aller Zeiten. 

„Und ich möchte dir sagen“, fuhr Dumbledore fort, „ich möchte es dir so früh wie nur möglich sagen, falls einem von uns später etwas zustoßen sollte, dass es mir wirklich, wirklich Leid tut. Alles was geschehen ist und alles, was noch geschehen wird.“ 

Die Augen des alten Zauberers glitzerten feucht. 

„Und ich bin sehr wütend!“, sagte Harry. „So wütend, dass ich sofort gehen will, es sei denn, es gibt noch etwas, was Sie mir sagen möchten!“ 

\emph{GEH einfach, bevor er dich in Brand steckt!}, schrien Slytherin, Hufflepuff und Gryffindor. 

„Ich verstehe“, sagte Dumbledore. „Dann noch ein Letztes, Harry. Du wirst dir die verbotene Tür im Korridor im dritten Stock \emph{nicht} näher ansehen. Du würdest es garantiert nicht schaffen, all die Fallen zu durchstehen, und ich möchte nicht hören, dass du dich bei dem Versuch verletzt hast. Außerdem könntest du ja nicht mal die erste Tür öffnen, da sie verschlossen ist und du den Zauberspruch \emph{Alohomora} nicht kennst –“ 

Harry drehte sich um und rannte so schnell er konnte zur Tür, der Türknauf drehte sich bereitwillig in seiner Hand und er raste die rotierende Wendeltreppe runter, stolperte fast über seine eigenen Füße, einen Moment später war er unten angelangt und der Wasserspeier schritt beiseite und Harry schoss aus der Tür wie eine Kanonenkugel. 

\later 

Harry Potter. 

Es musste an Harry Potter liegen. 

Schließlich war für alle Anderen auch Donnerstag, aber solche Sachen schienen niemandem sonst zuzustoßen. 

Es war Donnerstag Nachmittag, 18:21~Uhr, als Harry Potter, der wie eine Kanonenkugel aus der Tür schoss und so schnell er konnte weg rannte, direkt mit Professor McGonagall zusammenprallte, die gerade um die Ecke kam und auf dem Weg zum Büro des Schulleiters war. 

Glücklicherweise blieben beide fast unverletzt. Wie Harry kurz vorher erklärt bekommen hatte – vorhin, als er sich geweigert hatte, jemals wieder einem Besen nahe zu kommen –, bestanden die Klatscher im Quidditch aus hartem Stahl, damit sie überhaupt eine vernünftige Chance hatten, die Spieler zu verletzen. Zauberer hielten sehr viel stärkere physische Belastungen aus als Muggel. 

Harry und Professor McGonagall landeten beide am Boden und die Pergamente, die sie in den Armen gehabt hatte, verteilten sich im ganzen Flur. 

Eine schreckliche, schreckliche Stille herrschte. 

„Harry Potter“, keuchte Professor McGonagall, die direkt neben Harry am Boden lag. Ihre Stimme steigerte sich fast zu einem Kreischen. \emph{„Was haben Sie im Büro des Schulleiters gemacht?“} 

„Nichts!“, quiekte Harry. 

\emph{„Haben Sie über Professor Quirrell gesprochen?“} 

„Nein! Dumbledore hat mich hergebeten und hat mir einen großen Stein gegeben und gesagt, dass der meinem Vater gehört hat und dass ich ihn stets mit mir rumtragen soll!“ 

Wieder herrschte eine schreckliche Stille. 

„Ich verstehe“, sagte Professor McGonagall in etwas ruhigerer Stimme. Sie stand auf, klopfte Staub von ihrem Umhang und warf den Pergamenten einen zornigen Blick zu. Diese bildeten sogleich einen tadellosen Stapel und eilten an die Wand, als ob sie dem Blick auswichen. „Mein Beileid, Mr~Potter, und es tut mir Leid, dass ich an Ihnen gezweifelt habe.“ 

„Professor McGonagall“, sagte Harry. Seine Stimme zitterte. Er stützte sich ab, stand auf und blickte in ihr vertrauenswürdiges, \emph{vernünftiges} Gesicht. „Professor McGonagall …“ 

„Ja, Mr~Potter?“ 

„Glauben Sie, ich sollte das tun?“, sagte Harry in einer leisen Stimme. „Den Stein meines Vater stets mit mir rumtragen?“ 

Professor McGonagall seufzte. „Das müssen Sie und der Schulleiter unter sich ausmachen, fürchte ich.“ Sie zögerte. „Ich möchte anmerken, dass es fast nie klug ist, den Schulleiter zu ignorieren. Es \emph{tut} mir jedoch Leid, dass Sie in diesem Dilemma stecken, Mr~Potter, und falls ich Ihnen irgendwie helfen kann, eine Entscheidung zu fällen, –“ 

„Ähm“, sagte Harry. „Nun, ich habe mir überlegt, sobald ich weiß, wie es geht, könnte ich den Stein in einen Ring verwandeln und an einem Finger tragen. Wenn Sie mir beibringen könnten, wie ich eine Verwandlung dauerhaft aufrecht erhalte –“ 

„Es ist gut, dass Sie mich zuerst gefragt haben“, sagte Professor McGonagall mit etwas strengerem Gesichtsausdruck. „Wenn Sie es nicht geschafft hätten, die Verwandlung aufrecht zu erhalten, hätte die Rückverwandlung Ihnen den Finger abgerissen und vermutlich Ihre Hand gespalten. Und in Ihrem Alter ist selbst ein Ring ein so großes Zielobjekt, dass Sie die Verwandlung nicht langfristig aufrecht erhalten könnten, ohne Ihre Magie wesentlich zu belasten. Ich kann jedoch einen Ring schmieden lassen, mit einer Einfassung für einen Edelstein – einen \emph{kleinen} Edelstein –, der ihre Haut berührt, und dann können Sie mit einem sicheren Gegenstand wie einem Marshmallow üben. Sobald Sie es geschafft haben, die Verwandlung einen ganzen Monat lang, auch im Schlaf, aufrecht zu erhalten, werde ich Ihnen erlauben, äh, den Stein Ihres Vaters …“ Professor McGonagalls Stimme verlief sich. „Hat der Schulleiter \emph{wirklich} –“ 

„Ja. Äh … und …“ 

Professor McGonagall seufzte. „Das ist selbst für seine Verhältnisse etwas seltsam.“ Sie bückte sich und hob den Stapel Pergamente auf. „Es tut mir Leid, Mr~Potter. Ich bitte nochmals um Entschuldigung, dass ich Ihnen misstraut habe. Aber nun werde ich selbst den Schulleiter treffen.“ 

„Äh … viel Glück, würde ich sagen. Also …“ 

„Danke, Mr~Potter.“ 

„Ähm …“ 

Professor McGonagall ging zum Wasserspeier, sagte unhörbar das Passwort und trat auf die rotierende Wendeltreppe. Harry verlor sie aus den Augen, der Wasserspeier begann den Durchgang wieder zu versperren – 

\emph{„Professor McGonagall, der Schulleiter hat ein Hühnchen angezündet!“} 

„Er \emph{wa–}“
