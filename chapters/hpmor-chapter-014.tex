\chapter{Das Unverstandene und das Unverständliche}

\emph{Es gab mysteriöse Fragen, doch eine mysteriöse Antwort war ein Widerspruch in sich.} 

\later 

\lettrinepara[ante=„]{H}{erein“,} sagte Professor McGonagalls gedämpfte Stimme. 

\hplettrineextrapara
Harry trat ein. 

Das Büro der Stellvertretenden Schulleiterin war sauber und aufgeräumt; an den Wänden unmittelbar neben dem Schreibtisch war ein Haufen von Schubladen aller Formen und Größen, in denen oft einige Pergamentrollen steckten, und es erschien irgendwie sehr offensichtlich, dass Professor McGonagall genau wusste, was in jeder Schublade war, auch wenn es sonst niemand tat. Ein einziges Pergament lag auf dem Schreibtisch selbst, der ansonsten leer war. Dahinter war eine Tür, die mit mehreren Schlössern versperrt war. 

Professor McGonagall saß auf einem Hocker hinter dem Schreibtisch und sah irritiert aus – ihre Augen waren geweitet und möglicherweise ein wenig ahnungsvoll, als sie Harry sah. 

„Mr~Potter?“, sagte Professor McGonagall. „Worum geht es?“ 

Harrys Gehirn wusste nicht weiter. Das Spiel hatte ihm angeordnet, hierhin zu kommen, er hatte erwartet, dass \emph{sie} irgendetwas vor hatte… 

„Mr~Potter?“, sagte McGonagall und begann genervt auszusehen. 

Glücklicherweise erinnerte sich Harrys in Panik geratenes Gehirn in diesem Moment daran, dass es \emph{etwas} gab, worüber er mit Professor McGonagall sprechen wollte. Etwas Wichtiges, was ihre Zeit durchaus wert war. 

„Ähm…“, sagte Harry. „Wenn es irgendwelche Zaubersprüche gibt, die Sie sprechen könnten um sicherzustellen, dass uns niemand belauscht…“ 

Professor McGonagall stand auf, schloss die Tür bestimmt und zog ihren Zauberstab, mit dem sie Sprüche aufsagte. 

In diesem Moment bemerkte Harry, dass dies eine unbezahlbare und möglicherweise unwiederbringliche Gelegenheit war, Professor McGonagall einen Seltsaft anzubieten und er konnte nicht glauben, dass er das ernsthaft dachte und es würde nichts passieren, die Limo würde einfach nach einigen Sekunden verschwinden und er sagte diesem Teil von sich, er solle \emph{die Klappe halten.} 

Dieser tat es und Harry suchte zusammen, was er sagen würde. Er wollte dieses Gespräch eigentlich nicht \emph{so} früh führen, aber wenn er jetzt schon einmal da war… 

Professor McGonagall beendete einen Zauberspruch, der sehr viel älter als Latein klang, und setzte sich dann wieder. 

„In Ordnung“, sagte sie in einer ruhigen Stimme. „Niemand belauscht uns.“ Ihre Gesichtszüge waren sehr angespannt. 

\emph{Ach stimmt, sie geht davon aus, dass ich sie erpresse, um an Informationen über die Prophezeiung zu gelangen.} 

Naja, dafür würde Harry ein andermal Zeit haben. 

„Es geht um den Zwischenfall mit dem Sprechenden Hut“, sagte Harry. (Professor McGonagall blinzelte.) „Ähm… ich glaube, dass ein zusätzlicher Zauberspruch auf dem Sprechenden Hut liegt, irgendwas, wovon der Sprechende Hut selbst nichts weiß, etwas, was ausgelöst wird, wenn der Sprechende Hut ‚Slytherin‘ sagt. Ich habe eine Nachricht gehört, die ziemlich sicher nicht für die Ohren von Ravenclaws gedacht ist. Sie erklang in dem Moment, als ich den Sprechenden Hut von meinem Kopf nahm und spürte, wie die Verbindung abbrach. Es klang wie ein Zischen und wie Englisch zugleich“ – Professor McGonagall zog scharf die Luft ein – „und es lautete: ‚Ssssei gegrüssssst von Ssslytherin zu Sssslytherin. Wenn Du meine Geheimnissse ssuchsst, sssprich mit meiner Ssschlange.‘“ 

Professor McGonagall saß mit offenem Mund da und starrte Harry an, als wären ihm zwei zusätzliche Köpfe gewachsen. 

„Also…“, sagte Professor McGonagall langsam, so als ob sie die Worte nicht glauben konnte, die gerade aus ihrem eigenen Mund kamen, „haben Sie sich dafür entschieden, sofort zu mir zu kommen und mir davon zu berichten.“ 

„Nun, ja, natürlich“, sagte Harry. Er brauchte ja nicht zuzugeben, wie lange es gedauert hatte, bis er darauf gekommen war. „Anstatt, sagen wir, selbst nachzuforschen oder es irgendwelchen anderen Kindern zu erzählen.“ 

„Ich… verstehe“, sagte Professor McGonagall. „Und falls Sie zufällig den Eingang zu Salazar Slytherins legendärer Kammer des Schreckens finden; einen Eingang, den Sie und Sie alleine öffnen können…“ 

„Dann würde ich den Eingang schließen und Ihnen sofort davon berichten, sodass ein Team erfahrener magischer Archäologen zusammengestellt werden kann“, erwiderte Harry sofort. „Dann würde ich den Eingang erneut öffnen und sie würden sehr vorsichtig eintreten um sicherzustellen, dass dort nichts Gefährliches lauert. Ich würde vielleicht später eintreten um mich umzusehen, oder falls sie mich brauchen, um etwas anderes zu öffnen, doch erst nachdem das Gebiet freigegeben wurde und sie Fotos davon gemacht haben, wie alles aussah, bevor die Leute anfangen, auf ihrer unersetzlichen historischen Stätte herumzutrampeln.“ 

Professor McGonagall saß mit offenem Mund da und starrte ihn an, als hätte er sich gerade in eine Katze verwandelt. 

„Es ist offensichtlich, wenn man kein Gryffindor ist“, sagte Harry sanft. 

„Ich glaube“, sagte Professor McGonagall mit gepresster Stimme, „dass Sie \emph{weit} unterschätzen, wie selten gesunder Menschenverstand ist, Mr~Potter.“ 

Das hörte sich richtig an. Obwohl… „Ein Hufflepuff hätte das Gleiche gesagt.“ 

McGonagall stutzte. \emph{„Das} ist wahr.“ 

„Der Sprechende Hut hat mir Hufflepuff angeboten.“ 

Sie blinzelte ihn an, als ob sie ihren eigenen Ohren nicht traute. \emph{„Tatsächlich?“} 

„Ja.“ 

„Mr~Potter“, sagte McGonagall, nun mit gesenkter Stimme, „fünf Jahrzehnte ist es her, dass zuletzt ein Schüler innerhalb der Mauern von Hogwarts starb, und ich bin mir nun sicher, dass es fünf Jahrzehnte her ist, dass zuletzt jemand diese Nachricht hörte.“ 

Es schauderte Harry. „Dann werde ich \emph{sehr} genau aufpassen, dass ich \emph{absolut} nichts diesbezüglich tue, ohne mit Ihnen Rücksprache zu halten, Professor McGonagall.“ Er überlegte. „Darf ich außerdem vorschlagen, dass Sie die besten Leute zusammenrufen, die Sie kriegen können, und dann schauen, ob es möglich ist, diesen zusätzlichen Spruch vom Sprechenden Hut zu entfernen … und wenn das nicht geht, fügen Sie vielleicht \emph{noch einen} Spruch hinzu, einen Quietus, der vorübergehend aktiviert wird, gerade wenn der Hut vom Kopf eines Schülers genommen wird. Bitte sehr, keine toten Schüler mehr.“ Harry nickte zufrieden. 

Professor McGonagall sah noch schockierter aus, wenn das überhaupt vorstellbar wäre. „Ich kann Ihnen \emph{unmöglich} genug Punkte dafür geben, ohne den Hauspokal gleich an Ravenclaw zu verleihen.“ 

„Ähm“, sagte Harry. „Ähm. Ich würde lieber nicht \emph{so} viele Hauspunkte bekommen.“ 

Nun sah Professor McGonagall ihn seltsam an. „Warum nicht?“ 

Harry fiel es etwas schwer, das in Worte zu fassen. „Weil es einfach zu schade wäre, wissen Sie? Wie… wie damals, als ich in der Muggelwelt in die Schule gegangen bin. Wann immer eine Gruppenarbeit stattfand, habe ich stattdessen alles alleine gemacht, weil die anderen mich nur abgebremst hätten. Ich finde es okay, viele Punkte zu bekommen, mehr als jeder andere sogar, aber wenn ich genug bekomme, um ganz alleine die Häuserwertung zu gewinnen, dann ist das so, als ob ich das Haus Ravenclaw auf meinen Schultern trage, und das ist zu schade.“ 

„Ich verstehe…“, sagte McGonagall zögerlich. Es war offensichtlich, dass dieser Gedanke ihr nie gekommen war. „Angenommen, ich würde Ihnen also nur fünfzig Punkte geben?“ 

Harry schüttelte wieder den Kopf. „Es ist den anderen Kindern gegenüber nicht fair, wenn ich jede Menge Punkte für erwachsene Sachen verdiene, die ich machen kann, aber sie nicht. Wie sollte Terry Boot fünfzig Punkte dafür verdienen, dass er von einem Flüstern berichtet, was er vom Sprechenden Hut gehört hat? Das wäre absolut nicht fair.“ 

„Ich verstehe, warum der Sprechende Hut Ihnen Hufflepuff angeboten hat“, sagte Professor McGonagall. Sie beäugte ihn mit einem seltsamen Respekt. 

Das schnürte Harry etwas die Kehle zu. Er hatte wirklich gedacht, dass er Hufflepuff nicht verdient hätte. Dass der Sprechende Hut nur versucht hatte, ihn nicht nach Ravenclaw zu schicken, sogar lieber in ein Haus, dessen Werte er gar nicht besaß… 

Professor McGonagall lächelte nun. „Und wenn ich versuchte, Ihnen \emph{zehn} Punkte zu geben…?“ 

„Können Sie erklären, woher diese zehn Punkte stammen, wenn jemand nachfragt? Es könnte einige Slytherins geben – und ich meine nicht die Kinder auf Hogwarts – die wirklich, \emph{wirklich} wütend wären, wenn sie wüssten, dass der Spruch vom Sprechenden Hut genommen wurde und herausfänden, dass ich beteiligt war. Ich denke, Vorsicht ist an dieser Stelle besser als Nachsicht. Nichts zu danken, Ma'am, die Tugend ist ihr eigener Lohn.“ 

„So ist es“, sagte Professor McGonagall, „aber ich werde Ihnen etwas anderes sehr Besonderes geben. Ich sehe, dass ich Ihnen in Gedanken großes Unrecht getan habe, Mr~Potter. Bitte warten Sie hier.“ 

Sie stand auf, ging zur verschlossenen Tür an der Rückwand, wedelte mit dem Zauberstab und eine Art verschwommener Vorhang erschien um sie herum. Harry konnte weder sehen noch hören, was geschah. Einige Minuten später verschwand diese Verschleierung und Professor McGonagall stand dort, ihm gegenüber, und die Tür hinter ihr sah aus, als wäre sie nie offen gewesen. 

Und Professor McGonagall hielt in einer Hand eine Halskette; eine dünne goldene Kette mit einem silbernen Ring, in dessen Mitte eine Sanduhr eingearbeitet war. In ihrer anderen Hand war ein Faltblatt. „Das ist für Sie“, sagte sie. 

Wow! Er würde irgendeinen praktischen magischen Gegenstand als Belohnung erhalten! Offenbar funktionierte die Taktik, monetäre Entlohnung abzulehnen, bis einem ein magischer Gegenstand angeboten wurde, auch im wirklichen Leben, nicht nur in Computerspielen. 

Harry nahm seine neue Kette lächelnd an sich. „Was ist es?“ 

Professor McGonagall atmete tief ein. „Mr~Potter, dies ist ein Gegenstand, der normalerweise nur an Kinder verliehen wird, die sich bereits als außerordentlich verantwortungsbewusst herausgestellt haben, um ihnen mit komplizierten Stundenplänen zu helfen.“ McGonagall zögerte, als ob sie noch etwas hinzufügen wollte. „Ich \emph{muss} betonen, Mr~Potter, dass die wahre Natur dieses Gegenstands \emph{geheim} ist und dass Sie \emph{keinem} anderen Schüler davon erzählen und sich nicht bei der Benutzung beobachten lassen dürfen. Wenn Sie damit nicht einverstanden sind, dann können Sie es jetzt zurückgeben.“ 

„Ich kann Geheimnisse für mich behalten“, sagte Harry. „Also, was tut es?“ 

„Soweit es die anderen Schüler betrifft, ist dies ein Spirndlertörchen, das benutzt wird um ein seltenes, nicht ansteckendes magisches Leiden namens Spontane Duplikation zu heilen. Sie tragen es unter Ihrer Kleidung und während Sie keinen Grund haben, es irgendjemandem zu zeigen, haben Sie ebensowenig einen Grund, es furchtbar geheim zu halten. Spirndlertörchen sind nicht interessant. Verstehen Sie, Mr~Potter?“ 

Harry nickte mit breiter werdendem Lächeln. Er spürte, dass hier ein \emph{kompetenter} Slytherin am Werk gewesen war. „Und was tut es wirklich?“ 

„Es ist ein Zeitumkehrer. Jede Umdrehung des Stundenglases schickt Sie eine Stunde in die Zeit zurück. Wenn Sie es also benutzen, um jeden Tag zwei Stunden zurück zu reisen, dann sollte es Ihnen gelingen, immer zur gleichen Zeit einzuschlafen.“ 

Harrys Bemühungen, die magische Welt zu akzeptieren, scheiterten diesmal vollkommen.

\emph{Sie geben mir eine Zeitmaschine, um meine Einschlafstörungen zu behandeln.}

\emph{Sie geben mir eine \emph{Zeitmaschine}, um meine \emph{Einschlafstörungen} zu behandeln.}

\emph{Sie \shout{geben mir eine Zeitmaschine} um \shout{meine Einschlafstörungen zu behandeln}.}

„Äh…äh…äh…äh…“, sagte Harrys Mund. Er hielt die Kette nun von sich weg, als ob es sich um eine tickende Bombe handelte. Nun, nein, nicht, als ob es sich um eine tickende Bombe handelte, das wurde dem Ernst der Lage \emph{nicht mal annähernd} gerecht. Harry hielt die Kette von sich weg, als ob es eine Zeitmaschine wäre.

\emph{Sagen Sie, Professor McGonagall, wussten Sie, dass zeit-verkehrte normale Materie genau wie Antimaterie aussieht? Nun, ja, so ist es! Wussten Sie, dass ein Kilogramm Antimaterie, wenn es auf ein Kilogramm Materie trifft, mit der Energie von 43 Millionen Tonnen TNT explodiert? Ist Ihnen klar, dass ich selbst 41 Kilogramm wiege und dass die folgende Explosion \shout{einen riesigen rauchenden Krater hinterlassen würde, wo vorher Schottland war?}} 

„Entschuldigen Sie“, schaffte Harry zu sagen, „aber das klingt wirklich, wirklich, \emph{wirklich, \shout{WIRKLICH GEFÄHRLICH!}“} Harrys Stimme wurde nicht ganz zu einem Kreischen; er konnte unmöglich laut genug schreien, um dieser Situation gerecht zu werden, also würde er es gar nicht erst versuchen. 

Professor McGonagall sah ihn mit geduldigem Wohlwollen an. „Ich bin froh, dass Sie das ernst nehmen, Mr~Potter, aber Zeitumkehrer sind nicht \emph{so} gefährlich. Wenn sie es wären, würden wir sie nicht an Kinder herausgeben.“ 

„Wirklich“, sagte Harry. „Ahahahaha. Natürlich würden Sie keine Zeitmaschinen an Kinder herausgeben, wenn sie gefährlich wären, was \emph{habe} ich da bloß gedacht? Nur um das klar zu stellen, wenn ich dieses Gerät anniese, wird es mich nicht ins Mittelalter befördern, wo ich mit einer Pferdekutsche Gutenberg überfahren und so das Zeitalter der Aufklärung verhindern könnte? Denn, wissen Sie, ich hasse es, wenn mir so etwas passiert.“ 

McGonagalls Lippen zitterten gerade so wie immer, wenn sie versuchte nicht zu lächeln. Sie hielt Harry das Faltblatt hin, doch Harry hielt die Kette weiterhin vorsichtig mit beiden Händen von sich weg und starrte das Stundenglas an um sicherzustellen, dass es sich nicht drehte. „Keine Sorge“, sagte McGonagall nach einer kurzen Pause, als klar wurde, dass Harry sich nicht bewegen würde, „das kann unmöglich passieren, Mr~Potter. Der Zeitumkehrer kann nicht benutzt werden, um mehr als sechs Stunden in die Vergangenheit zu reisen. Er kann nicht mehr als sechs Mal an einem Tag verwendet werden.“ 

„Oh, das ist gut, sehr gut. Und wenn jemand mich anrempelt, wird der Zeitumkehrer auch \emph{nicht} zerbrechen und wird folglich \emph{nicht} das gesamte Schloss Hogwarts in einer unendlichen Zeitschleife aus Donnerstagen gefangenhalten.“ 

„Nun, sie \emph{können} sehr empfindlich sein…“, sagte McGonagall. „Und ich glaube, ich habe von seltsamen Dingen gehört, die passieren, wenn sie zerbrechen. Aber nichts \emph{dergleichen!“} 

„Vielleicht“, sagte Harry als er wieder sprechen konnte, „sollten Sie Ihre Zeitmaschinen mit einer Art \emph{Schutzhülle} versehen, statt \emph{das Glas so offen rumzutragen,} damit \emph{so etwas nicht passieren kann.“} 

McGonagall sah sehr überrascht aus. „Das ist eine exzellente Idee, Mr~Potter. Ich werde das Ministerium davon in Kenntnis setzen.“ 

\emph{Das war's, jetzt ist es offiziell, sie haben es im Parlament beschlossen, jeder in der Zaubererwelt ist vollkommen bescheuert.} 

„Und ich will ja wirklich nicht \shout{rumdiskutieren“} – Harry versuchte angestrengt, seine hysterische Stimme etwas zu senken – „aber hat irgendjemand mal darüber nachgedacht, was es \shout{bedeutet,} sechs Stunden durch die Zeit zurückzureisen und irgendetwas zu ändern, wodurch so ziemlich \shout{alle betroffenen Personen gelöscht} und \shout{durch andere Versionen von} –“ 

„Oh, man kann die Vergangenheit nicht \emph{ändern“,} unterbrach Professor McGonagall ihn. „Um Himmels Willen, Mr~Potter, glauben Sie, Schüler dürften die benutzen, wenn \emph{das} möglich wäre? Was, wenn jemand versucht, seine Klausurergebnisse zu ändern?“ 

Harry brauchte einen Moment um das zu verarbeiten. Seine Hände lockerten den verkrampften Griff um die Kette ein kleines bisschen. Als ob er keine Zeitmaschine, sondern nur einen tickenden nuklearen Sprengkopf hielt. 

„Also…“, sagte Harry langsam. „Die Menschen stellen fest, dass das Universum… sich irgendwie als selbstkonsistent herausstellt, obwohl es Zeitreisen enthält. Wenn ich und mein zukünftiges Ich interagieren, dann werde ich die gleichen Dinge aus beiden Perspektiven sehen, obwohl selbst bei meinem ersten Durchlauf mein zukünftiges Ich bereits von Ereignissen weiß, die aus meiner eigenen Sicht noch gar nicht passiert sind…“ Harrys Stimme verlor sich in den Unzulänglichkeiten der englischen Sprache. 

„Das stimmt, denke ich“, sagte Professor McGonagall. „Allerdings \emph{wird} Zauberern empfohlen, sich nicht von ihrem vergangenen Ich sehen zu lassen. Wenn Sie beispielsweise zwei Fächer zur selben Zeit belegen und ihren eigenen Weg kreuzen, dann sollte die erste Version von ihnen beiseite treten und zu einer bestimmten Zeit die Augen schließen – Sie haben schon eine Armbanduhr, sehr gut – sodass Ihr zukünftiges Ich vorbeigehen kann. Das steht alles hier im Faltblatt drin.“ 

„Ahahahaa. Und was passiert, wenn jemand diese Empfehlung \emph{missachtet?“} 

Professor McGonagall schürzte ihre Lippen. „Ich habe gehört, dass das äußerst verstörend sein kann.“ 

„Und es kreiert nicht, sagen wir, ein Paradoxon, welches das Universum zerstört.“ 

Sie lächelte geduldig. „Mr~Potter, ich denke, ich hätte davon gehört, wenn \emph{das} jemals passiert wäre.“ 

\shout{„Das macht die Sache nicht besser! Habt ihr Leute jemals von dem anthropischen Prinzip gehört? Und was für ein Idiot hat zum ersten mal eines von diesen Dingern gebaut?“} 

Professor McGonagall lachte tatsächlich. Es war ein angenehmes, frohes Geräusch, dass in diesem ernsten Gesicht überraschend fehl am Platz wirkte. „Sie haben wieder einen ‚Sie-haben-sich-in–eine-Katze–verwandelt‘-Moment, Mr~Potter. Sie wollen das vermutlich nicht hören, aber es ist ganz entzückend, wie süß sie wirken.“ 

„Sich in eine Katze zu verwandeln, kommt dem hier nicht einmal \shout{nahe.} Wissen Sie, bis vorhin hatte ich irgendwo in meinem Hinterkopf die schreckliche, verdrängte Befürchtung, dass die einzige verbliebene Antwort sei, dass mein Universum eine Computersimulation wie in dem Buch \emph{Simulacron 3} ist, doch jetzt \emph{ist selbst das ausgeschlossen,} weil dieses kleine Spielzeug \emph{NICHT TURING-BERECHENBAR IST!} Eine Turingmaschine könnte simulieren, wie man an einen bestimmten Punkt in der Zeit zurückkehrt, und dann von dort eine andere Zukunft berechnen; eine Orakel-Turingmaschine könnte sich auf das Halteverhalten von Maschinen niedrigerer Ordnung verlassen; aber was Sie sagen ist, dass die Realität irgendwie in einem Durchlauf selbstkonsistent berechnet wird und dabei auf Informationen zurückgreift, die noch nicht… geschehen… sind…“ 

Die Erkenntnis versetzte Harry einen Schlag. 

Jetzt ergab alles Sinn. Jetzt ergab \emph{endlich} alles Sinn. 

\shout{„So funktioniert der Seltsaft also! Natürlich!} Seine Magie \emph{erzwingt} nicht, dass komische Dinge passieren, sie bringt einen nur dazu, ihn \emph{trinken zu wollen,} wenn ohnehin komische Dinge passieren werden! Ich bin so ein Narr, ich hätte es schon bemerken sollen, als ich vor Dumbledores zweiter Rede Lust bekam, den Seltsaft zu trinken, ihn \emph{nicht} getrunken habe und mich dann stattdessen an meiner eigenen Spucke verschluckt habe – den Seltsaft zu trinken führt nicht zu etwas Lächerlichem; das Lächerliche führt dazu, dass man den Seltsaft trinkt! Ich habe erkannt, dass die beiden Ereignisse miteinander korrelieren und war davon ausgegangen, dass der Seltsaft der Grund und das Lächerliche die Folge sein musste, weil ich dachte, dass zeitliche Abfolge die Kausalität beschränkt und dass kausale Graphen azyklisch sein müssen, \shout{aber es ergibt alles Sinn, sobald Man die Folgepfeile \emph{entgegen der Zeitrichtung} zeichnet!“}

Eine \emph{weitere} Erkenntnis versetzte Harry einen Schlag. 

Dieses Mal schaffte er es ruhig zu bleiben und machte nur ein kleines ersticktes Geräusch, wie von einem sterbenden Kätzchen, als ihm klar wurde, wer heute Morgen den Zettel an sein Bett geklebt hatte. 

Professor McGonagalls Augen leuchteten auf. „Nachdem Sie Ihre Abschlussprüfungen bestanden haben, oder vielleicht schon davor, \emph{müssen} Sie wirklich einige dieser Muggeltheorien auf Hogwarts lehren, Mr~Potter. Die klingen sehr faszinierend, obwohl sie alle falsch sind.“ 

„Glahhrgh…“ 

Professor McGonagall sagte einige weitere Nettigkeiten; verlangte ihm ein paar weitere Versprechen ab, die Harry benickte; sagte etwas darüber, dass er nicht mit Schlangen reden solle, wenn jemand es hören könnte; erinnerte ihn daran, das Flugblatt zu lesen und dann fand Harry sich irgendwie vor ihrem Büro wieder, die Tür fest hinter ihm geschlossen. 

„Gaahhhrrrraa…“, sagte Harry. 

Nun, ja, er \emph{war} überrascht. 

Nicht zuletzt aufgrund der Tatsache, dass, wenn dieser Scherz nicht gewesen wäre, er womöglich niemals überhaupt einen Zeitumkehrer erhalten hätte. 

Oder hätte Professor McGonagall ihm den ohnehin gegeben, bloß etwas später am Tag, wenn er Gelegenheit gehabt hätte, wegen seiner Schlafstörungen nachzufragen oder ihr von der Nachricht des Sprechenden Hutes zu erzählen? Und hätte er sich dann, zu diesem Zeitpunkt, dazu entschieden, sich selbst einen Streich zu spielen, der dazu geführt hätte, dass er den Zeitumkehrer \emph{früher} bekam? Sodass die einzige \emph{selbstkonsistente} Möglichkeit die war, dass der Streich schon begonnen hatte, bevor er am Morgen aufgewacht war…? 

Harry stellte fest, dass er zum ersten Mal in seinem Leben in Erwägung zog, dass die Antwort auf seine Frage wortwörtlich \emph{unvorstellbar} war. Dass sein Gehirn – da es aus Neuronen bestand, die nur in positive Zeitrichtung funktionierten – tatsächlich \emph{nichts} tun könnte, keine Berechnung durchführen könnte, die den Auswirkungen eines Zeitumkehrers entsprach. 

Bis zu diesem Moment hatte Harry dem Hinweis von E. T. Jaynes Glauben geschenkt, dass Unkenntnis bezüglich eines Phänomens eine Aussage über das eigene Wissen war, nicht über das Phänomen selbst; dass seine Unsicherheit Teil von ihm war und nicht Teil dessen, worüber er sich Gedanken machte; dass Unwissen im Kopf, aber nicht in Wirklichkeit existiert; dass eine leere Karte keinem leeren Land entsprach. Es gab mysteriöse Fragen, doch eine mysteriöse Antwort war ein Widerspruch in sich. Ein heiliges Mysterium zu verehren hieß, das eigene Unwissen zu verehren. 

Also hatte Harry die Zaubererwelt kennengelernt und sich geweigert, eingeschüchtert zu sein. Menschen hatten kein Gefühl für Geschichte; sie lernten Chemie und Biologie und Astronomie und dachten, dass diese Dinge schon ewig Kern der Wissenschaft waren, dass sie nie rätselhaft \emph{gewesen waren.} Die Sterne waren einst Rätsel gewesen. Lord Kelvin hatte einst das Leben und die Biologie – wie Muskeln dem menschlichen Willen gehorchen und Bäume aus Samen entstehen – ein Rätsel genannt, dass „unendlich jenseits“ der Möglichkeiten der Wissenschaft liege. (Nicht etwa ein bisschen jenseits, sondern \emph{unendlich} jenseits. Es hatte Lord Kelvin einen richtigen Kick gegeben, \emph{etwas nicht zu wissen.)} Jedes Mysterium, das je gelöst worden war, war von Anbeginn der Menschheit ein Rätsel gewesen – bis jemand es schließlich gelöst hatte. 

Nun sah er sich zum ersten Mal einem Mysterium gegenüber, das \emph{unlösbar} zu sein drohte. Wenn Zeit sich nicht durch azyklische kausale Netzwerke beschreiben ließ, dann verstand Harry nicht, was mit Ursache und Wirkung gemeint war; und wenn Harry Ursache und Wirkung nicht verstand, dann verstand er nicht, woraus die Wirklichkeit sonst bestehen könnte; und es war sehr wohl möglich, dass sein menschliches Gehirn es niemals verstehen \emph{konnte,} da sein Gehirn aus \emph{altmodischen Neuronen} bestand, für die Zeit \emph{linear} verlief – was sich als eine ärmliche Teilmenge der Wirklichkeit herausgestellt hatte. 

Aber immerhin hatte er festgestellt, dass es für den Seltsaft, der einst allmächtig und vollkommen unglaublich gewirkt hatte, eine viel einfachere Erklärung gab. Die er \emph{nur deswegen} übersehen hatte, weil die Wahrheit vollkommen außerhalb seines Hypothesenraumes und jenseits von allem anderen, was sein Gehirn jemals zu verstehen gelernt hatte, lag. Doch nun \emph{hatte} er es tatsächlich herausgefunden – vermutlich. Was in gewisser Weise ermutigend war. In gewisser Weise. 

Harry blickte auf seine Uhr. Es war fast 11~Uhr morgens, er war letzte Nacht um 1~Uhr eingeschlafen und würde in der nächsten Nacht also normalerweise um 3~Uhr einschlafen. Damit er also bereits 22~Uhr einschlafen und 7~Uhr aufwachen konnte, müsste er insgesamt fünf Stunden zurückreisen. Das hieß, wenn er ungefähr 6~Uhr morgens in seinem Schlafsaal sein wollte, bevor irgendjemand anders wach war, sollte er sich lieber beeilen und… 

Selbst im \emph{Nachhinein} verstand Harry nicht mal bei der \emph{Hälfte} der Ereignisse, wie er sie zustande bekommen hatte. Wo war der \emph{Kuchen} hergekommen? 

Harry begann, sich wirklich vor Zeitreisen zu fürchten. 

Andererseits musste er zugeben, dass es \emph{tatsächlich} eine unwiederbringliche Gelegenheit gewesen war. Einen Streich wie diesen konnte man sich selbst nur einmal im Leben spielen, nämlich innerhalb von sechs Stunden nachdem man zum ersten Mal von Zeitumkehrern gehört hatte. 

Und das machte es sogar \emph{noch} verwirrender, wenn Harry genauer darüber nachdachte. Die Zeit hatte ihm den Streich als \emph{fait accompli} serviert und dennoch war es ganz eindeutig seiner eigenen Hände Arbeit. Konzept, Durchführung und Handschrift. Bis in jedes kleinste Detail, obwohl er viele davon immer noch nicht verstand. 

Nun, die Uhr tickte und ein Tag dauerte höchstens dreißig Stunden. Harry wusste wie er \emph{manche} Sachen tun konnte und den Rest, wie zum Beispiel den Kuchen, würde er halt zwischendurch herausbekommen müssen. Es aufzuschieben brachte nichts. Er konnte ja nichts tun, solange er hier in der \emph{Zukunft} feststeckte. 

\later 

Fünf Stunden früher schlich Harry sich in seinen Schlafsaal und hatte seinen Umhang über den Kopf gezogen um sich ein bisschen zu tarnen, nur für den Fall, dass jemand bereits aufgestanden war und ihn rumlaufen sah, während Harry noch im Bett lag. Er wollte niemandem sein kleines medizinisches Problem mit der Spontanen Duplikation erklären müssen. 

Glücklicherweise schliefen anscheinende alle noch. 

Außerdem lag neben seinem Bett offenbar ein Päckchen, das in grünem und rotem Papier eingepackt und mit einem hellen, goldenen Band verschnürt war. Ein vollkommen typisches Weihnachtsgeschenk, obwohl es nicht Weihnachten war. 

Harry schlich so leise er konnte durch's Zimmer, nur für den Fall, dass jemand den Quietus-Zauber auf eine niedrige Stufe eingestellt hatte. 

An dem Päckchen war ein Briefumschlag befestigt, der mit einfachem, normalem Wachs ohne ein eingedrücktes Siegel verschlossen war. 

Harry öffnete den Umschlag vorsichtig und zog einen Brief hervor. Dieser lautete: 

\begin{writtenNote}
Dies ist der Unsichtbarkeitsumhang von Ignotus Peverell, der seinen Nachkommen, den Potters, vererbt wurde. Im Gegensatz zu anderen Umhängen oder Zaubersprüchen hat er die Macht, dich zu \emph{verstecken} und nicht nur unsichtbar zu machen. Dein Vater lieh ihn mir kurz vor seinem Tod für Nachforschungen aus und ich gebe zu, dass er mir über die Jahre gute Dienste erwiesen hat.

In Zukunft werde ich mit Unsichtbarkeitszaubern auskommen müssen, fürchte ich. Es ist an der Zeit, dass der Umhang zu Dir, seinem rechtmäßigen Erben, zurückfindet. Ich hatte überlegt, ihn dir als Weihnachtsgeschenk zukommen zu lassen, doch er wollte bereits früher in deine Hand zurückkehren. Er scheint zu erwarten, dass du ihn benötigen wirst. Gebrauche ihn klug.

Zweifelsohne denkst du bereits über allerlei wunderbare Streiche nach, wie dein Vater sie zu seiner Zeit gespielt hat. Wenn all seine Missetaten bekannt würden, würden alle Mädchen aus Gryffindor sich sammeln, um sein Grab zu entweihen. Ich werde nicht zu verhindern versuchen, dass die Ereignisse sich wiederholen, doch sei \emph{besonders} vorsichtig, dass niemand davon erfährt. Wenn Dumbledore eine Chance sähe, eines der Heiligtümer des Todes zu besitzen, dann würde er bis zum Tag seines Todes nicht davon ablassen.

Frohe Weihnachten dir.
\end{writtenNote}

Der Zettel war nicht unterschrieben. 

\later 

„Wartet kurz“, sagte Harry plötzlich und blieb stehen, als die anderen Jungs gerade den Ravenclaw-Schlafsaal verließen. „Entschuldigt, ich muss noch etwas mit meinem Koffer machen. Ich komme in ein paar Minuten zum Frühstück nach.“ 

Terry Boot blickte Harry finster an. „Du hast hoffentlich nicht vor, in unseren Sachen rumzukramen.“ 

Harry hob eine Hand. „Ich schwöre, dass ich nichts dergleichen mit irgendwelchen Sachen von euch vorhabe; dass ich vorhabe, nur auf Sachen von mir selbst zuzugreifen; dass ich nicht vorhabe, irgendjemandem von euch einen Streich oder etwas ähnliches zu spielen und dass ich nicht davon ausgehe, dass diese Absichten sich ändern, bevor ich zum Frühstück in der Großen Halle erscheine.“ 

Terry runzelte die Stirn. „Warte, ist das –“ 

„Keine Sorge“, sagte Penelope Clearwater, die dabei war, um sie zum Frühstück zu führen. „Das war lückenlos. Gut formuliert, Potter, du solltest Rechtsanwalt werden.“ 

Harry Potter blinzelte bei diesen Worten. Ah, ja, Ravenclaw-Vertrauensschülerin. „Danke“, sagte er. „Vermutlich.“ 

„Wenn du nach der Großen Halle suchst, wirst du dich verlaufen.“ Penelope betonte das wie einen schlichten, unbestreitbaren Fakt. „Sobald du das tust, frage ein Porträt wie du in den ersten Stock kommst. Frage ein weiteres Porträt \emph{sobald} du vermutest, dass du dich wieder verlaufen hast. \emph{Insbesondere} dann, wenn es so scheint, als ob du höher und höher kommst. Wenn es so scheint, als ob du über dem gesamten Schloss bist, \emph{halte an} und warte auf Suchtrupps. Ansonsten werden wir dich vier Monate später wiedersehen und du wirst fünf Monate älter sein und einen Lendenschurz tragen und mit Schnee bedeckt sein und \emph{das setzt voraus, dass du im Schloss bleibst.“} 

„Verstanden“, sagte Harry schluckend. „Ähm, sollte man den Schülern all solche Sachen nicht sofort sagen?“ 

Penelope seufzte. „Was, \emph{alles} das? Das würde Wochen dauern. Im Laufe der Zeit bekommt man das schon mit.“ Sie drehte sich um und ging, gefolgt von den anderen Schülern. „Wenn ich dich nicht in dreißig Minuten beim Frühstück sehe, Potter, dann beginne ich die Suche.“ 

Sobald alle gegangen waren, befestigte Harry den Zettel an seinem Bett – er hatte ihn und alle anderen Zettel bereits geschrieben, während er im Untergeschoss seines Koffers gearbeitet hatte bevor alle anderen aufgestanden waren. Dann griff er vorsichtig in den Quietus-geschützten Bereich und zog den Unsichtbarkeitsumhang wieder von der schlafenden Form von Harry-1 runter. 

Und nur um der Verwirrung willen steckte Harry den Umhang in den Beutel von Harry-1, womit er wusste, dass dieser längst in seinem eigenen Beutel wäre. 

\later 

„Ich kann dafür sorgen, dass die Nachricht an Cornelion Flubberwalt weitergeleitet wird“, sagte das Gemälde eines Mannes mit aristokratischer Aura, jedoch vollkommen normaler Nase. „Aber darf ich fragen, wo sie \emph{ursprünglich} herkommt?“ 

Harry zuckte mit kunstvoller Hilflosigkeit die Schultern. „Mir wurde gesagt, dass sie von einer tonlosen Stimme gesprochen wurde, die aus einer Kluft mitten im Raum erschallte; einer Kluft, die sich über einem feurigen Abgrund auftat.“ 

\later 

„Hey!“, sagte Hermine in empörtem Ton von ihrem Sitzplatz auf der anderen Seite des Frühstückstisch. „Das ist \emph{unser aller} Nachtisch! Du kannst nicht einfach einen ganzen Kuchen nehmen und in deinen Beutel stecken!“ 

„Ich nehme nicht einen Kuchen, ich nehme zwei. Tut mir Leid, Leute, ich hab's eilig!“ Harry ignorierte die empörten Rufe und verließ die Große Halle. Er musste zum Kräuterkunde-Unterricht etwas früher erscheinen. 

\later 

Professor Sprout sah ihn scharf an. „Und woher wissen \emph{Sie,} was die Slytherins vorhaben?“ 

„Ich kann meine Quelle nicht nennen“, sagte Harry. „Ich muss Sie sogar bitten, so zu tun, als ob dieses Gespräch nie stattgefunden hat. Tun Sie einfach so, als ob Sie rein zufällig vorbeikommen; etwas besorgen müssen, oder so. Ich werde vorausrennen, sobald Kräuterkunde zu Ende ist. Ich denke, ich kann die Slytherins ablenken, bis Sie dort sind. Ich lasse mich nicht leicht einschüchtern oder drangsalieren und ich denke, die werden sich nicht trauen, den Jungen, der lebt, ernsthaft zu verletzen. Allerdings… ich möchte Sie nicht bitten, in den Gängen zu rennen, aber ich würde es schätzen, wenn Sie unterwegs nicht bummeln.“ 

Professor Sprout sah ihn lange an, dann besänftigte sich ihr Gesichtsausdruck. „Passen Sie bitte auf sich auf, Harry Potter. Und… danke sehr.“ 

„Achten Sie nur darauf, nicht zu spät zu kommen“, sagte Harry. „Und denken Sie dran, wenn Sie dort ankommen, dass Sie mich dort nicht erwartet haben und dieses Gespräch nie stattgefunden hat.“ 

\later 

Es war schrecklich, sich selbst dabei zuzusehen, wie man Neville aus dem Kreis der Slytherins rauszerrte. Neville hatte Recht gehabt, er hatte zu viel Kraft verwendet, viel zu viel Kraft. 

„Hallo“, sagte Harry Potter kühl. „Ich bin der Junge, der lebt.“ 

Acht Erstklässler, ungefähr gleich groß. Einer von ihnen hatte eine Narbe auf der Stirn und er verhielt sich nicht wie die anderen. 

\vskip 0pt plus 4\baselineskip\settowidth{\versewidth}{Ach, gäb' uns höh're Macht das Glück} \begin{verse}[\versewidth] Uns selbst zu sehen mit andrem Blick.\\ Dies hielt' von Fehlern uns zurück\\ Und Vorurteilen. \end{verse}

Professor McGonagall hatte Recht. Der Sprechende Hut hatte Recht. Es war offensichtlich, wenn man es von außen sah. 

Irgendetwas stimmte nicht mit Harry Potter. 
