\chapter{Selbstbeherrschung}

\emph{„Ich frage mich, was mit ihm los ist.“}

\later

\lettrinepara[ante=„]{T}{urpin,} Lisa!“

Flüster flüster flüster harry potter flüster flüster slytherin flüster flüster nein wirklich was zum kuckuck flüster flüster …

„RAVENCLAW!“

Harry stimmte in den Applaus ein, der das junge Mädchen begrüßte, als es schüchtern zum Ravenclaw-Tisch ging, während sich der Saum seiner Robe in ein dunkles Blau umfärbte. Lisa Turpin sah aus, als sei sie hin- und hergerissen zwischen ihrem Impuls, sich so weit wie möglich von Harry Potter wegzusetzen, und ihrem Impuls hinüberzurennen, sich in seine Nachbarschaft zu drängen und anzufangen, Antworten aus ihm herauszuholen.

Im Mittelpunkt von außergewöhnlichen und seltsamen Ereignissen zu stehen und dann in das Haus Ravenclaw gesteckt zu werden war ungefähr so, als würde man in Grillsoße getunkt und dann in ein Loch voller hungriger Kätzchen gesteckt.

„Ich habe dem Hut versprochen, nicht darüber zu reden“, flüsterte Harry zum n-ten Mal.

„Ja, wirklich.“

„Nein, ich hab dem Hut \emph{wirklich} versprochen, nicht darüber zu reden.“

„In Ordnung, ich habe dem Hut versprochen, über das \emph{meiste} nicht zu reden und der Rest ist \emph{privat,} so wie \emph{ bei euch}, also \emph{hört auf zu fragen.“}

„Du willst wissen, was passiert ist? In Ordnung! Hier ist ein Teil von dem, was passiert ist! Ich habe dem Hut gesagt, dass Professor McGonagall damit gedroht hat, ihn anzuzünden, und er sagte mir, ich solle ihr ausrichten, dass sie ein kleiner Flegel ist und ihre Nase gefälligst nicht in seine Angelegenheiten stecken soll!“

„Wenn du mir nicht glaubst, was ich sage, \emph{warum fragst du dann überhaupt?“}

„Nein, ich weiß auch nicht, wie ich den Dunklen Lord besiegt habe! Sag du es mir, wenn du es rausbekommen hast!“

\emph{„Ruhe!“,} schrie Professor McGonagall am Pult vor dem großen Lehrertisch. \emph{„Seid ruhig, bis die Zeremonie abgeschlossen ist!“}

Die Lautstärke nahm kurz ab, während alle abwarteten, ob sie irgendwelche konkreten und glaubhaften Drohungen aussprechen würde. Dann begann das Flüstern von neuem.
Dann stand der Uralte mit silbernem Bart auf und lächelte verschmitzt.
Sofortige Ruhe. Irgendjemand stieß Harry hektisch mit dem Ellbogen, als er versuchte weiterzuflüstern, und Harry schnitt sich mitten im Satz das Wort ab.
Dumbledore setzte sich wieder.

\emph{Notiz an mich selbst: Lege dich niemals mit Dumbledore an.}

Harry versuchte immer noch zu verarbeiten, was alles mit ihm unter dem Hut passiert war – nicht zuletzt das, was in dem Moment passiert war, als er den Hut vom Kopf nahm. In diesem Moment war ein leises Flüstern wie aus dem Nichts ertönt, etwas, das sich auf eine seltsame Art wie Englisch angehört hatte und gleichzeitig ein Zischen war, etwas, das gesagt hatte: „Ssssei gegrüssssst von Ssslytherin zu Sssslytherin. Wenn Du meine Geheimnissse ssuchsst, sssprich mit meiner Ssschlange.“

Harry nahm irgendwie an, dass dies nicht zum offiziellen Ablauf der Einteilung gehörte. Und dass es ein bisschen Extramagie war, die Salazar Slytherin bei der Herstellung des Hutes hinterlassen hatte. Und dass der Hut selbst nichts davon wusste. Und dass es ausgelöst wurde, wenn der Hut „SLYTHERIN“ sagte, plus oder minus ein paar andere Bedingungen. Und dass ein Ravenclaw wie er selbst es \emph{wirklich, wirklich nicht hätte hören sollen.} Und falls er einen verlässlichen Weg fände, Draco zur Geheimhaltung zu verpflichten, damit er ihm Fragen dazu stellen könnte, wäre das ein besonders guter Zeitpunkt, um etwas Seltsaft parat zu haben.

\emph{Junge, du hast dir vorgenommen, nicht den Weg des Dunklen Lords zu gehen und das Universum fängt an mit dir rumzuspielen, sobald du den Hut vom Kopf nimmst. An manchen Tagen lohnt es sich einfach nicht gegen seine Bestimmung zu kämpfen. Vielleicht warte ich einfach bis morgen mit meinem Vorsatz, kein Dunkler Lord zu sein.}

„GRYFFINDOR!“

Ron Weasley bekam eine \emph{Menge} Applaus und nicht nur von Gryffindors. Anscheinend war die Weasley-Familie allgemein beliebt hier. Nach einem Moment lächelte Harry und begann mit den anderen zu applaudieren.

Andererseits gab es keinen besseren Zeitpunkt als heute, um sich von der Dunklen Seite abzuwenden.

Zum Teufel mit der Bestimmung, zum Teufel mit dem Universum. Er würde es diesem Hut schon zeigen.

„Zabini, Blaise!“

Pause.

„SLYTHERIN!“, rief der Hut.

Harry applaudierte Zabini auch, ohne die seltsamen Blick zu beachten, die alle anderen ihm zuwarfen, Zabini eingeschlossen.

Danach wurde kein anderer Name mehr aufgerufen und Harald bemerkte, dass sich „Zabini, Blaise“ so anhörte, als ob das Alphabet fast zu Ende war. Großartig, nun hatte er also \emph{nur} Zabini applaudiert … Na ja.

Dumbledore stand wieder auf und begann sich in Richtung Podium zu bewegen. Es sah so aus, als ob sie gleich mit einer Rede beglückt würden –

Harry kam plötzlich ein brillantes Experiment in den Sinn.

Hermine hatte gesagt, dass Dumbledore der mächtigste lebende Zauberer war, richtig?
Harry griff in seinen Beutel und flüsterte: „Seltsaft.“

Wenn der Seltsaft funktionierte, müsste er Dumbledore dazu bringen, in seiner Rede etwas so absolut Lächerliches zu sagen, dass es Harry, der sich mental auf alles vorbereitet hatte, \emph{trotzdem} zum Verschlucken brachte. So etwas wie: Alle Hogwarts-Schüler dürfen das ganze Schuljahr über keine Kleidung tragen, oder jeder würde in eine Katze verwandelt.

Falls es \emph{irgendjemanden auf der Welt} gab, der dem Seltsaft wiederstehen könnte, dann war es Dumbledore. Wenn das also funktionierte, war der Seltsaft buchstäblich \emph{unbesiegbar.}

Harry öffnete den Verschluß der Dose unter dem Tisch, da er etwas unauffällig sein wollte. Die Dose gab ein leises Zischen von sich, ein paar Köpfe drehten sich und sahen ihn an, aber sie schauten auch schnell wieder weg als –

„Willkommen! Willkommen zu einem neuen Jahr in Hogwarts!“, sagte Dumbledore und strahlte die Studenten mit weit geöffneten Armen an, als ob ihm nichts mehr Freude bereiteten könnte, als sie hier alle zu sehen.

Harry nahm den ersten Schluck vom Seltsaft und setzte die Dose wieder ab. Er würde die Limo langsam und in kleinen Schlucken trinken und sich dabei nicht verschlucken, egal \emph{was} Dumbledore sagte –

„Bevor wir mit unserem Bankett beginnen, möchte ich ein paar Worte sagen. Und hier sind sie: Fröhlich, fröhlich, bumm bumm, Sumpf, Sumpf, Sumpf! Vielen Dank!“

Alle applaudierten und jubelten, während Dumbledore sich wieder hinsetzte.

Harry saß wie festgefroren da, während die Limo langsam aus seinen Mundwinkeln lief. Er hatte es \emph{zumindest} geschafft, sich \emph{leise} zu verschlucken.

Er hätte dies wirklich, wirklich, \emph{wirklich} nicht machen sollen. Bemerkenswert, wie \emph{viel} offensichtlicher das jetzt wurde, wo es \emph{eine Sekunde zu spät} war.

Im Rückblick hätte er wohl bemerken sollen, dass etwas falsch lief als er daran dachte, dass alle in Katzen verwandeln werden … oder er hätte sogar schon vorher an seine Notiz denken sollen, sich nicht mit Dumbledore anzulegen … oder seinen neu gefassten Beschluss, mehr auf andere zu achten … oder wenn er nur \emph{einen Funken gesunden Menschenverstand} hätte …

Es war hoffnungslos. Er war durch und durch verdorben. Es lebe der Dunkle Lord Harry. Man konnte nicht gegen sein Schicksal ankämpfen.

Jemand fragte Harry, ob mit ihm alles in Ordnung war. (Die anderen begannen damit, sich Essen zu nehmen, das magischerweise auf dem Tisch erschienen war, egal.)

„Mir geht's gut“, sagte Harry. „Entschuldigt mich. Ähm. War das eine … \emph{normale} Ansprache vom Schulleiter? Ihr schient alle … nicht sehr … überrascht …“

„Oh, Dumbledore ist verrückt, ganz klar“, sagte ein älter aussehender Ravenclaw, der neben ihm saß und sich mit einem Namen vorgestellt hatte, an den Harry sich überhaupt nicht erinnern konnte. „Sehr umgänglich, unglaublich mächtiger Zauberer, aber komplett abgedreht.“ Er pausierte. „Irgendwann später würde ich dich gerne fragen, warum grüne Flüssigkeit aus deinem Mund kam und dann verschwand, obwohl ich davon ausgehe, dass du dem Hut versprochen hast, auch darüber nicht zu reden.“

Mit großem Aufwand hielt sich Harry davon ab, auf die verräterische Dose Seltsaft in seiner Hand zu schauen.

Genau genommen hatte der Seltsaft die Quibblerschlagzeile über ihn und Draco nicht einfach \emph{erscheinen} lassen. Draco hatte es so erklärt, als ob es alles … einfach so passiert ist. Als ob der Saft die \emph{Geschichte so verändert, dass alles passt.}

Harry stellte sich vor, wie er seinen Kopf gegen den Tisch schlug. \emph{Bums, bums, bums} machte sein Kopf in seinen Gedanken.

Eine andere Schülerin senkte ihre Stimme zu einem Flüstern. „Ich habe gehört, Dumbledore ist insgeheim eine Graue Eminenz, und dass er eine Menge Sachen kontrolliert und den Wahnsinn als Tarnung benutzt, damit ihn niemand verdächtigt.“

„Ich hab das auch gehört“, flüsterte ein dritter Schüler, und es gab verstohlenes Genicke überall am Tisch.

Das konnte unmöglich Harrys Aufmerksamkeit entgehen.

„Ich verstehe“, flüsterte Harry, und senkte seine eigene Stimme. „Also weiß jeder, das Dumbledore insgeheim eine Graue Eminenz ist.“

Die meisten Schüler nickten. Ein oder zwei sahen plötzlich nachdenklich aus, inklusive dem älteren Schüler, der neben Harry saß.

\emph{Seid ihr sicher, dass dies hier der Ravenclaw-Tisch ist?,} hätte Harry fast laut gefragt.

„Brillant!“, flüsterte Harry. „Wenn es jeder weiß, wird niemand vermuten, dass es ein Geheimnis ist!“

„Genau“, flüsterte ein Schüler, dann grübelte er. „Warte mal, das hört sich nicht so ganz richtig an –“

\emph{Notiz an mich selbst: Das 75. Perzentil von Hogwarts, auch bekannt als das Haus Ravenclaw, ist nicht das weltweit exklusivste Programm für begabte Kinder.}

Aber zumindest hatte er heute etwas Wichtiges gelernt. Der Seltsaft war allmächtig. Und \emph{das} bedeutete …

Harry blinzelte überrascht, als sein Kopf endlich die offensichtliche Verbindung herstellte.

… \emph{das} bedeutete, sobald er einen Zauberspruch lernte, der zeitweise seinen Sinn für Humor veränderte, konnte er \emph{alles} geschehen lassen, indem er die Dinge so arrangierte, dass er \emph{nur diese eine Sache} überraschend genug fand, um sich zu verschlucken, und dann eine Dose Seltsaft trank.

\emph{Also, das war ein kurzer Weg zur Omnipotenz. Sogar ich hatte erwartet, dass es länger dauert als mein erster Schultag.}

Wenn er genau drüber nachdachte, hatte er außerdem ganz Hogwarts zerlegt, keine zehn Minuten nachdem er in ein Haus gekommen war.

Harry fühlte etwas Bedauern darüber – weiß Merlin, was ein verrückter Schulleiter in seinen nächsten sieben Schuljahren anstellen würde – aber er konnte es auch nicht vermeiden, ein kleines bisschen Stolz zu empfinden.

Morgen. Allerspätestens morgen würde er den Pfad verlassen, der zum Dunklen Lord Harry führte. Eine Vorstellung, die sich mit jeder Minute schlimmer anhörte.

Und doch irgendwie immer attraktiver. Ein Teil seiner Gedanken stellte sich schon die Uniformen seiner Untergebenen vor.

„Iss“, brummte der ältere Schüler neben ihm und stieß Harry in die Rippen. „Nicht denken! Essen!“

Harry began automatisch seinen Teller mit den Sachen vollzuladen, die gerade vor ihm standen. Blaue Würstchen mit kleinen leuchtenden Stücken, egal.

„Woran dachtest Du, etwa an –“, began Padma Patil, eine andere Ravenclaw-Erstklässlerin.

„Kein Herumfragen beim Essen!“, riefen mindestens drei Leute. „Hausregel!“, fügte ein weiterer hinzu. „Ansonsten verhungern wir hier alle.“

Harry begann wirklich, wirklich zu hoffen, dass seine clevere Idee nicht \emph{tatsächlich} funktionieren würde. Und dass der Seltsaft auf eine andere Weise funktionierte und nicht \emph{wirklich} die Allmacht besaß, die Wirklichkeit zu verändern. Es war nicht so, dass er nicht allmächtig sein \emph{wollte.} Es war nur, dass er den Gedanken nicht ertragen konnte, in einem Universum zu leben, das tatsächlich so funktionierte. Es hatte etwas \emph{Unwürdiges,} durch die clevere Verwendung eines Softdrinks aufzusteigen.

Aber er \emph{würde} es experimentell austesten.

„Weißt Du“, sagte der ältere Schüler neben ihm in einem durchaus angenehmen Tonfall, „wir haben eine Methode, um Leute wie dich zum Essen zu zwingen. Möchtest du herausfinden, wie wir es machen?“

Harry gab auf und begann die blauen Würstchen zu essen. Sie waren durchaus gut, ganz besonders die leuchtenden Stellen.

Das Abendessen ging erstaunlich schnell vorbei. Harry versuchte zumindest ein bisschen von all den Gerichten auszuprobieren, die er sah. Seine Neugier konnte den Gedanken nicht ertragen, \emph{nicht zu wissen,} wie etwas schmeckte. Glücklicherweise war dies kein Restaurant, wo man eine Sache bestellten musste und nie herausfand, wie all die anderen Sachen auf der Karte schmeckten. Harry \emph{hasste} das, es war wie eine Folterkammer für jeden mit einem Quentchen Neugier: \emph{Lerne nur eines der Mysterien auf dieser Karte kennen. Ha ha ha!}

Dann war es Zeit für den Nachtisch und Harry hatte völlig vergessen, dafür Platz zu lassen. Er gab auf, nachdem er ein kleines Stückchen Siruptorte probiert hatte. Sicherlich würde es alle diese Sachen im Laufe des Schuljahres mindestens ein weiteres Mal geben.
Also, was war auf seiner Liste zu erledigender Sachen, neben den normalen Schulangelegeheiten?

\emph{Nr. 1: Erforsche Gedanken-verändernde Sprüche, so dass du den Seltsaft ausprobieren kannst und herausfindest, ob du wirklich einen Weg zur Allmacht gefunden hast. Genaugenommen, erforsche einfach alle Arten von Gedanken-Magie, die du finden kannst. Das Gehirn ist die Grundlage unserer Kräfte als Menschen, jegliche Art von Magie, die darauf Einfluss nimmt, ist die wichtigste Art, die es gibt.}

\emph{Nr. 2: Eigentlich ist dies hier Nr. 1, und das andere Nr. 2. Gehe durch die Bücherregale der Bibliotheken von Hogwarts und Ravenclaw, präge dir die Sortierung ein und stelle sicher, dass du zumindest die Buchtitel alle gelesen hast. Zweiter Durchgang: Lies alle Inhaltsverzeichnisse. Sprich dich mit Hermine ab, die ein viel besseres Gedächtnis hat als du. Finde heraus, ob es auf Hogwarts Fernleihe gibt und sorge dafür, dass ihr beide, besonders Hermine, diese Bibliotheken auch besuchen könnt. Wenn die anderen Häuser eigene Bibliotheken haben, finde heraus, wie man da legal reinkommt oder sich hineinschleicht.}

\emph{Möglichkeit 3a: Schwöre Hermione auf Geheimhaltung ein und versuche ‚Von Slytherin zu Slytherin: Wenn Du meine Geheimnisse suchst, sprich mit meiner Schlange.‘ zu erforschen. Problem: Das hört sich hochgradig geheim an und es könnte durchaus eine Weile dauern, bis wir da zufällig auf ein Buch mit einem Hinweis stoßen.}

\emph{Nr. 0: Finde heraus, welche Arten von Informations-Such-und-Heraushol-Sprüchen es gibt, falls es sowas überhaupt gibt. Bibliotheksmagie ist nicht von so ultimativer Bedeutung wie Gehirnmagie, hat aber eine höhere Priorität.}

\emph{Möglichkeit 3b: Suche nach einem Spruch um Draco Malfoys Geheimhaltung magisch abzusichern oder überprüfe mit Magie die Ehrlichkeit seines Versprechens, ein Geheimnis zu bewahren (Veritaserum?), und befrage ihn dann über Slytherins Nachricht …}

Eigentlich … hatte Harry ein ziemlich ungutes Gefühl bei Möglichkeit 3b.

Jetzt, wo Harry darüber nachdachte, fand er Möglichkeit 3a auch nicht mehr so gut.

Harrys Gedanken sprangen zurück zum möglicherweise schlimmsten Moment seines bisherigen Lebens, diese langen Sekunden voll unerträglichem Schrecken unter dem Hut, als er dachte, er sei schon gescheitert. Er hatte sich damals gewünscht, nur ein paar Minuten in der Zeit zurückzugehen und etwas zu ändern, irgendwas, bevor es zu spät war …

Und dann stellte sich heraus, dass es doch noch nicht zu spät war.

Wunsch erfüllt.

Man konnte die Geschichte nicht verändern, aber man konnte es gleich von Anfang an richtig machen. Gleich beim \emph{ersten} Mal etwas anders machen.

Diese ganze Sache mit der Suche nach Slytherins Geheimnissen … schien sehr genau wie die Sorte Angelegenheiten, auf die man Jahre später zurückblickte und sich sagte: ‚Und \emph{seitdem} fing alles an, falsch zu laufen.‘

Und er würde sich verzweifelt die Möglichkeit wünschen, durch die Zeit zurück zu reisen und eine andere Entscheidung zu treffen …

Wunsch erfüllt. Und nun?

Harry lächelte vorsichtig.

Es war ein eher \emph{kontraintuitiver} Gedanke … aber …

Aber er \emph{könnte} – es gab keine Regel, die sagte, er könne nicht – er \emph{könnte} einfach so tun, als ob er das kleine Flüstern nie gehört hätte. Sollen die Dinge doch genau so weitergehen, wie sie es tun würden, wenn dieser eine entscheidende Moment nie stattgefunden hätte. Zwanzig Jahre später würde er sich sehnsüchtigst wünschen, dass es zwanzig Jahre zuvor genau so geschehen wäre, und zwanzig Jahre vor in zwanzig Jahren war glücklicherweise gerade jetzt. Die ferne Vergangenheit zu verändern war einfach, man musste nur zur rechten Zeit daran denken.

Oder … das war \emph{noch} kontraintuitiver … er könnte sogar jemandem Bescheid sagen, sagen wir \emph{Professor McGonagall}, statt Draco \emph{oder} Hermine. Und sie könnte ein paar gute Leute zusammentrommeln, die diesen kleinen Zusatzspruch vom Hut entfernten.

Ja, warum nicht. Das hörte sich wie eine \emph{bemerkenswert} gute Idee an, nun da Harry tatsächlich einmal daran \emph{gedacht} hatte.

So offensichtlich im Nachhinein, und doch waren ihm Möglichkeit 3c und Möglichkeit 3d irgendwie einfach nicht eingefallen.

Harry gab sich selbst +1 Punkt in seinem Anti-Dunkler-Lord-Harry-Programm.

Es war ein schlimmer Streich, den der Hut ihm da gespielt hatte, aber vom konsequentialistischen Standpunkt aus konnte man dem Ergebnis schwerlich widersprechen. Es gab ihm sicherlich auch einen besseren Einblick in die Sicht eines Opfers.

\emph{Nr. 4: Bei Neville Longbottom entschuldigen.}

In Ordnung, er war hier auf einem guten Weg, jetzt musste er nur noch dabei bleiben. \emph{Ich werde mit jedem Tag in jeder Hinsicht immer heller und heller …}

Die Leute um Harry herum hatten nun größenteils auch mit dem Essen aufgehört, und die Schüsseln mit Nachtisch begannen zu verschwinden, ebenso das benutzte Geschirr.

Als alles verschwunden war, erhob sich Dumbledore von neuem aus seinem Sitz.

Harry verspürte den Drang, einen weiteren Seltsaft zu trinken.

\emph{Du WILLST mich wohl verarschen,} dachte Harry zu diesem Teil seiner selbst.

Aber das Experiment zählte nicht, wenn man es nicht wiederholte, richtig? Und der Schaden war längst angerichtet, oder? Wollte er nicht sehen, was \emph{dieses} Mal passieren würde? War er nicht \emph{neugierig?} Was, wenn er diesmal ein anderes Ergebnis erhielt?

\emph{Hey, ich wette du bist der selbe Teil von meinem Gehirn, der mich zu diesem Streich mit Neville Longbottom angestiftet hat.}

Ähm, vielleicht?

\emph{Und ist es nicht total offensichtlich, dass ich es eine Sekunde nachdem es zu spät ist bereuen werde?}

Ähm...

\emph{Genau. Also: NEIN!}

„Ähm“, sagte Dumbledore vorne am Pult und strich sich über seinen langen silbernen Bart. „Nur noch ein paar Worte, nun nachdem wir alle gut gefüttert und gewässert sind. Ich habe ein paar Mitteilungen zum Schuljahresbeginn für euch.“

„Die Erstklässler sollten beachten, dass der Wald auf unseren Ländereien für alle Schüler verboten ist. Daher heißt er Verbotener Wald. Wenn er erlaubt wäre, würde man ihn den Erlaubten Wald nennen.“

Ganz klar. \emph{Notiz an mich selbst: Verbotener Wald ist verboten.}

„Außerdem hat mich Mr. Filch, der Hausmeister, gebeten, euch daran zu erinnern, dass in den Pausen auf den Gängen nicht gezaubert werden darf. Nun wissen wir alle, dass was sein \emph{sollte} und was \emph{ist} zwei verschiedene Dinge sein können. Vielen Dank, dass ihr das beachtet.“

Ähm …

„Die Quidditch-Auswahl findet in der zweiten Woche des Schuljahrs statt. Alle, die gerne in den Hausmannschaften spielen wollen, mögen sich an Madam Hooch wenden. Alle, die das gesamte Quidditchspiel verändern wollen, mögen sich an Harry Potter wenden.“

Harry verschluckte sich an seinem eigenen Speichel und bekam gerade in dem Moment einen Hustenanfall, als alle Augen sich auf ihn richteten. Wie zum \emph{Teufel}! Er hatte Dumbledore zu keinem Zeitpunkt direkt in die Augen gesehen … zumindest \emph{dachte} er das. Er hatte zu der Zeit ganz sicher nicht über Quidditch nachgedacht! Er hatte mit niemandem außer Ron Weasley gesprochen und er \emph{dachte} nicht, dass Ron es irgendjemand anderem erzählt haben würde … oder war Ron zu einem Lehrer gelaufen, um sich zu beschweren? \emph{Wie} um alles in der Welt …

„Und schließlich muss ich euch mitteilen, dass in diesem Jahr das Betreten des Korridors im dritten Stock, der in den rechten Flügel führt, allen verboten ist, die nicht einen sehr schmerzhaften Tod sterben wollen. Er wird von einer Reihe ausgefeilter, gefährlicher und potentiell tödlicher Fallen bewacht, und ihr könnt unmöglich an ihnen allen vorbeikommen, insbesondere dann, wenn ihr gerade erst im ersten Schuljahr seid.“

Harry war zu diesem Zeitpunkt wie betäubt.

„Und abschließend möchte ich meinen größten Dank an Quirinus Quirrell richten, der sich heldenhafterweise bereit erklärt hat, die Position des Lehrers für Verteidigung gegen die Dunklen Künste zu übernehmen.“ Dumbledores Blick bewegte sich suchend über seine Schüler. „Ich hoffe, dass alle Schüler Professor Quirrell den größtmöglichen Respekt und die \emph{Toleranz} entgegenbringen, die sein außergewöhnlicher Dienst an euch und dieser Schule verdient, und dass ihr uns \emph{nicht} mit \emph{irgendwelchen Nörgeleien} über ihn \emph{belästigen} werdet, es sei denn \emph{ihr} wollt seinen Posten übernehmen.“

Was war \emph{das} denn?

„Ich übergebe nun das Wort an unseren neuen Lehrer, Professor Quirrell, der ein paar Worte sagen möchte.“

Der junge, dünne und nervöse Mann, den Harry zuerst im Tropfenden Kessel getroffen hatte, schritt langsam zum Rednerpult und schaute furchtsam in alle Richtungen. Harry erhaschte einen Blick auf seinen Hinterkopf und es sah aus, als ob Professor Quirrell bereits kahl würde, trotz seiner scheinbaren Jugend.

„Ich frage mich, was mit \emph{ihm} los ist“, flüsterte der älter aussehende Schüler neben Harry. Ähnliches Geflüster wurden an anderen Stellen des Tisches ausgetauscht.

Professor Quirrell erreichte das Pult und stand blinzelnd dort. „Äh …“, sagte er. „Äh …“ Dann schien ihn sein Mut schlagartig zu verlassen und er stand schweigend dort und zuckte gelegentlich zusammen.

„Oh, super“, flüsterte der ältere Schüler. „Das sieht nach einem weiteren \emph{langen} Jahr im Verteidigungsunterricht aus.“

„Seid gegrüßt, meine jungen Lehrlinge“, sagte Professor Quirrell in einem trockenen, selbstsicheren Tonfall. „Wir wissen alle, dass Hogwarts bei der Besetzung dieser Stelle ein gewisses \emph{Pech} hatte, und zweifelsohne fragen sich viele von euch schon, welches Unglück mich dieses Jahr ereilen wird. Ich versichere euch, dieses Unglück wird nicht meine Inkompetenz sein.“ Er lächelte dünn. „Ob ihr es glaubt oder nicht, ich wollte mich schon seit langem eines Tages als Lehrer für Verteidigung gegen die Dunklen Künste hier an der Hogwarts-Schule für Hexerei und Zauberei versuchen. Der erste, der dieses Fach unterrichtete, war Salazar Slytherin hochstpersönlich, und bis in das vierzehnte Jahrhundert hinein war es üblich, dass die größten Kampfmagier jeglicher Gesinnung sich als Lehrer versuchten. Unter den ehemaligen Verteidigungslehrern sind nicht nur der legendäre wandernde Held Harold Shea, sondern auch die sogenannte \emph{unsterbliche} Baba Yaga, ja, ich sehe, einige von euch zittern immer noch beim Klang ihres Namens, obwohl sie schon seit sechshundert Jahren tot ist. Das muss damals eine interessante Zeit auf Hogwarts gewesen sein, meint ihr nicht?“

Harry schluckte schwer und versuchte, die plötzlich aufkommenden Gefühle zu unterdrücken, die ihn überwältigten, als Professor Quirrell anfing zu reden. Die präzise Diktion erinnerte ihn sehr an einen Dozenten in Oxford und es begann ihm zu dämmern, dass er sein Zuhause, seine Mutter und seinen Vater bis Weihnachten nicht sehen würde.

„Ihr seid daran gewöhnt, dass der Posten mit Inkompetenten, Tunichtguten oder Unglückseligen besetzt wird. Für jeden mit einem Sinn für Geschichte hat er jedoch eine ganz andere Reputation. Nicht jeder, der hier unterrichtete, gehörte zu den Besten, aber die Besten haben alle hier in Hogwarts unterrichtet. In so einer erhabenen Gesellschaft und nach so langer Zeit, die ich auf diesen Tag gewartet habe, würde ich mich schämen, einen geringeren Maßstab als Perfektion anzulegen. Und so habe ich vor, dass jeder von euch, wenn er an dieses Schuljahr zurückdenkt, sich an den \emph{besten} Verteidigungsunterricht erinnert, den ihr je gehabt habt. Was ihr dieses Jahr lernt, wird euch für immer als ein festes Fundament in der Kunst der Verteidigung dienen, egal, wer euch zuvor oder danach unterrichtet hat.“

Professor Quirrels Gesichtsausdruck wurde ernst. „Wir haben eine \emph{Menge} verlorenen Stoff nachzuholen, und nicht viel Zeit, um das zu tun. Daher habe ich vor, von den Lehrkonventionen in Hogwarts in verschiedenen Punkten abzuweichen, und auch einige freiwillige Zusatzangebote zu machen.“ Er hielt kurz inne. „Wenn das nicht genügt, finde ich eventuell ganz neue Wege, um euch zu motivieren. Ihr seid meine langerwarteten Schüler und ihr \emph{werdet} in meinem lang erwarteten Verteidigungsunterricht euer \emph{Bestes} geben. Ich würde noch ein paar furchtbare Drohungen anfügen wie ‚Oder ihr werdet schrecklich leiden‘, aber das wäre so klischeehaft, meint ihr nicht auch? Ich schätze mich dafür, eine größere Vorstellungskraft zu besitzen. Vielen Dank.“

Dann schien die Kraft und Selbsicherheit Professor Quirrell zu verlassen. Sein Mund stand weit offen, als ob er sich plötzlich vor einem unerwarteten Publikum wiederfand, und er drehte sich mit einem krampfhaften Zucken um und schlurfte zurück zu seinem Sitz, nach vorne gebeugt, als ob er gleich in sich zusammenfallen und implodieren würde.

„Er scheint etwas seltsam“, flüsterte Harry.

„Pah“, sagte der älter aussehende Schüler. „Du hast gar keine Ahnung.“

Dumbledore ging wieder ans Podium.

„Und nun“, sagte er, „bevor wir zu Bett gehen, singen wir die Schulhymne! Jeder nach seiner Lieblingsmelodie und seinem Lieblingstext. Los geht's!“
